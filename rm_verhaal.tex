\documentclass[10pt,twoside, openright]{memoir}
% \documentclass[12pt,twoside,openright]{memoir}
%\usepackage{createspace}
%\usepackage[size=pocket,noicc]{createspace}
\usepackage[paperwidth=14.81cm, paperheight=20.99cm,bindingoffset=0.5in]{geometry}
%\usepackage[T1]{fontenc}
\usepackage[utf8]{inputenc}

\usepackage[section]{placeins}
% \usepackage{flafter}

\usepackage[dutch]{babel}
\usepackage{mathpazo}
\usepackage[export]{adjustbox}

%\usepackage{mathpazo}
\usepackage[protrusion=true,expansion=true]{microtype}
%\usepackage{type1cm}
%\usepackage{lettrine}
\usepackage{xcolor,graphicx}
\selectcolormodel{gray}

% \setlength\midchapskip{10pt}
% \makechapterstyle{VZ23}{
%   \renewcommand\sectionnamenum{}
% \renewcommand\printchaptername{}
% \renewcommand\chapnumfont{\Huge\bfseries\centering}
% \renewcommand\chaptitlefont{\Huge\scshape\centering}
% \renewcommand\afterchapternum{%
%   \par\nobreak\vskip\midchapskip\hrule\vskip\midchapskip}
% \renewcommand\printchapternonum{%
%   \vphantom{\chapnumfont \thechapter}
%   \par\nobreak\vskip\midchapskip\hrule\vskip\midchapskip}
% }
% \chapterstyle{VZ23}

\usepackage{kpfonts}
\setSingleSpace{1.1}
\SingleSpacing
\usepackage{xcolor,calc}
\definecolor{chaptercolor}{gray}{0.8}
\usepackage[T1]{fontenc}
\newcommand\numlifter[1]{\raisebox{-2cm}[0pt][0pt]{\smash{#1}}}
\newcommand\numindent{\kern37pt}
\newlength\sectiontitleboxheight
\makechapterstyle{hansen}{
  \renewcommand\printchaptername{\raggedleft}
  \renewcommand\printchapternum{%
    \begingroup%
    \leavevmode%
    \chapnumfont%
    \strut%
    \numlifter{\thechapter}%
    \numindent%
\endgroup }
  \renewcommand*{\printchapternonum}{%
    \vphantom{\begingroup%
      \leavevmode%
      \chapnumfont%
      \numlifter{\vphantom{9}}%
      \numindent%
      \endgroup}
    \afterchapternum}
  \setlength\midchapskip{0pt}
  \setlength\beforechapskip{0.5\baselineskip}
  \setlength{\afterchapskip}{3\baselineskip}
  \renewcommand\chapnumfont{%
    \fontsize{4cm}{0cm}%
    \bfseries%
    \sffamily%
    \color{chaptercolor}%
  }
  \renewcommand\chaptitlefont{%
    \normalfont%
    \huge%
    \bfseries%
    \raggedleft%
  }%
  \settototalheight\sectiontitleboxheight{%
    \parbox{\textwidth}{\chaptitlefont \strut bg\\bg\strut}}
  \renewcommand\printchaptertitle[1]{%
    \parbox[t][\sectiontitleboxheight][t]{\textwidth}{%
      %\microtypesetup{protrusion=false}% add this if you use microtype
      \chaptitlefont\strut ##1\strut}%
} }
\chapterstyle{hansen}

\usepackage[font=scriptsize,font=it,labelfont=bf, hang]{caption}

\setlength{\belowcaptionskip}{0pt}

%\checkandfixthelayout

% See the ``Memoir customise'' template for some common customisations
% Don't forget to read the Memoir manual: memman.pdf

%\title{Mijn Eerste Jaren}
%\author{Ruud Meijns}
%\date{} % Delete this line to display the current date

% BEGIN TITLE

\makeatletter
\def\maketitle{%
  \null
  \thispagestyle{empty}%
  \vfill
  \begin{center}\leavevmode
    \normalfont
    {\LARGE\raggedleft \@author\par}%
    \hrulefill\par
    {\huge\raggedright \@title\par}%
    \vskip 1cm
%    {\Large \@date\par}%
  \end{center}%
  \vfill
  \null
  \cleardoublepage
  }
\makeatother
\author{Ruud Meijns}
\author{Ruud Meijns}
\title{Omkijken}
\date{}


\newlength{\drop}% for my convenience


% BEGIN DOCUMENT

\newcommand*{\titleS}{\begingroup% Scripts, T&H p 151
\drop = 0.1\textheight
\centering
\vspace*{\drop}

\begin{figure}
\includegraphics[width=\textwidth]{img/titelpag3}
\end{figure}

% \vfill
% \rule{0.4\textwidth}{0.4pt}\\[\baselineskip]
% {\large\itshape Zaanse Verhalen \\[0.5cm] 2016}\par
% \vspace*{\drop}
\endgroup}

\begin{document}

\pagestyle{plain}
\let\cleardoublepage\clearpage
\thispagestyle{empty}
\titleS




\frontmatter

\null\vfill
\thispagestyle{empty}
\begin{flushleft}
\textit{Omkijken}\\
\vfil
Zaandam 
2016
\end{flushleft}
\clearpage

\let\cleardoublepage\clearpage
\clearpage
\setcounter{tocdepth}{0}
\tableofcontents

\clearpage
~
\clearpage


\mainmatter
\sloppy


\label{deel1}
\chapter{Mijn Eerste Jaren}

% \epigraph{\emph{Het leven is niet het leven dat je hebt geleefd,
% maar dat je je herinnert en hoe je het je herinnert
% om het te vertellen.}}{Gabriel Garcia Marquez}

% \epigraph{\emph{La réalité ne se forme que dans la mémoire}}{Marcel Proust}

\thispagestyle{empty}
\begin{flushright}
\begin{figure}
\includegraphics[width=0.8\textwidth, right]{img/epi/epi1}
\end{figure}
\end{flushright}

\thispagestyle{empty}
\section*{Familie Meijns} % (fold)
\label{cha:familie_meijns}


Onze familie is, voor zover ik me dat kan herinneren, geen familie-familie geweest. Geen grote hechte clan waarvan alle leden altijd om elkaar heen draaiden. Ik heb dan ook weinig herinneringen aan mijn grootouders. Dat zal ook te maken hebben gehad met het feit dat ik een nakomertje was, geboren in 1946. Mijn oudere broers Gerard en Harry (geboren in respectievelijk 1930 en 1937) hebben hen veel langer meegemaakt. 

Opa en oma Meijns woonden in een huisje in de Jasykoffstraat. Ik sta nog op foto’s met hen voor het huis, dus ik ben bij ze geweest. Maar de herinnering is vaag. Later, toen oma overleden was woonde opa Meijns bij tante Lena in de Czaar Peterstraat, waar de familie Duif een wasserij had. Ome Jaap Duif was al overleden. (Ik kan me geen begrafenissen herinneren.)

\begin{figure}
\includegraphics{img/ch1/omaJasykofstr}
\centering
\caption*{\footnotesize Oma Meijns in de Jasykoffstraat}
\end{figure}

Het is nooit een zonder meer hartelijke familie geweest. Ik logeerde bijvoorbeeld nooit bij mijn grootouders, want mijn vader leefde met zijn vader in onmin. We gingen als gezin niet veel met ze om vanwege `de kwestie'. Niet dat het vechten of schreeuwen was, maar er was wel bedekt wantrouwen.

De kwestie was al aanwezig toen ik ter wereld kwam. Ik heb altijd gedacht dat dat te maken had met de communistische overtuiging van mijn vader. Dat speelde later inderdaad een rol tijdens de Russische overval op de Hongaren in 1956 toen de familie bij monde van tante Rina liet weten, te breken met mijn vader, en dus met ons hoewel het persoonlijk aan mijn vader gericht was. Maar de oorspronkelijke wrok kwam door een erfkwestie over de sigarenzaak van opa aan de Zuiddijk, waarom ook niet? De kwestie was in mijn jeugd al een beetje naar de achtergrond verdwenen, maar tegen die tijd waren oma en opa Meijns al overleden.

\begin{figure}
\centering
\includegraphics[width=\textwidth]{img/ch1/famMeijns2}
\caption*{\footnotesize Achter: moe, tante Lena, ome Cas, tante Nel en tante Riena. Zittend: Pa met Harry, oma en opa Meijns. Voor: Gerard, Elly Berends en Jaap. Harry is nog klein, dus deze is voor de Tweede Wereldoorlog genomen}
\end{figure}

De keren dat ik bij Tante Lena logeerde waren wel aardig, ik had geen weet van enige problemen. Ik sliep boven in een logeerkamertje, direct onder de pannen. Op een kastje stond een lampetkan met een ovale spiegel en een eigenaardig luchtje (was het kamfer?). 

’s Morgens ontbijt, aan tafel met een zeiltje. Opa at één broodje. Hij vertelde altijd dat hij er twaalf at, omdat tante Lena het ene broodje in twaalf kleine stukjes sneed. 

Om binnen te komen ging iedereen achterom. Eerst de steeg in, dan langs de familie Vonk. Die hadden zo’n kefferig hondje, dat als het de kans kreeg je probeerde aan te vallen, maar als je schreeuwde in elkaar kromp. Vanaf daar kwam je via de keuken achter bij tante Lena binnen. Het geheel werd omgeven door de oude huishoudschool en de houten wasserij.  

Nooit heb ik kunnen denken dat ik later nog eens recht tegenover in diezelfde straat zou wonen en de sloop van dat hele blok zou meemaken.

\begin{figure}
\centering
\includegraphics[width=\textwidth]{img/ch1/oudnieuw}
\caption*{\footnotesize Roel Berends, moe, tante Riena, Janny, Harry, ik, de man van Gre, ome Roel, tante Nel, ome Cas, Carla en tante Lena}
\end{figure} 

Ik kan me ook nog een oudejaarsavond herinneren met alle families: die van ome Cas, tante Nel en dochter Carla, van tante Riena, ome Roel en kleine Roel, tante Lena (was ome Jaap er?) en onze ploeg. Tante Riena was een zuster van m’n vader. Ome Roel was zeekapitein. Een zeer markant figuur. Hij kon vertellen. Of het nu verhalen of moppen waren, als hij ze vertelde dan waren ze goed. Volgens mijn moeder was het daar altijd de zoete inval. Als ome Roel thuis was, was het feest. De Gedempte Gracht, waar ze woonden, had toen nog gescheiden rijwegen en een plantsoentje in het midden. 

Op de TV---er was één kanaal---was een `Ballet der Lage Landen', waarop ome Roel zei dat het waarschijnlijk een ballet der lage handen zou zijn. Ik geloof dat de ruzie toen min of meer was bijgelegd, maar helemaal goed kwam het nooit.  
\nopagebreak

\section*{Opa Meijns} % (fold)
\label{cha:opa_meijns}

Opa Meijns leek op Drees. Een klein gedrongen mannetje, ook met zo’n snorretje. Hij had een aantal wandelstokken met ijzeren plaatjes van plekken waar hij had gewandeld, meestal in Duitsland of Zwitserland.

Opa had een sigarenzaak aan de Zuiddijk tussen kapper Leering en banketbakker Landsaat. Achter het dijkhuis lag een grote lap grond met daarop een zaaltje dat werd verhuurd. Tijdens de Tweede Wereldoorlog werd het soms als onderduik gebruikt. 

Toen opa met pensioen wilde gaan heeft hij met z’n oudste zoon Cas een overeenkomst gesloten waarbij hij de winkel en toebehoren aan hem overdroeg in ruil voor gratis huisvesting in de Jasykoffstraat. (Dit alles volgens de overlevering.) De andere kinderen bleven buiten de regeling en dat zette kwaad bloed. Mijn vader ging niet meer naar zijn vader, hoewel ik nog wel een aantal keren bij ze op bezoek ben geweest. Ik heb dus een eenzijdig beeld van mijn grootvader.

\begin{figure}
\makebox[\textwidth][c]{\includegraphics[width=\textwidth]{img/ch2/ch2-afb01}}
\caption*{\footnotesize Een trouwjubileum (veertig of vijftig jaar) van opa en oma Meijns. Van links naar rechts, achterste rij: ome Jaap Duijf, tante Nel Meijns, tante Rina Berends, Gerard Meijns (pa), Lena Meijns (ma), ome Cas (van Nel) en tante Lena Duijf (van Jaap.
Daarvoor: Gerard Meijns, Jaap Meijns (van Cas en Nel) en Ellie Berends (van Rina en Roel Berends)
Daarvoor: oma Meijns, (kleine) Roel Berends, opa Meijns en Harry Meijns. Grote Roel is afwezig, was kapitein en waarschijnlijk is dit in de Tweede Wereldoorlog en zat hij op zee.}
\end{figure}

\begin{figure}
\makebox[\textwidth][c]{\includegraphics[width=0.55\textheight]{img/ch2/ch2-afb02}}
\caption*{\footnotesize Met opa en oma voor het huis in de Jasykoffstraat}
\end{figure}

\begin{figure}
\makebox[\textwidth][c]{\includegraphics[width=0.55\textheight]{img/ch2/ch2-afb03}}
\caption*{\footnotesize De plakploeg van de SDAP in 1925. Opa staat zit het bord met hoed.}
\end{figure}

Door mijn gesnuffel in de Zaanse geschiedenis ben ik toch wat meer te weten gekomen over m’n grootvader. Hij bleek een actief lid van de Sociaal Democratische Arbeiders Partij (S.D.A.P.) en van de Geheelonthoudersbeweging.  

In dagblad \emph{Het Volk} vond ik advertenties van opa, en zijn nieuwjaarswensen voor 1930. Ook publiceerde hij een stuk waarin hij tegen het misbruik van alcohol op een partijcongres ageerde.

De eerste keer dat ik hem plotseling gewaar werd was toen ik in één van de boekjes \emph{Zaandammers \ldots kent u ze nog?} zat te bladeren. Ik zag opeens m’n opa op de foto met de plakploeg van de S.D.A.P. tijdens de verkiezingen van 1925. In die turbulente jaren waren de plakploegen ook verantwoordelijk voor het kalken op straat. Het werd zelfs zo erg dat er in de gemeenteraad vragen over werden gesteld. Sommige straten stonden zo vol dat het wel leek of het gesneeuwd had.

Toen ik gevraagd werd het boek \emph{Averechts} van Erik Schaap te recenseren kwam ik m’n opa opnieuw tegen. En jawel, daar stond het: ``bij sigarenwinkelier Meijns''. Opa verleende de hoofdpersoon Max Lewin enige tijd onderdak in het gebouwtje achter z’n huis. Daar had deze dus in het geheim de laatste twee maanden van de oorlog doorgebracht.

In de loop der tijd heb ik steeds meer gevonden. Zo ontdekte ik dat hij van vele verenigingen voorzitter of bestuurder was. 

\begin{figure}
\centering
\includegraphics[width=\textwidth]{img/ch2/ch2-afb05}
\caption*{\footnotesize Hier opa met het muziekkorps `Voorwaarts' in 1935 in het Volkspark. Hij zit op de eerste rij, zesde van rechts, achter de trommelaars.}
\end{figure}

\begin{figure}
\centering
\includegraphics[width=\textwidth]{img/14OpaMeijns7}
\caption*{\footnotesize Opa Meijns staat uiterst rechts vooraan als voorzitter van de Geheelonthouderszangvereniging De Korenbloem. Ze staan op de Burcht voor Café Zaandam. Een mooie plek voor geheelonthouders.}
\end{figure}

\begin{figure}
\centering
\includegraphics[width=\textwidth]{img/ch2/ch2-afb04}
\caption*{\footnotesize Dat mijn opa muzikaal was blijkt ook weer uit een foto waarop hij (in het midden) samen met ome Cas (tweede rij van boven, derde van rechts) met een instrument afgebeeld staat. Aan de bebouwing te zien is het in de Dominee Baxstraat.}
\end{figure}

\begin{figure}
\includegraphics[width=\textwidth]{img/ch2/ch2-afb08}
\caption*{\footnotesize Revuegezelschap Voorwaarts}
\end{figure}

Op een foto uit 1931 van het revuegezelschap `Voorwaarts' staat hij rechts vooraan, weer als voorzitter, voor het station Zaandam. Ze waren op weg naar Amsterdam. Dat geeft toch weer een ander beeld van iemand. Volgens mijn schoonzuster Nel was mijn oudste broer Gerard zeer gesteld op opa en oma. Hij ging vaak, met z’n verloofde, bij hen langs. 

% \begin{figure}
% \includegraphics[\textwidth]{img/ch2/ch2-afb04}
% \caption*{\footnotesize Opa Meijns recht vooraan in het midden. Ome Cas staat tweede rij van boven, vierde van rechts. \textbf{controleren}}
% \end{figure}

Opa deed ook werk voor de geheelonthouders en vocht, in de krant, een debat uit met partijgenoten en tegenstanders die hem nogal radicaal vonden. In het partijblad 'Het Volk' eindigt hij het debat met zijn opponent als volgt: 'Bedenkt, Veltman: Elke dronkaard ziet alras. wat er komt van 't eerste glas. Met partijgroeten G. Meijns.'

Ook vond ik advertenties, zoals ééntje uit 1939 waarin een `net meisje' wordt gevraagd. Hij zette, mede namens zijn vrouw, elk jaar een nieuwjaarsgroet in de partijkrant met de beste wensen voor de partijgenoten, geestverwanten en clientèle en een jaar waarin VREDE en VRIJHEID zullen zegevieren. Dat was eind dertiger jaren.

% \begin{figure}
% \centering
% \includegraphics[width=\textwidth]{img/ch2/collectie}
% \caption*{\footnotesize Gedempte Gracht boven, stukje over het probleem alcohol. Daarnaast een nieuwjaarswens en een advertentie in het Zaans Volksdagblad van 30 december 1939.}
% \end{figure}


% \begin{figure}
% \centering
% \includegraphics[width=\textwidth]{img/drankweer}
% \caption*{\footnotesize Boven: Revuegezelschap Voorwaarts. Onder: Drankweer.}
% \end{figure}

% \begin{figure}
% \includegraphics[width=0.5\textwidth]{img/ch2/ch2-afb07}
% \caption*{\footnotesize Advertentie.}
% \end{figure}

% \begin{figure}
% \includegraphics[width=0.5\textwidth]{img/ch2/ch2-afb09}
% \caption*{\footnotesize Nieuwjaarswens in het Zaans Volksdagblad van 30 december 1939}
% \end{figure}

% \begin{figure}
% \includegraphics[width=0.5\textwidth]{img/ch2/ch2-afb10}
% \caption*{\footnotesize ?}
% \end{figure}



Bij de controverse tussen mijn vader en zijn vader zal de politiek mede een rol hebben gespeeld. De strijd tussen de sociaaldemocraten en communisten is al een hele oude. Aan het begin van de twintigste eeuw is die splitsing tot stand gekomen, vanuit het gezamenlijke begin in de Sociaal Democratische Bond, en heeft altijd voortgeduurd. Zoals het gebruikelijk is ter linkerzijde, heeft men elkaar te vuur en te zwaard bestreden. Dat zal in de verhouding tussen vader en zoon niet anders zijn geweest. 

Zoals altijd bij een ruzie tussen volwassenen, de kinderen zijn meestal de dupe. Het is niet anders. Maar ik ben toch blij dat ik dit allemaal heb gevonden.

% \begin{figure}
% \includegraphics[\textwidth]{img/ch2/ch2-afb11}
% \caption*{\footnotesize Gedempte Gracht}
% \end{figure} 


% \begin{figure}
% \includegraphics[\textwidth]{img/ch2/ch2-afb13}
% \caption*{\footnotesize }
% \end{figure}



% chapter opa_meijns (end)

\section*{Familie Kriek} % (fold)
\label{cha:familie_kriek}



\begin{figure}
\centering
\includegraphics[width=\textwidth]{img/ch3/familie}
\caption*{\footnotesize Lena staat links, achterste rij.}
\end{figure}

Met de familie van mijn moeder, de Krieken uit Krommenie, ging het veel beter. Daar werd nog wel eens op zondagmorgen een bezoekje aan gebracht.   

Met de fiets, ik achterop in het zitje, gingen we langs de Zaan naar het Vermaningspad, helemaal aan het eind tegenover de begraafplaats. Soms gingen we met de bus. Dan zag je de hele Zaanse industrie langskomen. We stapten uit bij het Badhuis aan de Padlaan. 

\begin{figure}
\centering
\includegraphics[width=\textwidth]{img/ch3/Vermaningspad1910}
\caption*{\footnotesize De familie woonde aan het Vermaningspad (hier in 1910). In het eerste huisje links, met het puntdak, woonde de familie Kriek. Rechts is het kerkhof. Later is het allemaal bestraat. Twaalf kinderen zijn er geboren en grootgebracht.}
\end{figure}

Meestal waren ook de andere kinderen, ooms en tantes met aanhang, op zo’n zondagmorgen aanwezig. En natuurlijk was ome Piet er, die nog thuis woonde. 

Opa en opoe Kriek hadden, net als ik nu, vaste plaatsen. Ze zaten allebei aan een kant van de tafel die voor het raam in de achterkamer stond en uitkeek op de achterplaats. 

Opa zat links voor het raam. Achter zich had hij een rekje met diverse pijpen en een ander rekje voor de krant. Opoe zat links naast de deur naar de keuken.

Op het erf stond naast de werkplaats van opa een houten WC die boven de sloot loosde. Je zat met je kont in een open gat en onder je was de sloot, simpel. Vooral bij zware bommen kon het voorkomen dat de plons zo hoog kwam dat ie je billen raakte. Met krantenpapier afvegen.

\begin{figure}
\centering
\includegraphics[width=\textwidth]{img/ch3/opoekriek}
\caption*{\footnotesize Opoe Kriek}
\end{figure}

\begin{figure}
\centering
\includegraphics[width=\textwidth]{img/ch3/opakriek2}
\caption*{\footnotesize Opa komt hier net aangereden.}
\end{figure}

Opa was wat je nu loodgieter zou noemen, maar toen was het nog een blikslager. Er stond volgens mij nog een handkar op het erf. Je kwam achter binnen via het keukentje waar een aantal oliestellen stonden, dan een opstapje naar de woonkamer. 

In de voorkamer---de pronkkamer---stond een mooi dressoir met daarop foto’s van de ooms als worstelaars en de prijzen die ze hadden gewonnen. De mooiste foto was natuurlijk die waar ze allemaal opstonden met hun prijzen. 

\begin{figure}
\centering
\includegraphics[width=0.7\textwidth]{img/ch3/Louiscorfransfred}
\caption*{\footnotesize Louis, Cor, Frans, en Fred.}
\end{figure}

Boven waren de slaapvertrekken. Het was eigenlijk één grote ruimte die in het midden afgescheiden was door een gordijn. 

Toen alle kinderen nog thuis woonden sliepen de meisjes aan de ene en de jongens aan de andere kant. Aan de voorkant van het huis zaten buiten spionnetjes gemonteerd, spiegeltjes waarmee je van binnenuit de hele straat kon afkijken of er iemand aankwam. 

\begin{figure}
\centering
\includegraphics[width=\textwidth]{img/ch3/indebakfiets}
\caption*{\footnotesize In de bakfiets met Annie, Lena, Rieka, Siema, Jo en opa Kriek}
\end{figure}

Toen opoe begon te dementeren zat ze vaak in een stoel voor het raam te kijken of er iemand kwam. Meestal zat ze nog te wachten op ome Jan, haar oudste, die thuis moest komen voor het eten. 

Ze zette dan ook nog wel eens eten op, maar er kwam nooit meer iemand eten. Eén keer had ze het zo hoog opgezet dat de oliestelletjes begonnen te loeven. Het hele keukentje zag zwart. We zijn toen nog wezen schoonmaken en witten. Mij herkende ze toen al niet meer. Ik kwam er te weinig.

\begin{figure}
\centering
\includegraphics[width=0.8\textwidth]{img/ch3/opbezoek}
\caption*{\footnotesize Op bezoek met de familie bij Opoe Kriek. Ze keek altijd een beetje stuurs.}
\end{figure}

Onderling kwamen de broers en zussen nog regelmatig bij elkaar. De band was bij de Krieken hechter dan bij de Meijnsen. Bij zo’n zondags ritje van Zaandam naar Krommenie, meestal op de fiets, ging je de Zaanstreek door via allerlei geuren. 

In de Koog had je Honig, in Wormerveer Crok en Laan, en in Krommenie had je de Lum, ofwel de linoleumfabrieken. Aan het Vermaningspad lag de blikfabriek. 

Als je blind zou zijn geweest zou je aan de geur weten waar je ongeveer was. 

\begin{figure}
\centering
\includegraphics[width=\textwidth]{img/ch3/vandebios}
\caption*{\footnotesize Opoe en Opa Kriek uit de bios weer naar huis. Ze hadden vaste plaatsen, zegt een bron.}
\end{figure}

\begin{figure}
\centering
\includegraphics[width=0.8\textwidth]{img/ch3/opoe82jr}
\caption*{\footnotesize Moeder en dochter op het achtererf. Zie maar hoe een klein hokje het huis was.}
\end{figure}

\begin{figure}
\centering
\includegraphics[width=\textwidth]{img/ch3/Zussen}
\caption*{\footnotesize De gezusters: Lena, Sima, Rieka, Annie en Jo (en Opoe)}
\end{figure}

\section*{Mijn ouders} % (fold)
\label{cha:ouders}

Mijn vader vertelde dat hij in Culemborg geboren was. Waarom in die uithoek is mij nooit duidelijk geworden. Mijn moeder is in Krommenie geboren. De eerste vrouw van mijn vader, en de moeder van Gerard, overleed al heel jong. Ik heb wat foto's bij elkaar gezocht uit hun jonge jaren, voor het beeld.

\begin{figure}
\centering
\includegraphics[width=\textwidth]{img/JongeGeert.bmp}
\caption*{\footnotesize }
\end{figure}

\begin{figure}
\centering
\includegraphics[width=\textwidth]{img/JongeLena.bmp}
\caption*{\footnotesize }
\end{figure}

\begin{figure}
\centering
\includegraphics[width=\textwidth]{img/29moe}
\caption*{\footnotesize }
\end{figure}

\begin{figure}
\centering
\includegraphics[width=\textwidth]{img/29coppi}
\caption*{\footnotesize }
\end{figure}

\begin{figure}
\centering
\includegraphics[width=\textwidth]{img/30Heemskerk}
\caption*{\footnotesize }
\end{figure}

\begin{figure}
\centering
\includegraphics[width=\textwidth]{img/30lenasport2}
\caption*{\footnotesize Boven: Strand in Heemskerk, de jonge Geert liggend vooraan. Onder: Lena tijdens sportactiviteit. }
\end{figure}

\begin{figure}
\centering
\includegraphics[width=\textwidth]{img/ch4/opdebsa}
% \caption*{\footnotesize }
\end{figure}

\begin{figure}
\centering
\includegraphics[width=\textwidth]{img/ch2/ch2-afb12}
\caption*{\footnotesize Bij de foto staat ‘huisje Bakkum’. Van links naar rechts, achterste rij: opa en oma Meijns, tante Lena (Duif), tante Nel, ome Jaap Duif, en een onbekende. Vooraan moe en pa Meijns, Gerard (of Harry), en ome Cas.}
\end{figure}

\begin{figure}
\centering
\includegraphics[width=\textwidth]{img/ch2/ch2-afb13}
\caption*{\footnotesize Hier staat m’n vader uiterst links, tegen het hek.}
\end{figure}


\section*{Rosmolenstraat 88} % (fold)
\label{cha:rosmolenstraat}

\begin{figure}
\centering
\includegraphics[width=\textwidth]{img/33Rosmolenstraat}
\caption*{\footnotesize De Rosmolenbuurt}
\end{figure}

De Rosmolenstraat 88 was een benedenhuis: voor- en achterkamer, twee slaapkamers, keuken, lange gang en een behoorlijke tuin. De bovenburen hadden geen tuin en ook geen recht op de onze. 

Toen de verhouding met de bovenburen slechter werd en er ruzie ontstond, eindigde dat toen ze uit frustratie de pispot in onze tuin leeggooiden. Toen ze verhuisden kwamen er vreselijk luidruchtige Amsterdammers wonen. Het was vooral erg als ze samen ruzie hadden. 

Toen mijn vader daar een keer iets van wilde zeggen was het raak. Pa sprak in de deuropening tegen de buurman die bovenaan zijn trap stond. Plotseling trok die een touw waarmee de fietsen tegen de trap vaststonden los, zodat die met geraas naar beneden kletterden. Mijn vader greep me nog snel en trok hun voordeur dicht die nu van binnen geblokkeerd werd door fietsen.

Het was een blok van vier woningen, twee onder en twee boven, dat ingeklemd lag tussen de Poortstraat en de Jan Bouwmeesterstraat. Onze achterkant lag tegen de tuinen van de Poortstraat aan. Ik herinner me meer van wat er achter het huis lag en de uitgang naar de Poortstraat. 

Niemand van de familie kwam via de voordeur thuis. Alles ging achterom, de steeg in langs de schutting van onze buren---van der Schaaf---door de poort, langs de heg van buus de Jong, en je was achter op het erf. 

\begin{figure}
\centering
\includegraphics[width=\textwidth]{img/34Rosmstraat2}
\caption*{\footnotesize Achter het huis en die kleine ben ik.}
\end{figure}

Achter hadden we een tuin, stond een schuur, duivenhok, later een kippenhok en veel later nòg een schuur voor de fietsen en brommers. 

Als Harry thuis kwam dan vlogen de duiven hem al tegemoet. Later zijn ze nog jammerlijk aan hun einde gekomen. Geslacht door de schillenboer. Hoe die slachting ooit besloten is blijft in de familiegeheimen besloten. Gerard, Harry en ik stonden achter het gordijn naar die slachting te kijken. Ik vol afschuw en fascinatie, iemand die zomaar de koppen van de duiven afsneed en ze in een zak gooide. Later werden ze gebraden op tafel opgevoerd maar niemand moest er natuurlijk iets van hebben. Alles zinloos.

\begin{figure}
\centering
\includegraphics[width=0.9\textheight]{img/ch5/RuudHarry2}
\caption*{\footnotesize Met Harry, het wagentje komt nog terug}
\end{figure}

Pa had het kippenhok gebouwd voor z’n baas, Kraaijer. Het had een Zaans motief. Moe zorgde voor de kippen, vooral als er één gepikt werd en ingesmeerd moest worden. We hebben die kippen in Krommenie gehaald. Op een groot omheind stuk grond liepen honderden kippen en wij mochten uitzoeken (Pa en ik). Ik liep in korte broek en die kippen maar in m’n kuiten pikken. 

\begin{figure}
\centering
\includegraphics[width=\textwidth]{img/ch5/kippenhok}
\caption*{\footnotesize Het kippenhok}
\end{figure}

\begin{figure}
\centering
\includegraphics[width=\textwidth]{img/ch5/zaanselftal-2}
\caption*{\footnotesize Het Zaanse elftal dat in 1946/47 tegen het Belgische Mechelen speelde. Voorste rij rechts, naast Dirk Kuiper zit Rudie Michel, waar ik naar vernoemd zou zijn. Hij was ZFC’r, maar nog niet toen ik geboren werd, dat geeft wel te denken.}
\end{figure}

\begin{figure}
\centering
\includegraphics[width=0.9\textwidth]{img/ch5/peterme}
\caption*{\footnotesize Zelfde karretje als dat met Harry, nu met Peter Baas. Volgens mij dezelfde manteltjes. Ik mocht zeker het karretje niet uit.}
\end{figure}

Naast ons woonde de familie van der Schaaf. Net als wij waren zij communistisch georiënteerd, maar fanatieker dan bij ons thuis. 

De zonen van van der Schaaf werden naar partij-iconen van de communistische wereld vernoemd. Paul naar Paul de Groot, de partijvoorzitter van de Communistische Partij Nederland (CPN). En Karl naar de Duitse communist Karl Liebknecht. 

Omdat mijn vader en van der Schaaf van mening verschilden over het communisme spraken ze niet meer met elkaar. Met de familie boven ons, kregen we later ook weer ruzie. Waarom is mij niet bekend, maar ik weet wel dat ze rotzooi in onze tuin gooiden. 

Boven van der Schaaf woonde de familie Schaap. Met zoon Jaap zat ik in de klas en z’n vader werkte als chauffeur bij Bruynzeel. 

Toen de man een keer even thuis was en z'n truc in de straat parkeerde zei Jaap dat we er wel even in mochten. Door wat voor oorzaak dan ook startte de auto en reed heel langzaam vooruit. Van schrik sprongen we er beiden uit, maar als held ben ik er weer ingeklommen en pakte het stuur. 

\begin{figure}
\centering
\includegraphics[width=\textwidth]{img/ch5/indetuin}
\caption*{\footnotesize }
\end{figure}

Gelukkig kwam buurman er al aanrennen en nam het over. Maar omdat ik er alleen in zat viel er wel een sterke verdenking op mij. 

Schuin tegenover ons woonde een familie met een thuiswonende zoon die blindegeleidehonden africhtte. We mochten wel eens mee naar het Oostzijderveld, dat aan het eind van de Poortstraat begon. Hij had altijd Duitse herdershonden. Z'n ouders waren al bejaard. Als het gordijn dichtging wist je dat moeder weer bediend werd door de pastoor. Meestal gingen de gordijnen na een paar dagen weer open. 

Tegenover ons op de eerste verdieping kwam familie de Licht te wonen, met Coby, de mooiste van de straat. Met haar heb ik veel gehoelahoept en doktertje gespeeld in een tent achter het huis. Tot we onraad hoorden en een hoofd de tent binnen keek. Zweten en een rooie kop van de spanning. 

\begin{figure}
\centering
\includegraphics[width=\textwidth]{img/ch5/Bakkum2}
\caption*{\footnotesize Kamp Bakkum}
\end{figure} 

Dat we communistisch waren betekende niet zoveel voor de buurt, behalve in verkiezingstijd. Dan hadden wij een bordje aan de deur hang met rood vlak met daarin een witte 6, van lijst 6, Paul de Groot. 


De Jan Bouwmeesterstraat, de eerstvolgende straat na de onze, was helemaal katholiek, dus die stemden allemaal Katholieke Volkspartij (KVP). 

Als zij bij ons de bordjes van de CPN eraf sloegen dan loerden wij op een kans om met lange stokken door de Jan Bouwmeesterstraat te rennen om die KVP-bordjes er af te slaan.

Maar normaal was er geen enkele animositeit. Onze vaders waren als oude vrienden en stonden gewoon met elkaar te kletsen. 

Er werd nog wel eens gekeken of je wel met iemand vriendje mocht zijn. Toen er een nieuwe jongen in de straat kwam wonen moest hij eerst thuis vragen of hij vriendjes met mij mocht zijn. Het mocht, omdat ik bij ondervraging kon melden dat mijn vader Gerard Meijns was. ``Oh, Geer Meijns'', zei de vader. 

\begin{figure}
\centering
\includegraphics[width=\textwidth]{img/ch5/vanharte}
\caption*{\footnotesize Iets te vieren, moe de bloemen en Pa de emmer }
\end{figure} 

Toen was het goed, ze kenden elkaar. Je weet blijkbaar maar nooit, het was tenslotte Koude Oorlog. In een buurt waar iedereen elkaar kent hoef je nooit voorgesteld of gekeurd te worden, dus dat fenomeen was mij onbekend.

Ik kan me nog herinneren dat ik een keer met Ellie Nijzing uit de Jan Bouwmeesterstraat naar de nachtmis ben geweest. Later werd Ellie de vrouw van de kunstenaar Theo Blankesteijn. Wie had ooit kunnen denken dat ik later Theo zou ontmoeten en dus ook weer met Ellie kennis zouden maken?

Maar goed, ik met de familie Nijzing mee naar de St. Bonefatiuskerk in de Oostzijde. Vooral een landurig vertoning. Na afloop gingen we eten bij de familie Nijzing thuis. 

Ik heb geen idee hoe men er thuis tegenover stond. Ik ben veel later nog eens met Co Smit naar een kerstmis in Wormerveer geweest. Ik zat met Co in een bandje. Die mis duurde naar mijn mening ook te lang, net als de Matthaeuspassie ook altijd veel te lang is (maar dat terzijde).

De heer en mevrouw Pel hadden in huis een winkeltje in tabakswaren. Op de toonbank hadden ze net zo'n vlammetje als bij m'n ome Cas in zijn winkel aan de Zuiddijk. Door de gang en in het zijkamertje dreven zij hun negotie. 

Zij waren ook het huis met een TV. Zo’n enorme kast, met een naar verhouding klein beeld. Niet iedereen had toen TV en je keek bij buren die er wel één hadden.

Woensdag- en zaterdagmiddag gingen we TV-kijken met een stel kinderen. Schoenen uit en met z'n allen voor de buis. Ik weet niet of ze er ook iets voor vroegen.

We keken tante Hannie (Hannie Lips) met het programma \emph{De Verrekijker}, waarin wetenswaardigheden van over de gehele wereld werden getoond. 

Een ander programma was \emph{Coco en de Vliegende Knorrepot}. Je kon de draadjes waaraan hij door de lucht vloog duidelijk zien. 

\begin{wrapfigure}{r}{0.5\textwidth}
\begin{center}
\includegraphics[width=0.40\textwidth]{img/Monus}
\end{center}
\caption*{\footnotesize }
\end{wrapfigure}

Echt voor kinderen dus. \emph{Morgen Gebeurt Het} was een toekomstvisie met ruimteschepen en dergelijke. Dat was toen heel spannend, maar als je nu het decor ziet is het niet voor te stellen dat dat eng was.

Er was blijkbaar veel vraag naar verhalen over de toekomst, want op de radio was het hoorspel \emph{Monus, de man van de maan} erg populair. Ik was daar een grote fan van en had ook een boek van de serie gekregen.

De radio was eigenlijk het enige vertier. De hoorspelen waren meestal in het weekend want door de week moest iedereen vroeg naar bed.

We hadden ook een beeldhouwer in de straat. Hij woonde iets verder dan wij, bij de een grote steeg middenin het blok die naar achter de huizen leidde.  

Hij houwde beelden voor kerken en dergelijke. 's Zomers als zijn raam openstond kon je hem aan het werk zien. Ik heb de man hem nooit anders dan in zijn werkruimte gezien, die van buitenaf zichtbaar was. Nooit buiten of zo.

De schillenboer kwam met paard en wagen langs, stopte in de Poortstraat en haalde dan achterom z’n waar op. (Hij slachtte soms ter plekke.) 

Kolen werden natuurlijk ook achterom gebracht. In het oude schuurtje was de bak voor de kolen. Met een kolenkit haalde je de kolen naar binnen. Later, toen we een oliekachel kregen, kwam de olieboer met z’n wagentje langs. Hij woonde op het St. Catharijnepad.

De kolen en de olie hebben beiden een aparte geur. Vlak bij school, in de Herderstraat of zo, was een kolenboer. Op het terrein waren verschillende vakken voor alle soorten kolen die hij verkocht. Ook achter het spoor bij het station waren enkele opslagplaatsen voor kolenhandelaren. De melkboer kwam natuurlijk aan de voordeur.

De Poortstraat was eigenlijk de opening naar de wereld. Door de steeg was je zo bij de Sparwinkel en bij mijn vrienden Jan Schoen, Leen Bakker en Peter Baas, die in het volgende plantsoen woonde. 

\begin{figure}
\centering
\includegraphics[width=\textwidth]{img/ch5/Spar-Poortstr}
\caption*{\footnotesize }
\end{figure}

En bij buurvrouw (buus) de Jong natuurlijk. Ik ben er onlangs nog eens langs gereden. Buus de jong woonde om het hoekje. Als ze de achterdeuren naar de tuin open had staan en ze hoorde me langskomen kon ze me vragen om een boodschap te doen bij de Spar, of vroeg ze of ik trek had in iets lekkers. Dat was meestal ontbijtkoek, dik besmeerd met echte boter. Ze had van dat lange grijze haar in vlechten en was altijd wel een beetje ziek.

Bij haar mocht ik graag vertoeven. Er hing een geur van oude gezelligheid, van dikke kleedjes, sigaren en poezen. Buus hield van poezen. Ze had stapels \emph{Katholieke Illustratie} of \emph{Het Leven}, met veel plaatjes over verre werelddelen, rampen, grootse gebeurtenissen enzovoorts. Je kon heerlijk op de grond liggen bladeren. Buurman de Jong zei nooit zoveel. 

\begin{figure}
\centering
\includegraphics[width=\textwidth]{img/ch5/poortstrsteeg}
\caption*{\footnotesize In de Poortstraat, links de ingang van de steeg en het huis van Buus de Jong.}
\end{figure}

We waren allemaal van Zaanlandia. Buurman stond in de kantine. Toen mijn ouders hun 25-jarig huwelijk vierden stond hij er ook. Buurvrouw heb ik er nooit gezien, ze was altijd wat zwak en ziekjes. 

Vlak bij school had je banketbakkerij de Zeeuw. Als je daar de deur opendeed kwamen de heerlijke geuren van gebak en fijne allerhande je tegemoet.

Mijn broer Gerard was al jong de deur uit, naar Den Haag, waar hij bij de PTT werkte en een opleiding kreeg. Hij was dus al uit huis voordat ik me van hem bewust werd. Hij rookte Dr. Duskind, reed op een Vespa (scooter) en had een verloofde, Nel. Ze huurden een kamer in Amsterdam-West.

\begin{figure}
\centering
\includegraphics[width=0.9\textwidth]{img/ch5/gn0102}
\caption*{\footnotesize Gerard en Nel op hun etage in Amsterdam}
\end{figure}

Onlangs kreeg ik wat fotootjes uit die tijd. Ik ben nog eens met m’n vader naar hun etage in Amsterdam gefietst. 

Hun huwelijksfeest werd gehouden in het Pontrestaurant in Amsterdam Noord, waar ik mee mocht helpen met de bediening. Als een echte ober liep ik met het dienblad rond. Als het restaurant niet gesloopt was, zouden ze het er nog over hebben, zo’n succes was dat.

\begin{figure}
\centering
\includegraphics[width=0.8\textwidth]{img/ch5/trouwengn}
\caption*{\footnotesize Hier gaan we op weg bij Nel haar huis in Ransdorp. Kijk m’n pinkelhoutje? En natuurlijk korte broek.}
\end{figure}

Als kind vond ik het erg spannend om met een spiegel door het huis lopen. De spiegel hield ik voor m’n borst en probeerde dan alleen daarin kijkend door het huis te lopen. Bij de overgang van de kamer naar de gang stapte je dan in een diep gat, omdat het plafond in de gang dieper was, of leek. 

Het spannendst was om zo eerst een tijdje rondlopend aan het kijken in de spiegel te wennen en vanuit de keuken naar buiten te stappen, want dan viel je de lucht in. Werkelijk alsof je geen grond meer onder je voeten had. Bij de achterdeur naar de tuin was een afstapje. Als je dan met de spiegel liep had je het gevoel steeds verder de hemel in te stappen, heel eng. 

\begin{figure}
\centering
\includegraphics[width=\textwidth]{img/ch5/gevonden_0009}
\caption*{\footnotesize Het afstapje met m’n vader, moeder van Peter en mijn moeder.}
\end{figure}

Als m’n moeder de was deed stond er in het begin een grote wastobbe op het vuur. In de zomer stond ze buiten met het wasbord de kleding te kuisen. Later kwam er een wasmachine, of alleen een apparaat dat in de tobbe rond draaide. 

We waren ook lid van speeltuinvereniging Het Oosten in de Oostzijde, dat ook een gebouwtje voor films had. Zaterdag- of woensdagmiddag was het duwen en trekken om als eerste binnen te komen, want er waren altijd meer gegadigden dan stoelen. Bij de ingang stond een man van de speeltuin om orde te houden. 

Het was een grote man met een klompvoet en een schoen die twee keer zo groot was als z’n andere. De films waren meestal de Dikke en de Dunne (Laurel en Hardy), soms The Three Stooges, Comedy Capers of The Keystone Cops. 

\begin{figure}
\centering
\includegraphics[width=0.8\textwidth]{img/ch5/ComedyCapers}
\caption*{\footnotesize The Keystone Cops}
\end{figure}

In het filmzaaltje was de chaos bijna net zo groot als op het witte doek; veel geschreeuw. In de winter hadden we een keer al heel lang voor de deur staan wachten toen er iemand op m’n voet ging staan. De hele voorstelling lang had ik een vreselijke pijn, want mijn voet werd niet meer warm. Na de voorstelling hard naar huis rennen.

Eén keer heb ik in het Apollo Theater aan de Dam een voorstelling gezien van Sjors en Sjimmie in Afrika, waar ze het aan de stok kregen met wilden en krokodillen. Na de voorstelling bleek het buiten al donker te zijn geworden. Ik wist niet hoe snel ik thuis moest komen, want je weet maar nooit met krokodillen. Thuis direct naar m’n kamer en met een sprong op m’n bed, voor de veiligheid. 

\begin{figure}
\centering
\includegraphics[width=\textwidth]{img/ch5/apollo}
\caption*{\footnotesize Het Apollo theater}
\end{figure}

\section*{Kleuterschool} % (fold)
\label{cha:kleuterschool}

Gelukkig was de kleuterschool dichtbij. Vanuit huis liep ik er in twee minuten naartoe.

\begin{figure}
\centering
\includegraphics[width=0.8\textwidth]{img/ch6/Frobelschool}
\caption*{\footnotesize De man met de kar is Noë van de Poort-kruidenier, later De Spar.}
\end{figure}

\begin{figure}
\centering
\includegraphics[width=\textwidth]{img/ch6/klschool53}
\caption*{\footnotesize Op deze foto van de kleuterklas staat m’n juf, Winsemius, waarmee ik als ik groot zou zijn zou gaan trouwen. Jan Schoen zit uiterst links, op het verhoginkje. Peter Baas zit op de tweede rij uiterst rechts vooraan. Die met het witte truitje is Lyda Hooyschuur. Rechts daarvan zit ik, met de zwarte galgjes.}
\end{figure}

\begin{figure}
\centering
\includegraphics[width=\textwidth]{img/ch6/kleuterschool2}
\caption*{\footnotesize We zullen ongetwijfeld veel knip- en plakwerk hebben verricht. We kijken hier allemaal verrast om. `Wat, een foto?'}
\end{figure}

Een blijvende herinnering is wel dat ik met m’n autoped m’n grote liefde naar huis bracht. Ze was de dochter van de dominee die in de Oostzijde woonde, op de hoek van de Schoolmeesterstraat van de ijsboer \emph{De Danser} en de snoepwinkel waar we `bakkesvols' haalden. 

De poort naar de school was bevestigd aan twee stenen pilaren waar je op kon zitten. Tijdens één van die zittingen heb ik Jan Schoen een gat in z’n hoofd geslagen en hem daarna zorgzaam naar huis gebracht. Waarom is me nooit duidelijk geworden, want het was mijn beste vriend. 

\section*{Stegen en straten} % (fold)
\label{cha:stegen_straten}

In onze buurten waren de stegen erg belangrijk. Onze eigen buurtstegen waren een bron van veiligheid, omdat je precies wist waar ze naartoe leidden en wat je kon verwachten. Wat verder uit de buurt kon je weleens voor verrassingen komen te staan als je zomaar een steeg instoof. 

Er zijn stegen met onuitwisbare herinneringen, zoals toen ik de melkboer hielp en met flessen onder m’n arm via de Poortstraat de steeg van de Rosmolenstraat inliep en viel. Handen vol glas, waar nog de tastbare bewijzen van zijn te leveren. Of die keer, in dezelfde steeg, toen iemand me tekkelde en ik wel zere knieeën had, maar pas thuis merkte dat m’n broek aan m’n knie zat vastgeplakt van het geronnen bloed. Dat deed bloed toen nog.

En onze eigen steeg, die er recht tegenover lag. Daar zat ik tijdens de dagen voorafgaand aan luilak uitdagend te doen op een autoband die wij al hadden voor het feestvuur, tot een ploeg vanuit de Jan Bouwmeesterstraat plotseling kwam aanstuiven en de band onder m’n kont vandaan trok. Ik kon nog net m’n hand door het gat slaan en keihard om hulp roepen. Bij de kruising van de Poortstraat en het Albert Hahnplantsoen waren er genoeg vrienden te hulp geschoten om het in een vechtpartij te laten ontstaan die door ons werd gewonnen. 

Toen ik onder de kluwen omhoog kwam stond er een ‘vijand’ te lachen ik heb hem toen een kaakslag gegeven als een volleerde boxer.

Kort daarna liep ik op de Heijermansstraat tussen de Poortstraat en Jan Bouwmeesterstraat toen ik plotseling tussen een stuk of tien Jan Bouwmeester-straters stond en met m’n rug tegen de buitenmuur van het laatste huis werd geduwd. Er was geen ontkomen aan en ik kreeg flinke klappen. 

\begin{figure}
\centering
\includegraphics[width=\textwidth]{img/ch7/Bouwmeesterstraat}
\caption*{\footnotesize Heijermansstraat met rechts de de muur waartegen ik de klappen ontving.}
\end{figure}

Nog een incident vondt plaats toen een delegatie uit de Jan Bouwmeesterstraat--- naar later bleek versterkt door Rooms-Katholieke-geloofsgenoten uit Brabant---ons te lijf wilden gaan met stokken met voorin een grote spijker. Bij het uitdagen waren wij in de Poorstraat toch weer in de meerderheid en drongen wij de Roomsen een steeg in. Daar, ik moet het bekennen, maakte ik een van de stokken afhandig en duwde die een Katholieke Brabantse buik in. De jongen is nog naar het ziekenhuis vervoerd en er werd niet naar de dader gevraagd. Onze monden voor eeuwig gesloten. Zo hoort dat onder makkers.

Verderop hadden we ook wel vriendjes zoals in de Koning Williamstraat, maar dat was toch al verder dan onze directe buurt. Als je wat groter bent wordt je actieradius ook wat groter. Als je klein bent is een tocht naar een heel andere straat al een hele onderneming.

Wim Al was een vriendje. Hij woonde in de Leo XIII straat. Zijn vader had een transportbedrijf. Hij kwam veel in het Veem en vond daar een torenvalkje. Hij bracht het mee naar huis en Wim mocht het opvoeden. Eten geven bedoel ik. 

We gaven het beestje wormen die we eerst door de midden knipten, anders zou het valkje stikken. Dat ging heel goed en op een dag vloog hij weg.

Toen de wereld in de ban was van de eerste ruimtereizen besloten wij ook een poging te wagen. We hadden een kleine raket gebouwd, van hout met vleugeltjes, stopten er spiritis in en lucifers, staken het aan, \emph{woesj!} een kleine vuurbal en hij viel om. Einde ruimtevaartclub.

Toen Olga en ik later met Chris en Kelly in Drenthe vakantie vierden kwamen we langs een uitspanning met een kaboutergrot. Altijd leuk voor de kinderen. Je moest afdalen in de grond en daar was een pad uitgegraven dat je langs allerlei kabouters voerde die aan het werk waren of lagen te slapen. 

Boven gekomen gaan we nog wat drinken, dan hoor ik opeens: ``Hé Ruud Mainz''. Was het Wim Al die de eigenaar van deze uitspanning was. Hij sprak nog steeds Zaans.

\section*{Wat er door de straat kwam} % (fold)
\label{cha:straat}

De straat had nog het karakter van een openbare marktplaats. Van alles probeerde de waren huis aan huis te slijten: de melkboer, groentenboer, schillenboer, visboer, ijsboer, voddenman, muzikanten, ambulante handelaren \emph{und so weiter}. Die muziekanten vond ik wel vreemd. Werd er aangebeld, naar de deur lopen, openen, begint een man opeens viool te spelen en wij daar maar staan te kijken en luisteren.

Onze melkboer was Langelaar, die in onze straat z’n winkel had. Met z’n karretje verkocht hij losse melk die zo in een pannetje  geschonken werd. 

Ik mocht hem helpen en na z’n wijk te hebben gedaan moesten alle melkbussen en dergelijke schoongemaakt worden. 

\begin{figure}
\centering
\includegraphics[width=\textwidth]{img/64Rosmolenstraat-Langelaar-1966}
\caption*{\footnotesize De buurtwinkel van Langelaar}
\end{figure}

Dat gebeurde in z’n garage met heel veel water. 

Met dat helpen heb ik me nog eens aardig bezeerd. Ik liep met de flessen van de kar achterom een steeg in en kwam terug met lege flessen. Achter een heg zat iemand met een stok die tussen m’n benen werd gestoken. Ik viel voorover, strekte m'n handen om de val op te vangen vol in de gebroken flessen. Allemaal glas in m'n handen en een kapotte knie.

Van een groentenboer kan ik me niet zoveel herinneren. Van de aanverwante schillenboer meer, omdat die met een paard en wagen langs kwam. 

\begin{figure}
\centering
\includegraphics[width=\textwidth]{img/63schillenboer}
\caption*{\footnotesize Hier zit ik op het paard van de schillenboer.}
\end{figure}

Ik sta nog op een foto met dat paard, genomen vanuit de Poortstraat. De visboer kwam, voorafgegaan door z’n vrouw Marie op haar fiets, met de bakfiets door de straat. Zoals veel ambulante handel had Marie een kreet waardoor je van ver al wist dat de visboer eraan kwam. Ze had nog een ouderwets soort kleding aan met veel schorten over elkaar.

De voddenman, een klein gedrongen mannetje, kwam ook op de bakfiets en een eigen kreet de straat in. Je verstond er niets van, maar je wist `dat is de voddenman’. Hij had een pakhuis in de Zilverpadsteeg waar we als kinderen ouwe kranten konden aanleveren. Het was net een hol waar je binnenkwam. Het hing en stond vol met spullen. 

Er waren ook handelaren die de ene keer met dekens en dan weer met een kar vol met fruit langs kwamen. Ik heb ook nog meegemaakt dat een man met garen en band langskwam; een echte marskramer. En natuurlijk de ijskarren die je al van ver hoorde aankomen. Als je al in bed lag maar hopen dat er toch nog een ijsje zou komen.

De ijsboeren waren een slag apart. Je had `de danser’ uit de Schoolmeesterstraat en die van ijssalon \emph{Tempo} in de Damstraat. 

De eerste werd `de danser' genoemd omdat hij door het gewicht van z’n ijskar, als die nog vol was, omhoog geduwd dreigde te worden en z’n kracht dus op het neerdrukken moest richten. Hij kon als het ware zwevend lopen. 

\begin{wrapfigure}{r}{0.5\textwidth}
\begin{center}
\includegraphics[width=\textwidth]{img/66tempo}
\caption*{\footnotesize IJszaak Tempo}
\end{wrapfigure}

Tempo was een echte Italiaan met dito ijs. Zijn zaak in de Damstraat was vooral bekend door als startpunt te dienen voor het geflaneer van de opgeschoten jeugd.  Die gingen vanuit daar richting de Westzijde, ongeveer tot aan het Typhoongebouw en weer terug. Dat werd ook wel ‘billenavond’ genoemd. 

\section*{Bakkum} % (fold)
\label{cha:bakkum}

\begin{figure}
\centering
\includegraphics[width=\textwidth]{img/winkelplein22.jpg}
\caption*{\footnotesize Bakkum-de winkeltjes}
\end{figure}

Op vakantie gingen we naar Bakkum, het tentenkamp nabij Castricum. Vanaf m’n vroege jeugd kwamen we daar ’s zomers twee weken. Ik sta op fotoos als klein ventje samen met de familie van tante Siema, een zus van m’n moeder. Het begon allemaal als de kist met spullen voor de vakantie werd ingepakt. Deze werd dan met een bodedienst naar Bakkum verstuurd en door ons uit een loods weer opgehaald en naar het huisje gebracht. Het was een gehuurd huisje. 

Als kind hoef je niets te doen. Je zit achterop de fiets in een zitje en dan opeens ben je er. Dan moet er water gehaald worden voor in de tank. Dat deed je bij een gezamenlijk waterplaats waar de mensen zich ’s morgens wasten, schoren. 

\begin{figure}
\centering
\includegraphics[width=\textwidth]{img/Bakkumkranen}
\caption*{\footnotesize De kranen, Harry staat achter mij}
\end{figure} 

Het was een grote gemeenschap van mensen die elkaar soms al jaren weer op Bakkum tegen kwamen.  Voor kinderen was het ideaal en dus ook voor ouders. Je had al snel vriendjes. Op de Veldweg naar het strand, aan het eind van het kamp was een groot terrein waar gesport werd. Vliegeren was ook populair. En tussen de bomen bij de huisjes werden vele lijntjes gespannen voor het badminton.

Niet ver van de ingang waren de winkeltjes. Toen ook nog houten kraampjes net als de huisjes. Visboer, melkboer, poelier, ijsboer. In een wat groter gebouw kon je je butogasfles omwisselen voor een volle en petroleum halen voor de lamp. Bij de poelier vroegen we wel eens om kippenpootjes. Dat waren de afgehakte pootjes en als je het goeie spiertje te pakken had kon je de tenen laten bewegen en meisjes laten schrikken.

Je kon je eigen krant laten opsturen en die moest je dan ’s middags bij het postkantoor ophalen. Het postkantoor was in een oude bunker. Gelukkig kampeerden er veel CPN’rs zodat je niet de enige was die De Waarheid kwam afhalen. Achter het postkantoor was een amfitheater ‘De Pan’ waar soms muziek of theater werd gegeven. En je had De Apenlaan met bomen waarin je kon klimmen, het vijvertje wat verderop waar grote karpers in zwommen die de hele dag brood kregen van bewoners.
En natuurlijk was er de zee. Via de Veldweg liep je door de duinen naar het strand. Stoffig en hier en daar kon je onderweg bramen plukken. Bij het laatste duin was er dan eindelijk de zee. Plekje zoeken en daar bleef je dan een middag en daarna weer op huis aan. Teruglopen was minder dan heenlopen, de vermoeidheid sloeg toe. Moeders moest dan aan het eten maken en wij hingen rond tot het klaar was. 

’s Avonds een rondje lopen, voetballen of op ‘jacht’; Fazantenjacht. Met m’n vader en nog wat jongeren uit de buurt de duinen en het struikgewas in. De fazanten vlogen pas op als je heel dichtbij was en je schrok je lam en dat was de sensatie van ‘de jacht’.
Bakkum was echt fantastisch en ik heb daar uitstekende herinneringen aan. Met ons gezin zijn we er ook naartoe getrokken voor vakantie.

\section*{Luilak} % (fold)
\label{cha:luilak}

Luilak was sowieso een fantastische tijd waar veel over te vertellen valt. Als je klein bent loop je met je vader, broers en zijn vrienden, buren enzovoorts mee. Op elk kruispunt was wel een fik. Als je groter wordt ga je het zelf doen met leeftijdgenoten. 

\begin{figure}
\includegraphics[width=\textwidth]{img/ch9/luilak1954}
\caption*{\footnotesize Luilak, 1954}
\end{figure}

Meestal was de gezelligheid vooral de tijd die eraan vooraf ging: het verzamelen. 

Bij elke deelnemer aan een fik werd van alles opgeslagen, tot de dag daar was en het op de plek van ontsteken werd bijeengebracht. Spannend was natuurlijk wie de grootste fik had. Als het te groot was kon je last krijgen met de politie.

De grootste fik in mijn jonge jeugdjaren was het vuur op het grasveld voor de Noorderkerk, dat zo groot was dat het het grote, ronde, gebrandschilderde raam dreigde te smelten. Onze dienders hadden het maar druk met al het kattekwaad. 

Ik ben één keer opgepakt toen ik een motor in de brand had gestoken. Lang verhaal, maar er zat benzine in en dat wilden we aftappen. Af en toe staken we hem een beetje in brand en bliezen het weer uit tot het niet meer uitging door alle gemorste benzine. Toen het duidelijk was dat het niet meer te blussen was vloog iedereen weg. 

Ik kwam thuis en vertelde m’n ouders van de fik. M’n vader zei: ``Ga maar terug en kijk hoe het ermee staat, dat kun je niet zo laten gaan,'' of iets in die geest. Ik weer terug en iedereen was weer terug. Toch proberen hem uit te maken en plots stond er politie. 

Na veel vragen ben ik naar voren gestapt en heb me aangegeven als de aanstichter. De motor werd in een zandwagen geladen en ik met de chauffeur (geen politie) naar het politiebureau. 

\begin{figure}
\centering
\includegraphics[width=\textwidth]{img/ch9/politburo}
\caption*{\footnotesize Het politiebureau in de Vinkenstraat (nu weg).}
\end{figure}

Onderweg naar het bureau zei de chauffeur me niets te bekennen en vol te houden dat het een ongeluk was. Op het politiebureau zat ik in het wachtlokaal en hoorde alle oproepen om bijstand als er een geasfalteerde weg begon te smelten. Met mij deden ze niets. Lieten me alleen zitten. Uiteindelijk werd ik door m’n vader pas om acht uur opgehaald en de volgende dag als een verzetsheld door de vriendjes begroet. 

De eigenaar van de motor is nog langs geweest om via de verzekering iets te regelen. Hij noemde zijn naam, `Kattestaart', wat thuis tot veel hilariteit leidde.

Met luilak kon je ’s morgens al heel vroeg bij de bakker warme witte bollen kopen. Ik dacht altijd dat hij dan speciaal voor luilak zo vroeg open ging, maar hij was natuurlijk allang aan het bakken en deed alleen z’n winkel vroeger open. 

\section*{Vreemde zaken} % (fold)
\label{cha:vreemde_zaken}

Er was een oude man in de Tolstraat die zich vanwege een handicap in een (nu) antieke rolstoel moest voortbewegen. Twee stangen dreven het apparaat vooruit. Je moest ze met de hand op en neer duwen. De bochten waren het moeilijkst, leek me, als je vaart moest houden en tegelijkertijd een bocht moestnemen. Ik heb er altijd met fascinatie naar gekeken. 

\begin{figure}
\centering
\includegraphics[width=\textwidth]{img/ch10/rolstoel}
% \caption*{\footnotesize }
\end{figure}

Dat kijken vond mijn moeder wel eens hinderlijk. Elke keer als we op straat iemand met een handicap tegen kwamen bleef ik daar gefascineerd naar kijken. Soms deed ik het ook na, waarop mijn moeder dan me beschaamd toeriep om met die onzin te stoppen. Ik kon naar elke handicap blijven kijken.

In de Rosmolenstraat woonde een mevrouw die een krop had. Die persoon had dan een soort ballon in de nek hangen. 

Tegenwoordig zie je dat niet meer. Je ziet sowieso geen handicaps meer die vroeger normaal waren om tegen te komen. Er zijn veel meer mogelijkheden om foutjes van de natuur vroeg op te sporen en er iets aan te doen.

Mensen die slepen met een klompvoet. Eén voet normaal en de andere voorzien van een enorme schoen. Die mensen liepen altijd ongelijk. Ik heb zo’n leraar gehad.

In de Kopermolenstraat woonde een meisje, Beppie. Ze was klein, kleine handjes en voetjes en had een waterhoofd. Ik ken haar alleen maar als zittend in de deuropening met een schare kinderen om haar heen. Ze was niet mentaal gehandicapt, want ze kon goed met de kinderen om haar heen overweg. Maar dat hoofd, ik kon m’n ogen er niet vanaf houden.

\section*{Sport} % (fold)
\label{cha:sport}

Laatst dacht ik nog over de mooiste doelpunten die ik heb meegemaakt, naar aanleiding van een tv-uitzending. De mooiste waren de doelpunten bij ZFC thuis.Ook als je als kind speelde en niet keek en er steeg plotseling zo’n storm van geluid op dan kwam er zo’n heerlijk vreugdevol gevoel over ons allen. \emph{We} maakten een doelpunt en \emph{we} waren blij. In het seizoen 1958 werd ZFC kampioen van de Tweede divisie.

Mijn vader was een ZFC’er en ging de thuiswedstrijden altijd kijken. Hij had een vaste plaats op de staantribune achter het doel aan de kant van de Westzanerdijk. 

De kinderen speelden rond het terrein. We probeerden dan gratis op de tribune te komen. Iemand gooide dan het kaartje langs de kant naar beneden en zo kregen we dan soms toegang. 

\begin{figure}
\centering
\includegraphics[width=\textwidth]{img/74zfc}
\caption*{\footnotesize Het ZFC-veld, we kijken naar de staantribune aan de Westzanerdijk.}
\end{figure}

Onder de staantribune vonden we nog wel eens iets wat de mensen daarboven hadden laten vallen. Mijn vader kon ontzettend hard schreeuwen. Je kon hem overal bovenuit horen. Als de familie (pa, moe en ik) het terrein op kwam werd er op de staantribune al plaatstgemaakt door de vaste groep. Moe ging alleen mee als het lekker weer was. 

Als het afgelopen was moest je eerst je fiets weer terugvinden. Die stonden in lange rekken op de Westzanerdijk. Daarna ging die grote slang van fietsen weer terug naar huis, veroorzaakte opstoppingen bij de smalle Wilhelminabrug. Naarmate de tocht naar huis vorderde, werd de groep steeds kleiner. 

\begin{figure}
\centering
\includegraphics[width=\textwidth]{img/75zfc2}
\caption*{\footnotesize De fietsenstalling op de Westzanerdijk}
\end{figure}

Van m’n sportcarrière bij Zaanlandia weet ik nog enkele flarden. Bijvoorbeeld de wedstrijdjes bij de welpen, waar elk team de naam van een beroemde voetbalclub had (wij waren Barcelona) en we op een half veldje speelden. Dat was nog op het oude Zaanlandia terrein, met sloten er omheen, tegenover de Veeringstraat. Later kwam daar een verpleeghuis, en nu staat daar grotendeels het ziekenhuis.

Bij de aspiranten, ik zat in de F-jes, weet ik nog van een wedstrijd tegen ZFC op het B-veld aan de Westzanerdijk. Ik was altijd rechtsback, maar toen had ik een \emph{rush} naar voren. Omdat het veld zo groot was, had ik toen ik eindelijk in de buurt van het doel van de tegenstander kwam geen kracht meer om de bal er fatsoenlijk in te schieten. En daarna helemaal terug naar je eigen plek, weer dat hele veld over. 

Soms moest je ook heel ver uit spelen. Een keer, bovendien de laatste keer, speelden we bij een club aan het eind van Krommenie, aan de Provinciale weg. Op de fiets gingen we er heen, voetballen, en weer op de fiets terug. 

Toen ben ik maar op judo gegaan. Ik speelde bij Judoclub Zaandam, onder leiding van Siem Boering. Een strenge trainer, maar wel een goeie. Als hij zag dat je je voor een wedstrijdje met hem terug trok, dan haalde hij je eruit. Nooit terug trekken, vond hij. En terecht. We trainden in de Wilhelminaschool, aan de Gerhardstraat. Op die plaats staan nu woningen.

\begin{figure}
\centering
\includegraphics[width=\textwidth]{img/76schooljudo}
\caption*{\footnotesize De Lindeboomschool-Wilhelminaschool }
\end{figure}

Eénmaal heb ik in een wedstrijdteam in Amsterdam gejudood. In die partij kwam het tot een eindgevecht op de vloer, waarbij ik m’n tegenstander in de derde houtgreep kreeg. Altijd lastig. Dan duren die zestig seconden heel lang om hem erin te houden. Je ziet in een ooghoek je teamleden springen en schreeuwen en die man onder je licht te kronkelen om eruit te komen. Maar ik heb hem wel eronder gehouden. 

Zwemmen heb ik geleerd in de vierde klas. Mijn moeder stond dan bij school te wachten en uit school gingen we rechtstreeks naar het Sportfondsenbad in de Mauvestraat. 

Dat was ook het bad waar we schoolzwemmen kregen. Het was een mooi bad, maar de kleedhokken beneden waren nat en koud. Ik vond het smerig daar, maar dat terzijde. 

\begin{figure}
\centering
\includegraphics[width=\textwidth]{img/77sportfondsenbad}
% \caption*{\footnotesize }
\end{figure}

De eerste lessen krijg je met een haak om je buik. De zwemleraar---de vader van een klasgenote--hielp e door de eerste fase heen. Bij het alleen zwemmen kreeg ik krampen. Volgens de badmeester aanstellerij, maar toen dat maar aan bleef houden toch maar eens door de huisarts laten onderzoeken. Navelbreuk. Dus lag ik voor een operatie in het ziekenhuis en heb daarna nooit meer last gehad.

Ik lag een week, of misschien veertien dagen, in het ziekenhuis. Daar lag ik met nog een jongen op een apart kamertje, terzijde van de mannenafdeling. Toen ik weer mocht lopen ging ik de zaal op en kreeg nogal wat snoep toegestopt van de mannen daar. Goeie tijd gehad.

Toen ik thuis kwam kreeg ik, voor alle ontberingen, een voetbal. Geen echte van leer, maar iets wat er op leek. We zaten met z’n allen in de achterdeur naar de tuin met de voeten op het trapje. De bal werd uitgepakt, er werd wat meegespeeld en hij sprong af op een spijker en dat was het kado---lek. 

Nog een pechgeval. Op een verjaardag kreeg ik een keer van Jim, de Ierse echtgenoot van Elly Berends, een zeilboot om in elkaar te zetten. Omdat er op verjaardagen meestal niet veel te doen is, besloten Gerard en anderen de boot alvast in elkaar te zetten. Trots werd mijn kado als ‘klaar’ gepresenteerd. Daar had ik in ieder geval niet veel meer aan.

Als je wat ouder en wat wilder wordt gaat het wel eens mis. Zo ben ik eens tijdens een spelletje tikkertje van de hoge duikplank afgedonderd. Niet van de plank in het water. Nee, ik miste een tree bij het beklimmen van de trap, viel terug en klapte met m’n hoofd op het beton. 

Ik ben bewusteloos geweest en toen ik bijkwam was het een oorverdovend lawaai van schreeuwende kinderen. Naar het ziekenhuis in de Frans Halsstraat gebracht en daar behandeld.

Mijn begeleiders waren alweer vertrokken, zodat ik toch nog met een hersenschudding op m’n fiets weer naar de Rosmolenstraat terug moest rijden. Daar moest ik prompt overgeven. 

Toen ik een andere keer van het zwemmen naar huis reed kwam ter hoogte van de Jedelooschool en de toenmalige groentenboer m’n badtas tussen m’n voorwiel en dook ik over m’n stuur met m’n kin op de straatstenen. En zeer dat m’n kin deed. Pas toen de groenteman naar buiten kwam en me overeind wilde halen merkten we pas dat m’n kin open lag. Weer naar het ziekenhuis en krammetjes erin. En weer alleen op de fiets naar huis.

\begin{figure}
\centering
\includegraphics[width=\textwidth]{img/80ZiekenhuisFrHallstr}
\caption*{\footnotesize Het oude ziekenhuis}
\end{figure}

Bij het ziekenhuis stond de kraam van Cor Knikker. Cor had een lamme rechterhand. Als mijn moeder zag dat ik weer een lichaamsdeel met een elastiekje afknelde hield ze me voor dat Cor zo ook aan z’n verlamde hand was gekomen.

\begin{figure}
\centering
\includegraphics[width=\textwidth]{img/79corknikker}
\caption*{\footnotesize Cor met z'n lamme rechterhand}
\end{figure}

We hielden eens een wielerwedstrijd rond de Rosmolenstraat en de Koning Williamstraat. In een bocht vlak voor de kleuterschool, op de hoek, bij het huis van Lyda Hooyschuur, gleed ik door zand onderuit en belandde in het hek met prikkeldraad. Toen ik m’n arm eruit haalde kon ik zo naar binnen kijken. Weer naar ‘t ziekenhuis. Nog steeds is deze plek te bewonderen op m’n bovenarm. 

Dat wielrennen heeft wel een tijdje mijn belangstelling gehad. Dat kwam omdat mijn vader bij de Zaanlandse wielerclub DTS ooit een beker had gewonnen die thuis in de kast stond. Hij nam me mee naar het Heerenhuis waar toen de thuisbasis van DTS was. We mochten in het hok achter het pand waar de renners zich klaar maakten voor de wedstrijd. Dat heeft een speciale geur vanwege de olie die ze op hun geschoren benen doen. Daarna volgden we de wedstrijd op de brommer en zo kon je mooi het afzien van de verschillende renners zien. Er werden rondjes door de Beemster gereden. Pa nam me ook eens mee naar het Olympisch stadion in Amsterdam waar allerlei baanwedstrijden waren en als toetje werden de renners, die dat jaar de Tour de France hadden gereden, gehuldigd. Spectaculair vond ik de wedstrijden achter de motoren. De renners rijden dan achter speciale motoren en halen hierdoor hoge snelheden. Toen hij eens met een baanstuur voor mijn fiets thuiskwam was het einde natuurlijk zoek. Overal waar ik heen reed deed ik dat met de grootste spoed. Alles moest snel. We deden ook aan tijdrijden. Van ons huis in de Rosmolenstraat via de Heijermansstraat naar de Paaskerk en weer terug en mn vader nam de tijd op. Weer sneller dan de vorige keer, natuurlijk.

Bij de Paaskerk was Karel Appel net aan het schilderen en zag ik Freek de Jonge nog lopen. Hij zou er later uitvoerig over schrijven, maar mijn recordpoging heeft hij over het hoofd gezien. 

\begin{figure}
\centering
\includegraphics[width=\textwidth]{img/81paaskerkveembrand}
\caption*{\footnotesize Boven: Karel Appel in de Paaskerk. Onder: de veembrand, gefotografeerd vanuit de Hoveniersstraat.}
\end{figure}

% \begin{figure}
% \centering
% \includegraphics[width=\textwidth]{img/81paaskerk}
% \caption*{\footnotesize Karel Appel in de paaskerk}
% \end{figure}

Zoals Freek in zijn boek ook mijn aanwezigheid bij de brand in het Veem aan de Oostzijde onvermeld liet. De grootste brand die ik ooit gezien heb. We kwamen allemaal te laat op school. 

% \begin{figure}
% \centering
% \includegraphics[width=\textwidth]{img/81veembrand}
% \caption*{\footnotesize De foto is vanuit de Hoveniersstraat genomen.}
% \end{figure}

We woonden vlakbij het Veem en in het Konijnenpad en de Hoveniersstraat had je een goed uitzicht op de brand. De Zaan stond door de cacaoboter in brand. Harry, die bij Sabel werkte, had een goed overzicht en zag de panden instorten en de Zaan in brand staan.

Wij moesten weer naar school. Ik sprak laatst een oud brandweerman en die vertelde nog met liefde over de mooiste brand die hij ooit had meegemaakt. 

\section*{IJspret} % (fold)
\label{cha:ijspret}

Ik heb als kind altijd veel geschaatst. Vooral op de Gouw, die toen de voor geoefende rijders een uitvalsbasis was naar verre oorden, zoals Marken.

\begin{figure}
\centering
\includegraphics[width=\textwidth]{img/84bruggouw}
\caption*{\footnotesize Brug over de Gouw.}
\end{figure}

Toen in 1960 Tuindorp/Oostzaan overstroomde kon je er op de schaats gaan kijken. Omdat Harry en z’n vrienden ook op de Gouw schaatsten had ik het idee dat ik wel met hen op hun tochten meekon. Dat was een misrekening, ze waren tien jaar ouder. Bij de spoorbrug wilden ze me niet meer mee hebben en moest ik terug. Die brug over de Gouw was echt een verzamelplek. De allerjongsten leerden er schaatsen, anderen oefenden zich in het ijshockey, er stond een koek en zopie tentje.

Vanuit de Poortstraat en Jan Bouwmeesterstraat liep je naar de brug over de Gouw. Aan de overkant waren volkstuinen waar boeren met hun vlet langs kwamen om vee naar weilanden te brengen, enzovoorts. 

Op een keer, tijdens het schaatsen, vonden we onder het ijs een boekje (\emph{Nachten van Parijs} van Jan Brusse) waarop we nog net een foto van een half naakte danseres konden zien. Daar wordt je jongenshart warm van. Wat een mazzel. We zijn later terug gegaan, hebben gehakt en het er nat maar heel uitgekregen. Later zag ik het boekje bij Gerard en Nel in de kast staan. 

Op dezelfde plek is er nog wel een brug, maar de omstandigheden zijn gewijzigd. De brug lag op de plek achter de heuvel in het In ’t Veldpark. Bij dezelfde Gouw heb ik één keer gevist, maar het idee om zo’n levend beest aan de haak te krijgen vervulde me met afschuw en ik heb de hengel te water gegooid en ben naar huis gegaan. Dat wachten duurde me ook veel te lang, dus dat was dat. 

Later kreeg ik noren, met schoen. Dat was een raar gezicht, omdat ik in de zesde klas nog altijd met korte broek liep en ook zo schaatste. Deze schaatsen hebben me uiteindelijk de das om gedaan. 

M’n schaatsen werden te klein en ik stapte steeds met dooie voeten van het ijs. Toen was het over en heb ik nooit meer geschaatst. 

\begin{figure}
\centering
\includegraphics[width=\textwidth]{img/ch13/metpa2}
\caption*{\footnotesize Hier met Pa in het park met op de achtergrond het bejaardenhuis Oostererf, gesloopt. Nu staat het ziekenhuis er.}
\end{figure}

\section*{Jongensdorp} % (fold)
\label{cha:jongensdrop}

In de vroege jaren vijftig werd in de zomervakantie door de Zaanse Gemeenschap een zogenaamd `jongensdorp' georganiseerd. Dat gebeurde op een braak liggend stuk land achter het Noordzeekanaal. Het was dus voor jongens, die er een eigen dorp in elkaar timmerden. Ik mocht er ook een keer heen. 

Je moest eigen gereedschap meenemen. Omdat m’n vader timmerman was kreeg ik een hamer en een zaag mee. Die hamer heb ik nog. Er staat op getekend dat die van mij was, want in zo’n dorp kan makkelijk iets wegraken. 

\begin{figure}
\centering
\includegraphics[width=0.8\textwidth]{img/87-88jongensdorp}
\caption*{\footnotesize Jongensdorp, 1953}
\end{figure}

We bleven daar slapen. Hoelang weet ik niet meer. Er werd van alles georganiseerd, zogenaamd door de jongens zelf. Op de foto is zo’n aktiviteit te zien in het jaar 1953. Het kan zijn dat ik er toen bij was, want ik was toen al zeven jaar.

\section*{Scheepswerf} % (fold)
\label{cha:scheepswerf}

Mijn vader werkte bij scheepswerf Kraaijer, aan het eind van de Zuiddijk. Later werd dat scheepswerf De Beer. Bij Kraaijer had hij als scheepstimmerman gewerkt; later kon hij als magazijnchef aan de slag. Ik ging vaak mee en heb daar heel wat schepen te water zien gaan. Ik leerde ook veel mensen kennen die daar werkten. Er was een oud-Indiëganger die me Maleis probeerde te leren, en andere mannen die me meenamen onder de schepen. Ik zag hoe bij een tewaterlating de laatste blokken werden weggeslagen en hoe het schip de Zaan ingleed. 


\begin{figure}
\centering
\includegraphics[width=\textwidth]{img/90deBeer}
% \caption*{\footnotesizeMaryNubel}
\end{figure}

Soms werd je ook wel voor de gek gehouden. Dan werd je weggestuurd om een paar luchthaken te halen, of een doosje vuursteenvonkies. Toen ik dat door had wist ik wel beter. Op een keer vroegen ze me of ik een paar nieuwe bokkepoten wilde halen, ``voor het teren'' zeiden ze nog. Ik liet me niet gek maken. Hoe zeer ze ook vertelden dat dit geen grap was, ik geloofde ze niet. Later bleken die echt te bestaan. Ik ben ook nog eens tussen dekschuiten in het water terecht gekomen; ik stapte mis. Gelukkig kon ik goed zwemmen en zag waar ik naartoe moest zwemmen om er tussen vandaan te komen. 

Die oud-Indischman vertelde ook nog hoe je een mes kon maken van oude zaagbladen. Hij zat bij de ijzerzaag. Die bladen waren vier of vijf centimer breed. Toen ik voldoende geslepen had en er een handvat omheen had gemaakt deed ik er wat olie op om het goed te laten glimmen. Een grote doek gepakt om het af te vegen en ik sneeds dwars door die doek in mijn hand. Het was echt een vlijmscherp mes geworden. 

Dat litteken heb ik nog. Het mes niet meer, het werd in beslag genomen.

\begin{wrapfigure}{0.5\textwidth}
\begin{centering}
\includegraphics[width=0.48\textwidth]{img/ch15/schuttevaer12}
\caption*{\footnotesize Mary Nübel}
\end{wrapfigure}

Een heel groot schip was de Mary Nubel. Later werd gezegd dat het te groot was voor een werf als de Beer. Dat het meer kostte dan het had opgebracht en de ondergang van het bedrijf dichterbij had gebracht. 

De mooiste tewaterlating was de zijwaartse. Als het schip de helling afglijdt en het water raakt is het net alsof het om zal slaan, maar het richt zich toch weer op en drijft. De toeschouwers die op de kant van de haven stonden te kijken moesten dan uit de benen maken vanwege de enorme vloedgolf die op hen afkwam.

\section*{Muziek} % (fold)
\label{cha:muziek}

Toen we nog geen eigen radio hadden ontvingen wij de wereld via de radiodistributie. Het had een eenvoudige knop met vijf of zes kanalen en een luidspreker. 

Wat ik me daarvan vooral herinner is dat ik tijdens een ziekteperiode, liggend op de divan, luisterde naar het ochtendgebed van de KRO. Na een tijdje kende ik dat wel uit m’n hoofd, ``Moeder van God, bidt voor ons zondaars, nu in het uur van onze dood amen etc.'' Als ik dat trots liet horen waren mijn communistische ouders verontrust. 

We hadden bij de distributie een extra luidspreker in de keuken geïnstalleerd, zodat we ook daar konden genieten van Ray Conniff, Mitch Miller. Als je de draadjes omdraaide fungeerde die luidspreker als een microfoon, dan kon je horen wat er in de keuken gebeurde. Bij Jozef Schmidt kwamen de oorlog en de Joden weer om de hoek kijken. Bij Leni \& Ludwig en andere Duitse schlagers werd bedenkelijk gekeken, maar dat verwaterde naderhand. 

Ik stond eens op de Hempont met m’n vader toen er een Duitse auto de pont opreed. ``Zo, ze zijn alweer terug'', zei mijn vader. Communisten waren altijd bang voor een Duits `revanchisme’.

Toen we eindelijk een eigen radio kregen werd er ook een platenwisselaar aangeschaft. 

\begin{figure}
\centering
\includegraphics[width=\textwidth]{img/93jeugdfotoos_0010A}
\caption*{\footnotesize Radio met daaronder de platenwisselaar}
\end{figure}

Met een klap kwam zo’n 78-toerenplaat naar beneden, gleed even over de onderliggende plaat en werd dan meegenomen in de snelheid van de platen eronder. 

We kochten de platen bij Koopman op de Gedempte Gracht, zoals die van zangkoor de Maastrichter Staar, het Glenn Miller orkest en `Marina' van Rocco Granata. Later Papa Bue’s Viking Jazz Band en zo’n zangduo met een Nederlandse baron. Én een plaat van Johnny Jordaan, ‘Kerstfeest in de Jordaan’. Als die opstond hield mijn vader het niet droog, vanwege het trieste relaas van armoe. 

De aanschaf van een plaat was altijd een hele gebeurtenis. Er waren er ook zoveel bij Koopman. Gelukkig mocht je ze luisteren in boxen. Je deed je keus, kreeg een luisterbox toegewezen en de man achter de toonbank zette dan de plaat op. En maar genieten. 

Daarna besloot je of je de plaat kocht. Bij het vallen was zo’n plaat direct kapot. Met de komst van vinyl kwamen Perry Como, Louis Prima. Hoewel mijn Engels niet best was zong ik toch alles mee. Favoriet was natuurlijk Louis Prima met z’n swingende `Buona Sera', waarmee ik regelmatig voor de familie een playbackvoorstelling gaf.

% \begin{figure}
% \includegraphics[\textwidth]{img/...}
% \caption{Koopman op de Gracht}
% \end{figure}

Het eerste rockplaatje dat ik zelf kocht was een EP'tje met onder andere `To know know know you' en een nummer van Bobby Darin. Een verzamel-EP dus. Het was ongeveer op de plek waar later de V\&D was, bij een marktstal. In die bakken zoeken, en er is zoveel onbekend. Dat is heel heftig, omdat je eigenlijk alles wel wilt hebben, maar er maar één kunt kopen. Welke, welke, welke zal ik nemen? Dan is een verzamel-EP'tje geen slechte keus.

Later heb ik in diezelfde winkel nog eens een single gekocht met op de ene kant Little Richard en de andere Paul Anka. Tja, hoe het met Elvis was (of ik dat als eerste had of pas later)weet ik niet meer, want ik vond Elvis pas goed met `One Night'. `k Ben wel naar z'n eerste films geweest. Toen we in Duitsland op een Amerikaanse---nee, een Canadese---basis waren uitgenodigd, draaiden ze daar nog steeds Elvis-films van `Rock-A-Hula-Baby' enzovoorts.

M’n kennismaking met klassieke muziek begon in feite bij de familie Baas. Allemaal waren ze muzikaal. Ik kreeg gitaarles en soms werd er wel iets gedraaid of gespeeld wat voor mij op klassiek leek. Verder kregen we later op het Uitgebreid Lager Onderwijs (ULO) af en toe wat klassiek ingegoten door meneer Bakker. 

Gehandicapt door een klompvoet trachtte meneer Bakker ons enigszins in de richting van begrip voor de klassieken te brengen. Vooral Mozart, zijn favourite componist.

M’n eerste LP was een uitvoering van \emph{Wassermusik} van Händel door het Concertgebouworkest, onder leiding van Eduard van Beinum. Toen ik vele jaren later een uitvoering van Trevor Pinnock hoorde geloofde ik niet dat het dezelfde muziek was. Gelukkig heb ik de uitvoering van Van Beinum kunnen terugvinden, zodat ik weer van m’n eigen watermuziek kon genieten. 

% \begin{figure}
% \centering
% \includegraphics[width=0.8\textwidth]{img/96watermusic}
% % \caption*{\footnotesize Mary Nubel}
% \end{figure}

Soms duurt het even voor je weer klaar bent voor een nieuwe sprong in de muziek. Onder een documentaire van Hans Keller hoorde ik pianomuziek. Geen componist in de aftiteling, dus belde ik naar de VPRO. Het bleek Satie te zijn. Toen nog voor mij onbekend. Op naar \emph{de} platenwinkel van Zaandam van Hans Volmer op de Gedempte Gracht. ``Satie'', zei Hans, ``een ogenblikje,'' en kwam terug met een lijst waaruit ik kon kiezen. 

Hans was een muziekliefhebber die anderen probeerde te interesseren voor muziek. Zo bracht hij me ook in aanraking met Mahler. Dit gebeurde weer op dezelfde manier. In de film van Luchino Visconti, \emph{De dood in Venetië}, uit 1971 zit het prachtige adagio uit de vijfde. Ik weer naar Volmer. ``Een ogenblikje'' en ik kocht toen direkt maar alle symfoniën in een cassette.

En hoe groot kan toeval zijn dat later Peter Baas in die winkel kwam te werken? Zo was dat kringetje ook weer rond. Hans heeft zn winkel lang geleden opgedoekt en is bladmuziek gaan verkopen.

\begin{figure}
\centering
\includegraphics[width=\textwidth]{img/97volmer}
\caption*{\footnotesize Muziekhandel Volmer}
\end{figure}


\section*{Jaren '50} % (fold)
\label{cha:jaren50}

Ik hoorde een uitzending met Gerrit Komrij en zijn donker getinte, maar toch heldere, analyse van het bestaan. Hij had het over hoe hij gevormd is door de jaren `40-`45 zonder er bij te zijn geweest. ``Ik wist precies wie de goeie en de slechte waren, kende de ontberingen, de heldenmoed van de onzen en de slechtheid van de anderen." Even later zie ik op de TV de oude heer Drees spreken over Buchenwald en zie na-oorlogse beelden. Bij het zien van Drees kreeg ik zelfs een brok in m’n keel, vanwege mijn eigen verleden dat al zo lang en breed is. Komrij memoreerde het beeld van iemand die vlak na de oorlog geboren werd en in een omgeving opgroeide waarin iedereen nog bezig was om die oorlog te verwerken. 

Zelfs ik heb nog beelden, waar of niet waar, van mannen bij elkaar, met hoeden, in de slaapkamer. Voor communisten was het wel een heel speciale oorlog geweest, waarin de Russen een belangrijke rol hadden gespeeld en een wereldmacht waren geworden. Maar wij zaten hier in het Westen en nu was er een kans. Dat soort gedoe. Maar dat is vijftig jaar geleden. Vóór mij, vijftig jaar geleden, was Drees net zo oud als ik en keek hij vooruit. Een plotselinge schok.

Je zo verloren voelen in het immense bestaan hoeft niet eenzaam te zijn. Als je ziet en herkent hoe alles is gegaan en gelopen, geeft dat een gevoel van tevredenheid met de grote voortgang van de natuur. De onontkoombaarheid van het persoonlijk einde, dat geenszins bepalend is voor de voortgang van het geheel.

Soms is het onmogelijk om aan een gemeenschappelijk tijdsbeeld te ontsnappen. De jaren vijftig bijvoorbeeld. Hoe is die tijd niet neergezet in liedjes, sketches, films en \emph{whatever}? Toch heeft ieder van ons die erbij was die tijd zelf meegemaakt. In hoeverre klopt ons eigen, werkelijke beeld wel of niet met dat wat we in ons gemeenschappelijk hoofd hebben? Hetzelfde gaat op voor de jaren zestig. Het zijn bijna mythische periodes waarvan ik soms beelden zie die ik helemaal niet herken. ``Dat heb ik helemaal niet meegemaakt, daar weet ik niets van'', denk ik dan. Of ``was dat er ook?''

De jaren vijftig begonnen toen ik 3 jaar was; in 1950 werd ik vier. Toen was ik nog thuis, nog geen kleuterschool. Ik kreeg de bof, een ziekte waar je dood aan kunt gaan. Daar weet ik niets meer van dan flarden, beelden die wel of niet waar zijn. Dan de kleuterschool. In de klas zitten, dat is alles wat we deden. En buiten in de zandbak spelen. Maar dat deden we daar ook al na school. Er waren later, toen ik eens langs fietste, twee zandbakken, met grote betonnen randen. Maar in \emph{onze tijd} zat er nog een dak boven. Een grote open schuur met een dak erop en erin allemaal zand. Zo was het in \emph{onze tijd}!

En dan Grote School. Ook vooral veel in de klas zitten. De hoogtepunten die ik me herinner zijn stomme fouten, vergissingen, straf en één keer een aardige opmerking voor een tekening van een arend. Juf de Boer herinner ik me alleen maar als iemand waar ik van droomde. Een soort Florence Nightingale in oorlogstijd. Laatst nog een foto van haar gevonden in de krant. Ze leek nog sprekend op het beeld dat ik van haar had. En dan nog meer meesters en juffen en helemaal geen fluit onthouden. Ik kon wel lezen, schrijven, rekenen en figuurzagen. 'k Wist waar Boedapest lag en hoe een rechthoek er uitzag. Maar voor de rest niks onthouden.

Wat ik wel heb onthouden is de sfeer van het anders-zijn als kind van communisten. Er was een tijd van saamhorigheid in die Grote Oorlog en de communisten hadden daar een grote rol in gespeeld. Maar dat was toen. En we lazen elke dag \emph{De Waarheid}, die later een leugen bleek te zijn. Toen geloofden we erin: in de grote vooruitgang van de Russen; de grote oogsten van de lachende Georgiërs op grote tractoren; de vijfjarenplannen die elke keer een grote stap voorwaarts waren. We moesten opletten voor die Duitsers die nu in de NATO kwamen. De hoop kwam uit het oosten en het gevaar van de Duitsers. Je moest op je hoede zijn.

We lazen de gelukkige berichten in het blad van de Vrienden van de USSR en we hoopten op een mooie toekomst. Toen er een auto moest komen vanwege de hartklachten van m’n vader werd dat een Skoda uit het Oostblok. Want die waren veel beter en sterker, nog van echt staal. 

Een vergelijking met streng gelovigen doet zich al snel voor. Het geloof in het absolute ‘gelijk’ komt sterk overeen. Bij de communisten was het de partijleider, de \emph{Pravda}, het Centraal Comité; voor gereformeerden is dat het Oude Testament, Paulus. 

\begin{figure}
\centering
\includegraphics[width=\textwidth]{img/ch17/waarheid}
% \caption*{\footnotesize Muziekhandel Volmer}
\end{figure}

De leugens die je voor zoete koek aanneemt, het niet meer na kunnen denken, de angst voor het ongelijk dat de rede en redelijkheid blokkeert. 

Alles slik je maar, het gekuip, het beschimpen, de achterklap, het sectarisme, de schisma's. Altijd op je hoede zijn, \emph{der Feind hört mit}. Geen prettige tijd. 

Er er hingen ook donkere wolken, die voor mij als kind onduidelijk bleven. De Duitse herbewapening, het revanchisme, de NATO met (alweer) Duitse generaals. De 4 mei herdenkingen werden, volgens mijn vader, overschaduwd door de aanwezigheid van foute mensen. De offers van de communisten verdwenen naar de achtergrond. In Amerika werd een echtpaar ter dood gebracht dat zogezegd had gespioneerd voor de USSR; ook daar een communistenjacht. Gelukkig waren de bevriende partijen in Italië en Frankrijk groot en sterk en werden onze mannen in Moskou feestelijk onthaald. En ’s avonds naar Radio Moskou luisteren. 

Je merkte wel dat het niet makkelijk te ontvangen was, want de Amerikanen stoorden de zender. Maar als je het groene oog van de radio goed in de gaten hield kreeg je hem wel te pakken. 

Volgens mijn vader was het een Wormerveerder die daar vanuit Moskou ons toesprak en de goede berichten steeds maar weer de ether instuurde. Maar met de inval van het Sovjetblok in Hongarije trokken toch wel wat donkere wolken zich samen. 

De familie Meijns kwam zelfs vertellen dat ze nu alle banden met ons verbraken. \emph{De Waarheid} werd aangevallen en Felix Meritis belegerd. We hielden wel vol, onder andere door het Waarheidfestival, maar plots was alles niet zo zeker meer.

\begin{figure}
\centering
\includegraphics[width=0.8\textwidth]{img/ch17/Felix_Meritis1}
\caption*{\footnotesize Op de foto de aanval op Felix Merites in ’51. De redactie van de Waarheid zat aan de Keizersgracht in Felix Meritus.}
\end{figure}

Over de oorlog van `40-`45 valt niet veel te vertellen, behalve wat ik van m’n broer Harry heb gehoord. Onder het huis in de Rosmolenstraat lagen Duitse uniformen verstopt, die later door het verzet werden opgehaald en teruggebracht. Ik vroeg Harry daarnaar omdat ik van de vriendschap van m’n vader met Arend en Jan Kat wist, die allebei een rol in het verzet hebben gespeeld. 

Soms, bij alarm, vertrokken de mannen uit de buurt om met roeibootjes het Oostzijderveld in te trekken. Mijn vader dook onder in het dierenasiel waar een buurtgenoot de sleutel van had. Anderen gingen richting Oostzaan tot het gevaar weer geweken was. Maar meer dan dat kon hij me niet vertellen. Later hoorde ik nog dat pa samen met Grootes in de Binnenlandse Strijdkrachten (BS) had gezeten.

\section*{Lagere school} % (fold)
\label{cha:lagere_school}

Mijn eerste juf op de Leeghwaterschool was juf de Boer. Ik hield haar voor de eerste drie jaar. Haar vader was de schilder Jan de Boer die in de Oostzijde woonde. 

We begonnen in het lokaal links naast de ingang. In de vierde klas heb ik meester Winter gehad, wiens zoon ik later nog bij Pieter Schoen zou ontmoeten. Meester Winter was volgens mijn vader vroeger schilder geweest en had het tot onderwijzer gebracht. In boosheid trok hij je aan je oor uit de bank, ook als je met je benen nog vast zat in het bankje. 

Later kreeg ik meester Koelemaij die veel van wandelen hield. Op hete zomerse dagen stond hij alleen toe dat je je mond twee keer spoelde en één slokje doorslikte. 

De laatste klas deed ik bij meester Lefferman, die tevens hoofd der school was. Van de zenuwen of stress strafte hij nog wel eens de verkeerde. Ook mij. Hij schopte me een keer letterlijk de klas uit. Hij was helemaal over de rooie omdat ik bleef ontkennen ergens schuld aan te hebben. Ik vertrok naar huis. 

\begin{figure}
\centering
\includegraphics[width=\textwidth]{img/ch18/lwschool}
\caption*{\footnotesize 1954. Schoolfoto. Jan Schoen met het bord, rechts daarnaast ik en Peter Baas. Rechtsachter Juf de Boer en links meester Winter. Die grote in het midden is Jaap Schaap.}
\end{figure}

Toevallig was m’n vader thuis en die stelde voor om met de brommer naar een bloemencorso in Limmen te gaan. En zo reden we kort na het incident vrolijk toeterend langs de school richting Limmen. 


\section*{ULO} % (fold)
\label{cha:ulo}

Mijn verblijf in de ULO-school heeft drie jaar geduurd. Over de eerste klas deed ik twee jaar en ook de tweede dreigde ik te doubleren. 

Ik was niet geschikt om de hele dag binnen te zitten. In de tweede klas overleed m’n vader en door deze combinatie besloot ik de school te verlaten. Ik had het idee opgevat dat ik nu geld moest gaan verdienen. Nogal romantisch en een plotseling excuus om van school af te komen.  

De school was een ellende. Ik had er geen zin in, behalve natuurlijk in de vriendschap met Mieke en andere meisjes. Ook heb ik op de ULO een aantal vriendschappen opgedaan die later weer van pas kwamen. Henk van der Wissel, Henny Wijngaarden, Ed Eijchenberger. Maar het leren was niets. 

\begin{figure}
\centering
\includegraphics[width=\textwidth]{img/ch19/ULOSchool}
\caption*{\footnotesize Ik zit tweede rij, midden. Rechts voor mij Henk vd Wissel, rechts achter mij Siebe (blonde kuif) en achter hem Mieke. }
\end{figure}

In de eerste klas kwam ik het eerste kwartaal zelfs nog in korte broek naar school, omdat zo’n lange me veel te warm was. Ik schaatste tenslotte ook nog in een korte. Een keer reed ik in volle vaart, in plaats van rechtsaf naar de fietskelder, rechtdoor met m’n fiets het overblijflokaal in. Toen iedereen moest lachen, en het onder andere over m’n korte broek ging, ben ik maar overstag gegaan. Eén moment is me altijd bijgebleven. 

De eerste schooldag na het overlijden van m’n vader was niet makkelijk. Niemand zei iets en iedereen keek, behalve Henk, die zich uit de groep losmaakte en me plechtig een hand gaf en me condoleerde. 

Henk zie ik overigens nog steeds en hij schijnt nu een relatie te hebben met het meisje rechtsonder met blond haar, Joekie (Jolanda). Het meisje links naast de leraar is Mieke waar ik mee liep en fietste, die nu als vrijwilliger in het Mennisterf werkt. Zo zie je maar.

\section*{Vader} % (fold)
\label{cha:vader}

Ik merk steeds meer hoe belangrijk m'n vader voor me was en hoe belangrijk het ontbreken van zijn aanwezigheid is geweest. Althans dat denk ik. Een psychologe zei eens: ``Moet je nagaan hoe belangrijk de aanwezigheid van je moeder voor je is geweest, positief of negatief''. Er zijn van die filmpjes waarin een vader met z'n zoon praat, naar kleinkinderen kijkt, enzovoort.

Zoals ik kan genieten van Chrissy en Kelly, in wat ze zijn en worden, zo kan ik me voorstellen hoe hij van hen zou hebben kunnen genieten. Het duurde wel even voor ik Harry, Janny en Nel meer heb durven vragen over m'n vader. Dat was een hele stap, maar ook heel nodig omdat ik hem sinds zijn dood diep had weg gestopt. 

\section*{Brommers en auto's} % (fold)
\label{cha:brommers_autos}

Uiteraard had iedereen een fiets, maar met het stijgen der welvaart konden sommigen al beschikken over een brommer. 

\begin{figure}
\centering
\includegraphics[width=\textwidth]{img/ch20/Typhoon2-55A}
\caption*{\footnotesize Pa in de Poortstraat, nog met tuintjes, met de eerste Typhoon.}
\end{figure}

De eerste die mijn vader kocht was een Typhoon. Deze werd gekocht in een zaak aan de Zuiddijk, waar later de dansschool van Diederich Hoorn gevestigd was. 

De eerste Typhoon voldeed zo goed dat er jaren later een opvolger van werd gekocht, waarna de eerste overging naar Harry. Op de terugweg vanuit Bakkum en het strand reden mijn vader en ik voorop met de nieuwste Typhoon, daarachter Harry en Jannie met de oudere versie. 

Bij Wormerveer sloeg het onheil toe. Opeens lagen we languit op het rijwielpad. M'n vader onder het bloed, Harry en Jannie in de berm. Bleek dat de voorvork van de brommer van het resterende deel was afgebroken. M'n vader's oor lag er half af. Hoe het verder is gegaan weet ik niet meer. Ik kan me niet herinneren in een ziekenhuis te zijn geweest. Er is nog wel getracht een schadevergoeding te krijgen van de handelaar danwel de fabriek, maar door juridische trucjes en eindeloos touwtrekken is dat niets geworden. 

Toen m'n vader getroffen werd door hartproblemen waren de fiets en brommer taboe en kwam er een auto. De eerste was een prachtig Fiatje, tweezitter met achterklep, waar ik net aan in kon zitten. 

\begin{figure}
\centering
\includegraphics[width=\textwidth]{img/111Fiat-Topolino.jpg}
\caption*{\footnotesize De Fiat Topolino, wat is 'ie mooi}
\end{figure}

Daarna kwam de Skoda 1200. Er zijn vervolgens alleen maar Skoda's gekomen, wegens trouw aan de kameraden in Joegoslavië. 

Ik kon al vroeg autorijden, met tien jaar al. Dat kwam zo. Een vriend van m'n vader, Jan Kat, had bij de Tuin der Nederlanden aan de Oostzijde een autospuitbedrijf. Dat was particulier terrein. Daar mocht ik wel eens een auto wegzetten en later ook wel in onze eigen auto rijden. 

Nog weer later reed ik zo achteruit de loods in als het moest. Ik kan me mijn vreugde herinneren elke keer als ik zo'n ding naar binnen reed en m'n vader met ome Jan zag lachen dat ik zo makkelijk deed.

\section*{Hof van Holland 40} % (fold)
\label{cha:hofvanholland}

Omdat de bovenburen een voortdurende bron van ergernis bleven en m’n vader ernstig ziek werd---hartklachten, kanker---besloten we te verhuizen naar Hof van Holland. Dit was deel van een net gebouwde wijk, Hofwijk, gelegen achter het Smaal en de Oostzijde. 

Op de foto is goed te zien hoe nieuw alles nog was. Een nieuw huis met drie slaapkamers, douche en een zolder (een kleine zolder). 

\begin{figure}
\centering
\includegraphics[width=0.8\textwidth]{img/ch21/skoda}
\caption*{\footnotesize Pa, Jannie en moe bij de Skoda 1200.}
\end{figure}

Een douche hadden we nooit gehad, dus dat was wel een hele vooruitgang. Jannie en Harry woonden in, omdat ze door de woningnood geen huis konden krijgen. Ze hadden één kamer tot hun beschikking, m’n ouders één en ik ook één, dus dat was goed verdeeld vond ik. 

Het huis waarop we hier kijken werd door de familie Valstar bewoond. Ze hadden een zoon en een dochter. De laatste was van mijn leeftijd. In die tijd ging ik naar de ULO en raakte ik bekend met de muziek en de bandjes. Door die bandjes kwam ik ook in kontakt met Jan Abbing en diens vader en moeder. 

Na het overlijden van m’n vader ben ik min of meer opgevangen door de familie Abbing. 

Met Jan als vriendje kwam ik daar al eerder over de vloer, maar na deze gebeurtenis hebben ze me liefdevol een warme plek geboden.

\begin{figure}
\centering
\includegraphics[width=0.8\textwidth]{img/ch21/luxvak}
\caption*{\footnotesize Foto’s vakantie Luxemburg met de familie Abbing}
\end{figure}

Vader en moeder Abbing hebben altijd een bijzondere plaats in m’n herinneringen ingenomen. Al moet ik bekennen dat ik vooral vader Abbing nooit meer heb opgezocht of zo iets, behalve dan bij z’n begrafenis. 

Moeder Abbing zat in de mode. Ze maakte en verstelde jurken, japonnen en dergelijke. Vader zat in de verzekeringen, en was vroeger bakker geweest. Hij hield ontzettend veel van jazz, vooral Benny Goodman en van de conferences van Toon Hermans. Dan lag hij slap van het lachen.

Jan had in z’n slaapkamer een bibliotheek, vertelde hij. En hij had een collectie 78-toeren platen die we massaal het raam hebben uitgezeild, waardoor ze op de net nieuwe Prins Bernhard rotonde uit elkaar spatten. ’t Was niet veel volgens mij, maar we deden wat samen. 

Hij had een vriendinnetje, Beppie. Die had ook weer een vriendin, dus dat kwam goed uit. Hadden we allebei wat. 

Ik heb geen idee meer hoe ze heette, maar meestal op een zondagmiddag zagen wij vieren elkaar in de buurt van speeltuinvereniging St. Theresia, want daar woonden de meisjes en dan zoenen enzo.

Jan’s ouders hebben me binnen gehaald. Als een stelletje aten we om de week bij hen en dan bij mij thuis. Ze hebben me ook meegenomen op vakantie naar Luxemburg. Daar hebben we, zoals de foto’s bewijzen, gekampeerd. 

\begin{figure}
\centering
\includegraphics[width=\textwidth]{img/ch21/gitarist}
\caption*{\footnotesize Mijn slaapkamer met teksten op de muur geschreven.}
\end{figure}

Zij waren ook de organisatoren van een bezoek aan de Stadsschouwburg waar Cliff Richard optrad met The Shadows. Een fameuze inspiratiebron voor ons eigen bandje natuurlijk. 

En ze namen me mee naar enkele toneelvoorstellingen in Amsterdam met Ank van der Moer in een stuk van Tennessee Williams. \emph{De Getatoueerde Roos}. Hele, hele lieve mensen die ik erg dankbaar ben voor hun liefde in beroerde tijden. 

De gebrekkige gezondheid van m’n vader domineerde het gezin minder dan in de Rosmolenstraat. Het leek zich allemaal wel te stabiliseren als hij zich maar niet teveel inspande. Wel nog weer die kanker. We trokken daarvoor naar dokter Cornelis Moerman in Schiedam, die bekend werd door zijn genezingen en een therapie op basis van duivenvoer. `Duiven kunnen geen kanker krijgen', heette het. Hier een stukje uit Wikipedia:

\begin{quote}
Als duivenmelker viel hem op dat de kwaliteit van voeding de weerstand en prestaties van duiven kon verhogen. Duiven konden volgens hem geen kanker krijgen, en daaruit trok hij de conclusie dat een aangepast dieet kanker bij mensen zou kunnen genezen. In 1939 ``genas" hij op deze manier zijn eerste patiënt, hoewel buitenstaanders twijfelden aan het feit of deze patiënt wel kanker had. 

Moerman ondernam in het begin regelmatig pogingen om zijn therapie onderzocht te krijgen maar hij vond geen weerklank bij de reguliere geneeskunde. In 1955 kwam zijn therapie in een stroomversnelling toen in het Dagblad voor de Zaanstreek, \emph{De Typhoon} een artikel over hem verscheen. De populariteit van Moerman nam daarna met sprongen toe, en in 1956 werd daarom een commissie in het leven geroepen onder leiding van de huisarts Delprat om de therapie dan toch te onderzoeken.
% \footnote{Bron: https://nl.wikipedia.org/wiki/Cornelis_Moerman, geraadpleegd voorjaar 2016.}
\end{quote}

Door dat artikel in \emph{De Typhoon} zullen mijn ouders wel op Moerman attent zijn gemaakt. We zijn er enkele keren op bezoek geweest, wachtkamer vol patiënten. Volgens mij is de kanker toen weggebleven bij m’n vader. 

Maar het ging niet goed met hem. Hij kreeg hartproblemen en moest bloedverdunners slikken. Hij kon op het laatst geen honderd meter meer lopen zonder even stil te moeten staan. Hij was altijd sportief geweest en dat hij er zo aan toe was deed hem veel pijn.

In die tijd was ik veel in de weer met allerlei bandjes en onttrok me wat meer aan het huiselijk gebeuren. Moe werkte ’s avonds als schoonmaakster bij Zaanchemie (overgenomen door Scado N.V.) in de Westzijde en Pa en ik deden de afwas. 

En op zo’n avond, na de afwas stoeiden we wat en kreeg pa een hartaanval. Hij viel plots en kwam met een harde knal met z’n hoofd op de grond terecht. Toen het me duidelijk werd dat het ernstig was heb ik Harry van boven geroepen. Via de buren Zeeman werd er de huisarts gebeld en was het afwachten. Moe gebeld op haar werk. Niemand wist iets van re-animeren, dus je staat er maar in al je onmacht. Paniek alom. De huisarts heeft nog wat geprobeerd met een injectie in het hart, maar uiteindelijk was de conclusie: overleden. Gerard werd ook gebeld en die kwam ’s avonds laat nog aan. 

Een rare avond en ook volgende dagen. Pa werd beneden in een kist opgebaard en allemaal mensen die langs kwamen. Af en toe ging ik even naar beneden om naast de kist te zitten en te kijken. Het is net of de hele wereld door gaat en jezelf in de tijd vast staat. Je bent maar bezig met hoe het was en hoe het verder moest zonder dat je een richting wist. Vanwege de commotie rookte ik officieel m’n eerste sigaret. Stiekum rookte al heel wat langer. Als ik nog wakker was, sloop ik uit m'n kamer de trap af, zorgvuldig vanwege het kraken, en pikte dan wat shag uit m'n vaders tabaksdoos en wat vloeitjes en kon de volgende dag weer wat roken. Toen ik een keer in de badkamer, hangend uit het raam rookte, kwam Harry net thuis en vertelde dat ik wel een vuurtoren leek.

De begrafenis was aan de Zuiddijk tegenover mijn vaders oude werkgever. Iedereen die wilde kreeg verlof om naar de begrafenis te gaan, dus stond het vol met ex-collega’s van de scheepswerf. 

\section*{Hof van Holland 40, vervolg} % (fold)
\label{cha:hofvanholland2}

M’n tweede jeugd, na de Rosmolenstraat, heb ik doorgebracht in Hof van Holland tot en met de trouwerij met Olga. Want net als Harry en Jannie hebben wij daar de eerste jaren op een kamertje ingewoond.

\begin{figure}
\centering
\includegraphics[width=\textwidth]{img/ch22/joho_0001}
% \caption*{\footnotesize }
\end{figure}

\clearpage
~
\clearpage
\chapter{Rock aan de Zaan}

\thispagestyle{empty}
\begin{flushright}
\begin{figure}
\includegraphics[width=0.8\textwidth, right]{img/epi/epi3}
\end{figure}
\end{flushright}

\section*{Rock aan de Zaan}

\noindent Geachte lezer, ik moet een bekentenis doen: Het optreden voor een publiek is het leukste wat ik ooit heb ervaren. Laten we beginnen.

Hoe kom je aan muzikaal- of ritmisch gevoel? Geen idee. Hoewel? Tijdens recent onderzoek kwam ik te weten dat mijn grootvader in de dertiger jaren in een aantal muziekgroepen zat. Veel muziek werd er thuis niet gespeeld. Wel platen, 78-toeren, zoals Edmundo Ross met `London is the Place for Me', of Johnnie Ray, een crooner met `Cry' en `Walking in the Rain'. De pianiste Winnifred Atwell kon er wat van, je kreeg veel piano voor je geld. Ik kan ze nog dromen. Gek hoeveel muziek je kunt onthouden. Als ik de beginmaten hoor kan ik zo verder fluiten, al is het zestig jaar terug. 

\begin{figure}
\centering
\includegraphics[width=\textwidth]{img/128-koopman}
\caption*{\footnotesize Koopman op de Gracht}
\end{figure}

De meeste muziek kwam van de radio natuurlijk. Eerst hadden we draadomroep met een schakelaar met vijf of zes stations en één luidspreker. 

Later kregen we een echte buizenradio waar je de hele wereld op kon ontvangen. Zo maar uit de lucht geplukt door de radio. Korte-, midden- en lange golflengtes. Je had mysterieuze zenders die wel uit het heelal leken te komen. En onze eigen geheime zender `Radio Moskou’. Genoeg over het wonder van de radio.

Mijn muzikale loopbaan begon als leerling van ome Freek Baas, vader van mijn vriend en kleuterschoolklasgenoot Peter. Ome Freek gaf klassiek gitaarles. Als ouders bevriend zijn is het al snel een `ome'. Ik kreeg les boven, waar ook nog een bas stond. Soms deed Peter ook mee, op de klarinet. Ik heb geen idee meer hoelang dit heeft geduurd. 

Na het plotseling overlijden van ome Freek vond ik onderdak bij de muziekschool van Rien Waerts aan de Zuiddijk. Maar daar werden geen noten gelezen. We kregen les in een zaaltje dat achter de winkel aan de Zuiddijk lag. 

Dit keer was ik niet de enige. In tegenstelling tot de privélessen bij ome Freek zaten er hier wel een stuk tien à vijftien kinderen in het zaaltje. Allemaal met een gitaar, in rijen van drie achter elkaar. 

Als eerste werden ons het C, F en G-akkoord geleerd. Op commando barstte “Faria, Faria, Faria”, vijftienvoudig door het kleine zaaltje. Wat een vreselijk lawaai, alles door elkaar. Dat was heel erg en ik heb het niet lang volgehouden. Nog even heb ik een periode les gehad bij de andere muziekhandel Eshuys, met lessen van moeder Eshuys. Ik heb de jonge Margriet nog wel om een hoekje zien kijken. 

\begin{figure}
\centering
\includegraphics[width=\textwidth]{img/ch23/Eshuijs}
\caption*{\footnotesize De winkel van Eshuys is die met het scherm}
\end{figure}

Iets wat we nu niet meer kennen waren de openbare optredens van drum-, fanfare- of harmoniekorpsen, of zelfs van muzikanten die langs de deuren kwamen. 

Ik vond het door de straat komen van zo’n harmonieorkest bijzonder opwindend. Zodra je het eerste geluid in de verte hoorde hoopte je dat ze ook door onze straat zouden komen. Zo’n tambour-maître die voor het spul uitloopt en de maat en de richting aangeeft, met dan het hele korps er in de maat achteraan. Dat massale geluid als ze allemaal gingen blazen ondersteund door het slagwerk. Als je dan, nadat je een tijdje had meegelopen, weer huiswaarts keerde, liep je nog in een marstempo.

\begin{figure}
\centering
\includegraphics[width=0.8\textwidth]{img/131CzPeter3.jpg}
\caption*{\footnotesize Drumband Czaar Peter}
\end{figure}

Tijdens een van die schaarse optochten liepen we mee met de stoet, dansend en huppelend over de stoepen, toen Peter Baas een oplawaai kreeg van een van de trommelaars. Hij werd vol op z’n oog geraakt door een overslagstok van zo’n trommelaar. 
Dat kan flink aankomen. Niemand van het korps merkte het, want met al dat lawaai hoor je de pijnkreet van zo’n jongetje natuurlijk niet.

Van een ander kaliber waren de straatmuzikanten. Zij probeerden hun inkomsten op te halen door langs de huizen te gaan. Het waren voornamelijk violisten, waarvan er één ‘de danser’ werd genoemd. Hij maakte dansjes terwijl hij speelde. Ik heb hem ook wel bij bouwwerken zien staan. Voor de bouwvakkers was hij een onderwerp van plezier en wat spot, in ieder geval een leuke onderbreking.

% \section*{De muziek in} % (fold)
% \label{cha:muziek_in}

\section*{Es geht loss} % (fold)
\label{cha:gehtloss}

Hoe ik bij het gezelschap van Siem Vonk ben beland weet ik niet meer. Ik was wel bekend met Vonk omdat hij naast mijn tante Lena in de Czaar Peterstraat woonde. Mogelijk is die connectie van belang geweest. Ik kon meedoen met een variétéachtig gezelschap van jongeren onder leiding van Siem als dirigent. Een wonderlijk groepje. Een allegaartje van jongeren die iets speelden, met Siem ervoor als dirigent.

Rob Hendriks, die de Zaanse Popgeschiedenis optekende in het \emph{Noord Hollands Dagblad}, liet henzelf aan het woord. Hier spreekt de drummer, Ruud Meijns: 

\begin{quote}
Onze manager/dirigent was Siem Vonk. We repeteerden in de voorkamer van de ouderlijke woning van Vonk, in de Czaar Peterstraat in Zaandam. We traden regelmatig op in de Zaanstreek, ik herinner me de Jonge Prins in Wormerveer, en de zaal van de Lindeboomschool in Koog aan de Zaan, en een enkele maal buiten de Zaanstreek, onder andere op een feestavond in kamp Bakkum samen met ``The Allround Players" van Ed Vermeulen.
\end{quote}

Onder welke naam we speelden is mij ontschoten. We traden ook op voor weeskinderen van Oosterweide, waarbij we voor de stemming clowneske bewegingen en blokfluiten met talkpoeder tussen de muziek inlasten, met veel lawaai en vallen. 

\begin{figure}
\centering
\includegraphics[width=\textwidth]{img/134-Oosterweide}
\caption*{\footnotesize Oosterweide}
\end{figure}

Een paar maal speelden we voor bejaarden. Dit gebeurde in samenwerking met de Coca-Cola, die dan voor zonnekleppen en versnaperingen zorgde. Als we opkwamen zagen we een zaal vol cola-zonnekleppen. De meeste sliepen al. Ik speelde eerst gitaar, maar door het wegvallen van de drummer werd ik als zodanig ingeschakeld. 

\begin{figure}
\centering
\includegraphics[width=\textwidth]{img/135-rollers-foto1}
\caption*{\footnotesize The Rollers, Ed Vermeulen is de tweede van links}
\end{figure}

Via \emph{The Rollers}, volgens mij \emph{de} band van dat moment in de Zaanstreek, kon ik van Ed Vermeulen een overtollig drumstel kopen en was ik drummer. Gelukkig niet voor erg lang, want ik had dan wel ritmegevoel, maar absoluut geen techniek. Ik was altijd blij als een nummer afgelopen was. Doodmoe probeerde ik m'n armen te ontspannen, maar de tijd was altijd tekort. Siem Vonk had de gewoonte te dirigeren. Volgens ons droeg dat niets bij aan het succes. Wij wilden meer en een rockgroep met een dirigent vonden we niks, dus: uit elkaar. 

\section*{Blue Rocking Strings} % (fold)
\label{cha:bluerocking}

Arnout Rol in het \emph{Noord Hollands Dagblad}: 

\begin{quote}
In 1961 speelde ik klarinet in een bandje genaamd The Blue Rocking Strings. De andere bandleden waren Jan Abbing (sologitaar en zang), Nico Bakker (gitaar), Jan Krook (gitaar), Ruud Meijns (drums), en de zangeressen Jolanda Vergouw en Mieke Schneider en basgitarist Arjan Hooft. Jan Hos was ook betrokken bij de organisatie rondom de band. [...] Het bandje kwam voort uit een schoolorkestje van de toenmalige IVO-school [Individueel Voortgezet Onderwijs] in Zaandam, waarin Jan Abbing, Nico Bakker en Arnout Rol destijds (1960) speelden. We speelden eerst onder de naam The IVO Rocking Stars en later dus als The Blue Rocking Stars.
\end{quote}


\begin{figure}
\centering
\includegraphics[width=\textwidth]{img/ch25/bluerocking}
\caption*{\footnotesize De Typhoon, zaterdag 9 september 1961}
\end{figure}

\noindent Ik (Ruud) zat toen al niet meer op school, denk ik, maar ik kende iedereen van de ULO. Ed Eichenberger, Henny van Wijngaarden (later The Rolling Four), Henk van de Wissel (later de Rocking Idolaters), Mieke Schneider en Jolanda Vergouw zaten daar ook allemaal op, of hadden erop gezeten. Zo komen die bandjes bij elkaar.

\begin{figure}
\centering
\includegraphics[width=\textwidth]{img/ch25/nico}
\caption*{\footnotesize Nico Bakker (met gitaar) in 1960 tijdens de IVO-school periode.}
\end{figure}

Via de ULO vormde ik met vriend Ed en een buurjongen een tijdje The Flyland Four, naar de Vlielandstraat waar Ed en de buurjongen woonden. De bas was een echte theekist. Ik geloof dat we een keer in de Lindenboom gespeeld hebben, in een revue van volleybalclub De Molenwiek. 

In deze samenstelling trokken we over de wereld en speelden de muziek die bij een stad paste. We eindigden in Amsterdam met `Aan de Amsterdamse grachten'. Enkelen van ons groepje waren bij Hawaï al achter een palmboom in slaap gevallen van de drank.   

Een aardig intermezzo vormen de bezigheden van Dick Eichenberger, de broer van Ed. In die tijd was hij bassist in de jazzformatie Les Routiers du Jazz. Als wij bij de buren met de Flyland Four repeteerden konden we soms zijn studie aan de bas horen. Favoriet was bij ons Het Wilhelmus dat hij zong terwijl hij de bas zichzelf strijkend begeleidde. 

Dick's formatie trad wel op, denk ik, maar er is niets van bewaard gebleven. Memorabel was het avondoptreden in de tribune van de Kooger Football Club (KFC). Wij waren de enige toeschouwers: Ed, ik en Nel Deyle, de dochter van loodgieter. Zij was zangeres van de band en werd door elke jongen begeerd. De drummer was Theo. Theo had al een paar biertjes op en was vol vuur en vlam toen hij aan z’n solo begon. Het probleem was dat hij tijdens z’n solo geen maat hield en wij hem met flink maat stampende voeten terug in het stuk moesten brengen. We hadden het hele voetbalveld, inclusief de tribune, voor onszelf. 

In die tijd zijn we ook nog op vakantie naar Frankrijk gegaan, dat wil zeggen Dick en Ed Eichenberger, de trompettist Carl Kalf en ik. Met de Renault Dauphine van Pa Eichenberger. Het was wel wat krap inclusief bagage en tent. 

We zijn tot Fécamp gekomen, meen ik. De eerste avond kwamen we laat op de camping aan, zetten de tent op en gingen het dorp in. De volgende morgen werd ik wakker en merkte dat m’n hoofd binnen en m'n lijf buiten de tent lag. We stonden op een hellinkje. Veel Benedictine gedronken, ruzie gekregen met een snackbarhouder die dacht dat we Duitsers waren. \emph{De} oorlog zat er daar nog goed in. (Het is allemaal goed gekomen.)

\section*{The Rolling Four} % (fold)
\label{cha:rollingfour}

In de periode daarna hebben we nog verschillende pogingen tot het vormen van een bandje gedaan. Enige tijd heb ik gekeken bij een groep die repeteerde bij Koert Groot thuis in de Oostzijde. Dat was vlakbij Hof van Holland, waar ik toen woonde. Wat ik me daarvan voornamelijk nog herinner is het nummer `I’m gonna knock on your door, ring on your bell' waarbij de drummer, Co Smit, een elektrische bel onder z’n drumstel had gemonteerd.

\begin{figure}
\centering
\includegraphics[width=\textwidth]{img/142-RF}
\caption*{\footnotesize Ed, Co, Piet van Petten, Gonnie Rijkers et moi.}
\end{figure}

Ik ontmoette Co via mijn werk bij Apotheek Hulp Artsen, een pillenfabriek in Zaandam. Co had me uitgenodigd eens langs te komen. Andere leden van groep waren, naast Koert, die zang en gitaar deed, Piet van Petten als sologitarist en Luc Smit.  

Ze speelden topmuziek, goed geluid. Veel Cliff en The Shadows, inclusief de danspasjes. Ook deden ze Amerikaanse popmuziek. Vanuit deze groep vormde zich The Rolling Four. Ik verving Luc, waarom weet ik niet meer.

\begin{figure}
\centering
\includegraphics[width=\textwidth]{img/ch33/four}
\caption*{\footnotesize De eerste bezetting van The Rolling Four in actie met van links naar rechts Piet van Petten, Koert de Groot, Luc Smit en Co Smit.}
\end{figure}

Co was de oudste en was de leider van de groep. Hij woonde in de Javastraat in Wormerveer bij snackbar Java (hele goeie patat, overigens). Co had een rijbewijs en een auto, een Opel Olympia. We hebben heel wat reisjes gemaakt in dat Opeltje, waar iedereen en alles in moest. Wat een prachtig autootje!

Toen dat een beetje begon te lopen hebben we onze kledingkeuze in Amsterdam gemaakt bij Tip de Bruyn in de Kalverstraat. Een zwarte broek en jasjes in verschillende kleuren. We deden veel werk van The Shadows, keurig in een rijtje en dezelfde pasjes om het wat leuker te maken. De Four werden later uitgebreid met Gonnie Rijkers als zangeres. 

\begin{figure}
\centering
\includegraphics[width=\textwidth]{img/ch26/opel_olympia_big}
\caption*{\footnotesize Opel Olympia, ook al zo'n mooi karretje}
\end{figure}

We repeteerden bij de oma van Co in de Dubbele Buurt te Wormerveer. We traden regelmatig op in onder andere Café Jongejans in Assendelft, in het Moriaanshoofd in Wormer. Een keer speelden we voor slechts een paar meisjes, die hebben we ook nog met ruzie de deur uit hebben gewerkt. 

\begin{figure}
\centering
\includegraphics[width=\textwidth]{img/ch26/Moriaanshoofd}
\caption*{\footnotesize Moriaanshoofd in Wormer}
\end{figure}

Ook speelden we in een of andere tent in Beverwijk. Daar was het heel leuk spelen. Eens verscheen zigeunerkoning Koko Petalo met een uitgebreid gevolg van vooral mooie meiden. Op een gegeven moment zei hij: ``Geef mij die gitaar maar eens, dan zal ik hem laten huilen''. Hij kon er echt geen hout van, waarop ik zei: ``Geef maar weer terug, dan zal ik hem weer laten lachen.'' Hij waardeerde dat en zo mochten we met de dames dansen in de pauze.

In die tijd hebben we nog contact gehad met Bob Bouber, die toen zanger was bij ZZ en de Maskers. Hij had wat meer groepjes onder zich en kwam ons bekijken. Maar omdat hij allerlei wisselingen in de groep wilde, hebben we van verdere samenwerking maar afgezien. 

We hadden het idee dat we meer de popkant op wilden en minder de dansmuziek. Toen zijn we met z’n drieën verder gegaan; Henny van Wijngaarden kwam in de plaats van Co Smit. We hebben zelfs nog eens een groep gehad van zeven man.

\begin{figure}
\centering
\includegraphics[width=\textwidth]{img/ch26/1965}
\caption*{\footnotesize We werden steeds groter, in ieder geval in aantal}
\end{figure}

Uiteindelijk verdween ook Gonnie, die een solocarrière begon met Bob Bouber onder de naam `Deedee Pit'. Ze heeft nog een aantal plaatjes bij hem gemaakt.

Ed Eijchenberger: 

\begin{quote}
In het voorjaar van '62 ging ik voor 14 dagen naar Londen, i.h.k.v. een soort uitwisseling van jongeren. Ik was in huis bij een Engelse jongen, die net Please Please Me van de Beatles (wie zijn dat?) had gekocht. Het Beatlegeweld was daar net aan het losbarsten. Het continent moest nog volgen. 

Ik werd meteen gepakt door het specifieke geluid van de merseybeat. De ritmische accenten kwamen anders te liggen, het fluwelen gitaargeluid verdween en er kwam veel meer harmonische zang in. Met die ervaring kwam ik enthousiast terug. Die kant wilden wij toen ook op en we begonnen toen al met de echte “beatmuziek”. Een groot verschil met de gladde jaren vijftig rock\&roll die alle bandjes in die tijd speelden.\footnote{(in de krant)}
\end{quote}

\begin{figure}
\centering
\includegraphics[width=\textwidth]{img/147-MoriaanA}
\caption*{\footnotesize In het Moriaanshoofd in Wormer. Van links naar rechts: Piet van Pette, Ed Eichenberger, Henny van Wijngaarden, Ruud Meijns.}
\end{figure}

Later, toen Henny al in de Rolling Four zat, ben ik nog eens een weekeinde bij hem op Texel gaan logeren. Zijn ouders hadden daar een huisje gehuurd. Op de brommer naar Texel, verliefd geworden op een meisje uit Utrecht, haar nog eens opgezocht maar elkaar op het station al misgelopen. Geen goed voorteken, lijkt me.

\section*{The Devotions} % (fold)
\label{cha:devotions}

The Devotions zijn ontstaan uit leden van twee Zaanse bandjes: The Rolling Four en The Rolling Devotions. In maart 1965 spraken Jan Abbing en ik over het opzetten van een nieuwe band uit leden van beide groepjes. Jan had me gevraagd om in te vallen voor een gitarist die in dienst moest (een veel voorkomend euvel op die leeftijd). Vanuit de Rolling Four kwamen Henny van Wijngaarden, Ed Eijchenberger en ik over; Jan Abbing en Nico Bakker kwamen uit The Rolling Devotions. Vanwege het karakter van de band---een zanger met een band---werd gekozen voor de naam: `John Hatton and the Devotions'. (Jan Abbing was John Hatton. De rest van de naam is afgeleid van het Everly Brothers nummer `Devoted to You'.)

\begin{figure}
\centering
\includegraphics[width=\textwidth]{img/ch27/promo3a}
\caption*{\footnotesize The Devotions}
\end{figure}

We repeteerden in de kantine van de rolschaatsbaan St. Theresia aan de Leo de XIIIe-straat. Ondanks het feit dat we nog maar een paar maanden bestonden besloten we mee te doen aan het Brasemconcour in Roelofarendsveen. Volgens ons was het optreden een kleine ramp want we begonnen met het verkeerde nummer en tijdens het optreden besloten we over te gaan naar het bedoelde nummer. Wat overigens met één knik naar elkaar goed lukte. 

\begin{figure}
\centering
\includegraphics[width=\textwidth]{img/4-devotions1}
% \caption*{\footnotesize The Devotions}
\end{figure}
\begin{figure}
\centering
\includegraphics[width=\textwidth]{img/4-devotions2}
% \caption*{\footnotesize The Devotions}
\end{figure}
\begin{figure}
\centering
\includegraphics[width=\textwidth]{img/4-devotions3}
% \caption*{\footnotesize The Devotions}
\end{figure}

Dit in onze ogen dramatische optreden deed ons besluiten toch maar op vakantie te gaan. Voor de jury was het echter voldoende om ons een prijs toe te kennen. Radio- en TV-opnames werden ons beloofd en we mochten een plaatje maken. Maar de één zat op Texel en de anderen waren verspreid over Europa. Toen we weer bij elkaar waren brachten we op een zaterdag nog een nachtelijk bezoek aan het Spui, waar zich een happening afspeelde. Tot onze verrassing werd Ed Eijchenberger opgepakt en uiteindelijk berecht.

Daarover kan ik het volgende vertellen. Die avond waren we bijeen gekomen in Het Spinnenwiel, een café waar de `intellectuele elite' van Zaandam samenkwam. Ed en ik troffen die avond onder andere Sarah, nog wat mensen en iemand met een auto. Sarah Jansje Duys was de vriendin van Rob Stolk, de bekende Zaanse anarchist die bij Provo betrokken was. Eerst trokken we met wat drank naar het In ’t Veldpark tot Sarah vertelde dat er die avond nog een happening aan het Spui zou zijn. Daarop besloot de hele ploeg naar Amsterdam te vertrekken. 

Alles was aanwezig voor een fijne avond: omstanders en politie. Wij stonden voor de Atheneum Boekhandel het schouwspel te bekijken, deden mee aan het ‘ge-uch’ van de verstokte roker, het Lieverdje brandde al. De politie wilde iedereen uiteendrijven. Plots stopte er voor ons een politiebusje en werd Ed vanuit het publiek het busje ingetrokken. Wij nog proberen om hem er weer uit te trekken, maar dat mislukte en het busje weer weg. Huh?! Je bent in een shock. Waarom, hoe kan dit? Wat nu? 

Dat laatste werd voor ons beslist, omdat de politie met motoren de trottoirs begon schoon te vegen. Ons groepje werd uiteen geslagen. Met Hans, een Indische jongen, kwam ik terecht in de smalle Huidenstraat naar de Singel. Daar kwam ons een motor met zijspan tegemoet, dus gingen we weer terug naar het Spui. We hadden drie keuzes: het Spui richting Singel waar nog behoorlijk gevochten werd, de politiemotor, of het Spui richting Centraal Station. We hebben maar voor het laatste gekozen, want daar stonden maar een paar agenten. 

\begin{figure}
\centering
\includegraphics[width=\textwidth]{img/152-duijn-stolk}
\caption*{\footnotesize Voor het huis van Van Duijn. Links Roel van Duijn, derde van rechts Rob Stolk, en helemaal rechts Rens Adelaar, ook uit Zaandam}
\end{figure}

Toen we het op een rennen zetten kwamen er uit de zijstraatjes nog een flink aantal agenten het Spui op. Al rennend kregen we nog een paar flinke meppen met de wapenstok en o wonder, de rest van ons groepje stond te wachten. Sarah nam ons mee naar het huis van Roel van Duyn (foto)waar we de nacht op zolder doorbrachten. Ed moest voorkomen en werd uiteindelijk veroordeeld voor ‘er zijn’ denk ik. 


Die volgende ochtend zijn we naar Zaandam gereisd en ben ik naar Ed’s ouders gegaan om het nieuws over te brengen. 

\begin{figure}
\centering
\includegraphics[width=\textwidth]{img/ch27/153aug65EdE}
\caption*{\footnotesize Een Provo-knipsel over het proces}
\end{figure}

Daarna ben ik nog een paar keer met Sarah en haar zus Henny op stap geweest. Met Henny ging ik naar een feest nabij het Tropenmuseum. Toen we het pand betraden lagen er al vanaf de eerste verdieping mensen in gangen en kamers te blowen, te snuiven. Er was weer een lading aangekomen. Op drie hoog had de bewoner een bed aan het plafond bevestigd en zweefde zo door de kamer. Dat was m’n eerste goeie kennismaking met marihuana. Op de terugweg de volgende ochtend, was ik nog niet geheel bekomen van de stuff, want ik kletste maar door over het houden van konijnen (volgens derden).

Intussen kwam er van allerlei kanten belangstelling voor het management van onze band. Er was zelfs sprake van dat we met het bureau van impresario Paul Acket, die later nog het North Sea Jazz Festival heef opgezet, in zee zouden gaan. Ik had in ieder geval m’n baan opgezegd. Door het uitblijven van grote contracten ben ik ’s avonds gaan schoonmaken via van Heusden en kaas gaan inpakken bij Simon de Wit. Toch hielden we wat over aan het concours.

Wat de singletjes betreft kwamen we bij platenlabel Dureco terecht, met als producer Cees van Zijtvelt; een bekende dj van Veronica enzo. We mochten in de Decca-studio in Brussel de opnamen maken. Ter ondersteuning hadden we Carl Kalf voor op de trompet meegenomen. Uiteindelijk zijn er twee singles verschenen: (Eigenlijk drie, want men had op één persing een verkeerd nummer gezet.) In Amsterdam werden bij studio Hartland professionele foto’s genomen voor de hoesjes. De bekende fotograaf Claude van Heye fotografeerde ons voor een blad bij het Krimp.

\begin{figure}
\centering
\includegraphics[width=\textwidth]{img/4pieces}
\caption*{\footnotesize Een onduidelijke foto uit Brussel}
\end{figure}

Voor de NCRV deden we een TV-opname, die we in een cafeetje ergens in het land nog hebben kunnen zien. Gelukkig kwam Ed in 2010 via een van zijn zoons met een opname van dit optreden. Ed: 

\begin{quote}
Opvallend was in ieder geval dat vele organisatoren/zaaleigenaren ons steeds weer terugvroegen. Of dat nu kwam omdat we flink publiek trokken (dat hoop ik maar) of omdat we ons lieten onderbetalen, dat weet ik niet. Wat ik wel weet is dat deze optredens meestal voor ons en ik denk ook voor het publiek, een echt feest waren. Dat bleek ook wel uit de fanmail die we (Jan vooral) dan ontvingen.
\end{quote}

Ik herinner mij de zaal Groot in Medemblik, waar we na het optreden altijd uitgebreid onthaald werden met gebakken eieren en andere lekkernijen. Na zo'n optreden lustte je wel wat, want we gaven ons dan wel helemaal. De adrenaline gierde nog lang door je aderen. Ook herinner ik me nog Maasdijk in het Westland, waar we met een touringcar naar toe trokken, samen met zo'n vijftig fans. 

\begin{figure}
\centering
\includegraphics[width=\textwidth]{img/156-Hartland3.jpg}
% \caption*{\footnotesize }
\end{figure}

Die zaken werden prima georganiseerd door Marja, de vriendin en later vrouw van Jan Abbing. We speelden ook bij de Kruisweg in Marum, provincie Groningen, waar pas daarvoor The Kinks nog hadden opgetreden.

In Stadskanaal kregen we nog onenigheid met een paar meisjes in het publiek, die dan ook de zaal verlieten. (Gelukkig niet allemaal.) Waar het door kwam weet ik niet meer, maar dat we qua taal op een andere golflengte zaten telde wel mee. 

Er waren ook heel veel fijne optredens in de Zaanstreek, Heemskerk, Purmerend, Bergen, Arnhem (waar Hennie z'n drumsticks niet mee had), Rotterdam, Groningen en weet ik waar allemaal. Ten zuiden van de grote rivieren zijn we dacht ik nooit geweest. 

We speelden overal door het land. In november 1966 kregen we een contract in de nachtclub Tudor aan de Nieuwe Binnenweg in Rotterdam voor een maand. We speelden vijf (of zes?) dagen van 9 uur `s avonds tot `s morgens 4 uur. Zo bouw je in ieder geval heel veel routine en kennis van het nachtleven in Rotterdam op. Na deze maand deden we het nog een maand; wegens succes geprolongeerd. Wij hadden kamers op de Kruiskade en hebben bijna geen daglicht gezien omdat we in de winter optraden. Ed: 

\begin{quote}
Gedurende enkele maanden hebben we 6 dagen per week gespeeld in dancing/nachtclub Tudor in Rotterdam. Dinsdags t/m donderdags een nachtclub, maar in het weekend een dancing waar uitsluitend uitgaansjeugd kwam.
\end{quote} 

\section*{Rotterdam} % (fold)

Toen we veel in Tudor speelden woonden we in Rotterdam in een pension tegenover de bioscoop Arena. We verbleven er voornamelijk van de vroege ochtend tot de late middag. Na het ontbijt gingen we in pyjama het nieuwe nummer voor die dag instuderen, even de stad in, eten in restaurant Shell (daar hadden ze lekkere gebakken lever en een leuke cheffin). Dan weer ruim voor openingstijd in de zaal om het nieuw ingestudeerde nummer nog gauw een paar keer te oefenen. 

Ons repertoire bestond op een gegeven moment uit enige honderden nummers, zowel covers als nummers van Jan Abbing, want die was altijd al creatief en schreef nogal wat. Ik wil ons niet te veel op de borst te kloppen, maar na twee weken stond het in het weekend zwart van het volk in die tent. Ook werden we goed bekeken door plaatselijke en Haagse bandjes. Ik weet nog dat de nieuwe Beatles single `Daytripper/We Can Work It Out' zou uitkomen. Eén dag voor de officiele release speelden we hem al. We hadden een dealtje met een plaatselijke platenzaak die hem al een paar dagen in huis had, zodoende. Reputatie gevestigd. 

Ook gingen we na sluitingstijd om drie uur `s nachts nog wel stappen. Dat kon in Rotterdam ook toen al heel goed. Eén keer heb ik toen heel onbezonnen een glas jenever dat op tafel stond in Hennie z'n pilsje gegooid. De gevolgen bleven natuurlijk niet uit. In de vroege ochtend liepen we naar huis en plots werd Henny niet lekker. Hij ging op de grond zitten en zei heel dramatisch ``Laat me hier maar zitten, ga maar''. Maar het sneeuwde, dus hebben we hem onder de arm gepakt en meegenomen. (Ik heb er nog steeds spijt van, dus als je dit leest: bij deze nogmaals sorry, Hen!) Nico was er die ochtend al niet meer bij. Hij was meegenomen door een oudere serveerster die op hem viel en hem op een of andere manier had uitgedaagd en Nico tot een `Ja' verleidde.

Verder was het toch een tamelijk gedisciplineerd bestaantje met hard werken, en natuurlijk ook heel veel meisjes. Mijn veelvuldige relaties duurden meestal niet langer dan twee weken, waarvan sommige elkaar nog overlapten ook. Hoe liederlijk... Later is het overigens allemaal toch wel goed gekomen met mij. 

Door onze populariteit in de Tudor drongen allerlei mensen zich aan ons op. Zo kregen we een `bodyguard'. Hij---een vent type klerenkast---liep dan `s nachts als we naar huis gingen met ons mee voor de veiligheid. Als we na sluitingstijd soms met de kelners meegingen naar een lokaal waar ook door pooiers en dames werd afgerekend, werden we gewaarschuwd ons om ons nergens mee te bemoeien, wat er ook gebeurde. En er gebeurde nog wel eens wat, bijvoorbeeld als er niet genoeg geld werd afgedragen of als mensen elkaar niet verdroegen. En natuurlijk de meisjes die gek waren op onze muziek, die waren er ook. Tot ons genoegen. 

Er kwam van alles in de Tudor Bodega. In een aangrenzend zaaltje, met veel pluche, kwam Jan een dame tegen. Hij was helemaal verkikkerd. Wij kijken, gezellig aan het tafeltje met de dame zitten kletsen en toen weer terug. Zegt Ed of Henny: ``Heb je niet naar z’n handen gekeken? Dat is een vent.'' 

\begin{figure}
\centering
\includegraphics[width=\textwidth]{img/4-TudorR}
\caption*{\footnotesize Nieuwe Binnenweg met rechts een stukje Tudor Bar.}
\end{figure}

\section*{Duitsland} % (fold)
\label{cha:duitsland}

Na Rotterdam vertrokken we naar Duitsland. In Soest, bij Dortmund, speelden we in de Oase, de club van Frau Krückel. Weer zes dagen per week, van 9 uur `s avonds tot 4 uur `s morgens. Je leerde niet alleen routine, maar ook behoorlijk innemen. Bij Frau Krückel woonden we in kamers boven de dancing. Op muren waren nog sporen van andere groepen te vinden, waaronder The Thielman Brothers. We voelden ons opgenomen in de wereld van de grote namen. Het had ook wel wat. Frau Krückel was een kleine, tanige tante. Als er gelazer was in de heren WC schrok ze er niet voor terug om daar naar binnen te stormen en de lastpakken bij hun nekvel te pakken en de zaak uit te gooien.

\begin{figure}
\centering
\includegraphics[width=\textwidth]{img/ch29/double}
\caption*{\footnotesize Oase-bar in Dortmund. Rechts: Pete, Tony en Webber}
\end{figure}

Bezoekers waren lokale jeugd, maar vooral Engelse en Canadese militairen. Naast ons was er ook een vast koppeltje dames om te animeren. Bij het animeren werden wij van tijd tot tijd ook nog ingeschakeld, zeker als er veel te verteren was op de betaaldag van de militairen. De meeste drankjes werden door ons dan onder tafel op de vloer geleegd, anders zou het teveel worden want we moesten nog een hele avond spelen. Geleerd van de meisjes die meedeelden in de omzet. 

Ook in deze zaak hadden we snel weer fans en aanhang. Vooral de soldaten kwamen graag. Ze nodigden ons uit om naar hun kazerne te komen. Bij de Canadezen, met Henny, zagen we nog een film van Elvis in Hawaï. Ik heb in een lokale bios \emph{A Bridge Too Far} gezien, in het Duits. Dat was heel raar. Ook de Engelsen en Amerikanen spraken allemaal Duits. Gelukkig wonnen de geallieerden.

\begin{figure}
\centering
\includegraphics[width=\textwidth]{img/5Soest1A}
\caption*{\footnotesize Mijn bed}
\end{figure}

In Soest maakten we vrienden, gingen vaak Torten eten en haalden op de terugweg Bratwürsten bij het stalletje op het plein. We kookten zelf, en dronken des te meer. Frau Krückel moedigde ons aan om elke glaasje dat ons werd aangeboden ook aan te nemen en dat resulteerde al snel in een vorm van alcoholisme. Drank doet gekke dingen. Zo hebben we Nico eens in een kast opgesloten (geintje!) en hem daarna vergeten. Toen hij wakker werd kon hij bijna niet meer lopen van de kramp. Zelf ben ik eens wakker geworden, voelde om me heen en dacht ``Doperwten?'' M’n haar zat vol, maar ik had helemaal niets gemerkt. 

\begin{wrapfigure}{r}{0.5\textwidth}
\begin{center}
\includegraphics[width=0.48\textwidth]{img/165Soest1}
\caption*{\footnotesize Anita}
\end{wrapfigure}

En romances. Ik had een lokaal vriendinnetje, Anita. Geen idee wat ze overdag deed, maar ze stond ons vaak op te wachten en kwam ook naar de Oase. Er was nog een meisje dat mij wel erg claimde. Toen ik daar niet op in ging kwam ze met Beatle-teksten om mij terecht te wijzen. Ed deed het wat romantiek betreft nog beter. Hij leerde Ushi kennen, die een auto had. Ze raakten sterk bevriend en hij trok zelfs bij haar in. Na ons vertrek uit Duitsland heeft hij haar meegenomen naar Zaandam. Het is nooit iets permanents geworden.

\begin{figure}
\centering
\includegraphics[width=\textwidth]{img/164beatclub}
\caption*{\footnotesize Een schnabbel in de plaatselijke Beatclub}
\end{figure}

Op de dagen dat we vrij hadden schnabbelden we ook nog wat in andere tenten. In Beat Club Soest bijvoorbeeld. Geld heb ik nooit gezien. 

In andere tenten in de omgeving speelden ook bandjes. Soms gingen we kijken of kwamen zij bij ons. 	

Ed:

\begin{quote}
Door de Canadezen werden we later nog uitgenodigd om op te treden op een of ander jaarfeest van (kinderen van) Canadese militairen. Het was geen probleem voor ze om ons met een paar wagens even uit Nederland te halen en weer terug te brengen. Vet opgetreden zoals je nu zou zeggen. Daarna nog doorgezakt met de dochter van de Brigadegeneraal in hun landhuis aan de Möhnesee. Dat was onze tweede kennismaking met het leger. We hadden ook al eens opgetreden voor Amerikanen op de luchtmachtbasis Soesterberg. 

De derde kennismaking was met de Nederlandse Krijgsmacht. Die maakte een eind aan de pret, want ze konden blijkbaar niet zonder Hennie en mij. Ik was er weliswaar tamelijk snel weer uit dienst, maar het kwaad was reeds geschied. The Devotions hebben nooit meer gespeeld.
\end{quote}

Het sprookje uit. Klaar. Kort maar krachtig. Er waren nog stukken in de Zaanse krant over het einde en over de verloving van Ed met zijn Ushi uit Duitsland. Maar het was over.

Achteraf gezien hadden we nooit in het nachtclubcircuit terecht moeten komen. Je verkoopt lokaal wel wat plaatjes, maar verder kom je nergens. Als we in die Nederbeattijd furore hadden willen maken, hadden we elke dag ergens anders moeten spelen.

\begin{figure}
\centering
\includegraphics[width=0.8\textwidth]{img/ch29/rockadzaan8}
\caption*{\footnotesize Op de trap in het kantoor van dagblad de Typhoon waar Derk Peeters wat foto’s van ons nam voor een `Back in Town'- verhaal.}
\end{figure}

\section*{The Shameless} % (fold)
\label{cha:shameless}

Toen The Devotions stopten heb ik maar een baantje gezocht. Dat vond ik via bemiddeling van Henny Cornet, eigenaar van een kroeg in de Lage Horn. Ik kreeg werk als lasser en sloper bij een losse ploeg bij van Hattum en Blankenvoort in Spaarndam.

Ook probeerde ik weer een bandje te vinden. Enkele keren heb ik gerepeteerd bij The Rocking Idolaters (met Grad IJff, Henk van de Wissel), maar uiteindelijk werd het The Shameless met Jan en Jaap Stompé, Nico Way en Jan Visser (welke ook in de Skunks speelde). De manager was Joop Koekkoek. Joop deed in het begin alles nog op z’n brommer en tufte door heel het land. Later kreeg hij een behoorlijk groot theaterbureau. 

\begin{figure}
\centering
\includegraphics[width=\textwidth]{img/ch25/ri-2}
\caption*{\footnotesize Rocking Idolaters, met Grard IJff en Henk van der Wissel}
\end{figure}

\begin{figure}
\centering
\includegraphics[width=\textwidth]{img/169JoopB}
% \caption*{\footnotesize }
\end{figure}

Vader Stompé reed de bus en wij hingen achterin. Soms gingen Olga en Annie, de vriendin van Jan Visser, mee. Dan was het met Jan songs van Dylan reciteren, maar vooral heel laat thuiskomen en doodmoe je bed in.

Voor m'n baan werd ik in die tijd rond half zeven opgehaald door Willem in z’n volkswagentje, waarin nog wat makkers van de ploeg zaten. Maar als je om drie uur thuiskomt van een avondje spelen en om zes uur weer op moet, ben je niet al te fris. Willem wist altijd wel waardoor dat kwam. ``Weer aan de pretsigaretjes gezeten zeker?''

\begin{figure}
\centering
\includegraphics[width=\textwidth]{img/170biljet}
% \caption*{\footnotesize }
\end{figure}

In het begin was Nick Way onze zanger, maar door een geintje gaf hij er de brui aan. Jan en ik  vonden dat Nico wat de ster begon uit te hangen. Het was in Assendelft en als geintje deden we midden in een nummer iets wat op een afsluiting leek---gewoon \emph{pom-pom}. Daardoor was hij zo van z’n stuk gebracht dat we het nog eens deden. Toen vertrok hij.

Jan en ik namen de zang over en we speelden daarna veel ruiger dan wat ik in The Devotions gewend was, meer R\&B. We deden Otis Redding, The Kinks, The Zombies, Small Faces, dat soort muziek.

Net als zoveel bandjes trokken we door het hele land en zelfs nog een enkele keer naar Duitsland. Natuurlijk speelden we ook veel in de Zaanstreek. Soms was het wel heel apart, zoals een weekend op Texel. We speelden daar in Circus Sarasani, met onder andere de Bintangs en Cuby + Blizzards. Verder was het elke week de bus in en ergens spelen. Lange reizen heen en terug.

\begin{figure}
\centering
\includegraphics[width=\textwidth]{img/ch30/optournee2a}
% \caption*{\footnotesize .}
\end{figure}

Soms voel je je zo goed dat je denk alles te kunnen maken. Een keer in de Achterhoek speelden we in flinke zaal. Olga en Annie waren ook mee, die zaten aan tafeltje. In de pauzes gingen wij weer bij ze zitten. 

Maar een paar jongens uit het publiek pakten onze plaatsen in en deden dat ook de volgende pauze weer. Opeens had ik er zo genoeg van dat ik door de microfoon riep of die boeren eens plaats wilden maken. Dat was dom en onhandig. Het hele publiek keerde zich tegen ons. De meisjes snel achter het toneel en wij stonden klaar met de poten van de microfoon om ons te verdedigen. We zijn uiteindelijk onder begeleiding van de politie in ons busje afgevoerd.

\begin{figure}
\centering
\includegraphics[width=\textwidth]{img/ch30/film}
\caption*{\footnotesize Aankondiging \emph{Leve het Leven}}
\end{figure}

Als Zaanse band werden we uitgenodigd om mee te doen aan de opnames van een film die Lau Ruyter zou gaan maken. De opnames vonden plaats in de Lindenboomschool in Koog aan de Zaan. Het was een ambitieus plan, want alle bekende groepen waren op komen draven. Hoe ze dat financieel voor elkaar kregen weet ik niet, maar ik geloof niet dat wij er ook maar een cent voor kregen. De eer was al genoeg. Tijdens onze opnamedag stonden we met Cuby + Blizzards, The Haigs (toen nog met Barry Hay) op het podium. Van deze film is later nooit meer iets vernomen. Naspeuringen van de redacteur van de Zaanse Pophistorie, Rob Hendriks, hebben ook niet veel opgeleverd, behalve het volgende citaat op de site van Rob Hendriks: 

\begin{quote}
Het laatste woord is aan Emile Rempt van de band Don’t Know die ook optrad in de Lindenboom tijdens de filmopnamen: “We hebben destijds in de Lindenboom voor deze film gespeeld. Onze manager was Lau Ruyter, vandaar. Later zijn we na een probleem overgestapt naar manager Pierre Roth, die samen met Peter Schotvanger het geluid van die film zou verzorgen.

Van hem destijds wat meer inside info gekregen: de film was volledig fake. Het verhaal gaat dat er niet eens filmtape in de camera zat. Het sterke vermoeden was dat het meer een truc was om bands goedkoop of zelfs voor niets naar de Lindenboom te krijgen!
\end{quote}

Prachtig verhaal, echt iets voor Lau Ruyter.

We trokken naar Duitsland voor een optreden in Hamm. De reis was al bijzonder. Vlak voor de grens ontdekte Jaap dat hij z’n paspoort vergeten was. Na rijp beraad besloten we hem in de aanhanger te verbergen en zo de grens over te gaan. Eerst alles eruit, Jaap achterin de aanhanger en daarna alles weer opgevuld. De grens was geen probleem en een paar kilometer in Duitsland hebben we Jaap er weer uit gehaald. 

\begin{figure}
\centering
\includegraphics[width=\textwidth]{img/ch30/optourneeAs}
\caption*{\footnotesize Ik tijdens een eetpauze op weg naar een optreden. Ik droeg toen m'n lasbrilletje}
\end{figure}

In Hamm moesten we spelen in een enorme hal. Herman Brood met The Moan waren hier eerder geweest. Ze probeerden ons als de band van de \emph{Provo’s aus Amsterdam} te slijten. Ik geloof dat we ze aardig plat hebben gespeeld.

\begin{figure}
\centering
\includegraphics[width=\textwidth]{img/ch30/schamensich}
% \caption*{\footnotesize .}
\end{figure}

Op de terugweg volgde weer dezelfde procedure met Jaap. Eén probleempje deed zich voor toen een douanier merkte dat een achterlichtje van de aanhanger kapot was. Of we dat bij de eerstevolgende bezinepomp even wilden herstellen. Opgelucht kwamen we het vaderland weer binnen. Zeker Jaap, die alles vanuit die aanhanger hoorde.

Al die tijd zat ik overdag in de metaal in Spaarndam en dan in het weekend spelen. Dan kwam ik om 3 of 4 uur ’s nachts thuis en moest om 6.30 uur weer bij de Prins Bernhardbrug staan om opgepikt te worden voor de dagtaak. 

Ik vond het wel goed zo. De professionele gitaar heb ik aan de wilgen gehangen en ook nooit meer opgepakt. Er zijn nog wel verzoeken gekomen om opnieuw iets te formeren, maar voor mij was het genoeg. Enige jaren geleden heb ik nog een paar jaar klassiek les gehad. Veel leuke componisten leren kennen.

\begin{figure}
\centering
\includegraphics[width=\textwidth]{img/ch30/Devotions.jpg}
\caption*{\footnotesize Mijn gitaar werd door de heer Thomson gemaakt.}
\end{figure} 

\section*{Waar we zoal verder naartoe gingen} % (fold)
\label{cha:naartoegaan}

Wat ik me als vroegste bezoek kan herinneren is Huize Negrijn op de Hogendijk, waar regelmatig Jazzconcerten werden gegeven. Soms was het Provadja waar alternatieve popmuziek te horen was en soms was het een Jazzclub. Ik heb er Theo Loevendie zien spelen.

Het eerste grote optreden waar Olga en ik samen naartoe gingen was Flight to Lowlands Paradise in Utrecht, 1968.

\begin{figure}
\centering
\includegraphics[width=\textwidth]{img/180ohjeugd}
% \caption*{\footnotesize Mijn gitaar werd door de heer Thomson gemaakt.}
\end{figure} 

\begin{figure}
\centering
\includegraphics[width=\textwidth]{img/181-flightutrecht}
\caption*{\footnotesize Flight to Lowlands Paradise Utrecht 1968}
\end{figure}

In een grote tent speelden onder andere Pink Floyd, Bonzo Dog Doo-Dah Band, Tyrannosaurus Rex. Pink Floyd speelde met van die dia-vloeistofprojecties op de achtegrond. Later hoorde ik van Frans Room dat het zijn projecties waren geweest die daar te zien waren.

Het was vooral ontzettend druk. Je kon je alleen in een stroperige rij verplaatsen om bij een ander optreden te komen. 

Pink Floyd reeg het ene uitgesponnen nummer aan het andere. Wat me vooral is bijgebleven is de Bonzo Dog Doo-Dah Band, die er een grote circusact van maakten. We hebben veel om die gasten gelachen. Jimi Hendrix die ook was aangekondigd kwam niet, net zoals The Free.

Natuurlijk gingen we naar Frank Zappa, de grote meester. We hebben later nog verschillende van zijn concerten bijgewoond.

In Ahoy, Rotterdam, zagen we hem op 27 november 1971 voor The Flo \& Eddie tour. Al op de heenweg was het een avontuur, omdat de auto door z’n water heen was en we rokend langs de snelweg stonden. 

Toen we eindelijk bij Ahoy aankwamen liepen we wat om het gebouw heen en kwamen we Eddie, één van de zangers, tegen. Wat gekletst en hij zei dat ik wel mee naar binnen mocht om wat foto's te maken. De band was nog aan het repeteren. Er is en wielerbaan in Ahoy en op die baan liep de saxofonist wat te toeteren. Een waanzinnig beeld van een muzikant op die steile wielerbaan. Geen gelukte foto, mocht niet flitsen. Zappa was druk bezig en ik kon m’n camera bijna niet vasthouden van de zenuwen. Gelukkig heb ik nog wat van de repetitie gedaan en later het concert.

\begin{figure}F
\centering
\includegraphics[width=\textwidth]{img/ch31/zappa}
\caption*{\footnotesize Frank Zappa en band in Rotterdam. Boven repetitie, onder optreden}
\end{figure} 

Na afloop kwam Frank nog even terug en begon in z’n eentje aan een solo/improvisatie. Er waren al mensen vertrokken, anderen dachten misschien dat er iets via het systeem afgespeeld werd. Maar uit wat hij daar op toneel deed bleek maar weer eens wat een fabuleus gitarist hij was.

Ook zagen we Zappa nog een keer in Amsterdam. Soms weet ik niet meer precies waar het was, het Concertgebouw of de Stadschouwburg. Ik had Olga mee naar het podium genomen om haar samen met Zappa op de foto te krijgen. Kijk ‘m kijken. Mijn fantasieën zijn vaak opwindender dan de werkelijkheid.

\begin{figure}
\centering
\includegraphics[width=\textwidth]{img/ch31/zappaolga}
\caption*{\footnotesize Op deze foto is het haar van Olga links nog te zien.}
\end{figure}

Er zijn nog wat concerten geweest waar ik geen camera bij me had, zoals King Crimson in de Stadschouwburg. Volgens Google was het in het Concertgebouw op 23 november 1973. Dit had een indrukwekkende opening met heen-en-weer zweven, waarbij het geluid van de Moog minutenlang doorging. De band opende met een donderend geweld met `21st Century Schizoid Man'. Ik dacht dat de zaal ontplofte.

Wel heb ik plaatjes van het concert van Billy Cobham in Amsterdam. Tja, het was wel erg donker daar. Dat was in de tijd dat Cobham en Chick Corea erg populair waren met hun Return to Forever. Wat een energie. 

Een hele serie van groepen die allemaal tot Miles Davis terug te herleiden zijn kwamen bovendrijven. Weather Report, Stanley Clark. Jon McLaughlin die op den duur wel heel erg begon te zweven met z’n Mahavishnu Orchestra.

Rock'n'Roll Animal van Lou Reed was een heel ander optreden. Hij had kort wit haar, was volledig in het leer en volgens velen zwaar onder invloed. Mij overtuigde hij wel als rock'n'roll animal.

\begin{figure}
\centering
\includegraphics[width=\textwidth]{img/ch31/reed}
\caption*{\footnotesize Lou Reed.}
\end{figure}

Een vriendin van Olga was getrouwd met Dick Vennik, een bekend saxofonist die op allerlei podia speelde, onder andere met Rein de Graaff en in het Metropole Orkest. Ik ben een aantal keren met hem naar z’n eigen optreden geweest. Zo ging ik mee naar een optreden bij Max Teeuwisse in Den Oever, of naar Ben Webster in Paradiso. Hij liet me toen het koffertje van een niet nader te noemen saxofonist zien, waarbij de ene kant gevuld was met rieten enzovoorts en de andere kant met allerlei stimulantia. Dick vertelde dat het weinig uitmaakte, want als je hem een saxofoon in z’n mond duwde hij gewoon speelde.

Toen ik al lang gestopt was, maar Jan Visser nog volop in de race was, ging ik wel met hem mee naar de Dizzy Man's Band. Ik kende wel wat mensen in die band en het was altijd feest op het toneel. Nog een keer ben ik met Jan mee naar een opname geweest voor zijn soloproject Fisher and Friends. Hij had toen net een MG gekocht. Geen beste aankoop, wel een mooie en een schip van bijleg vanwege de hoge onderhoudskosten.

We reden in de winter in z’n MG---linnen dak, kacheltje deed het niet---naar de studio. Tijdens de opname moest er nog een stukje sitar in. Er werd gebeld. Een uurtje later stapte Robbie van Leeuwen binnen, speelde wat hem gevraagd werd en verdween weer. Van Leeuwen was componist van ‘Venus’ en gitarist in Shocking Blue. Wat het plaatje gedaan heeft weet ik niet.

\begin{figure}
\centering
\includegraphics[width=\textwidth]{img/ch31/ff}
% \caption*{\footnotesize }
\end{figure}

Met mijn zwager Howard ben ik nog een keer naar Thin Lizzy geweest, omdat die daar fan van was. Dat was vooral heel hard. 

In Den Haag zijn we naar Todd Rundgren geweest met z’n Utopia, een stuk dat als een soort opera bedoeld was. Een goeie rocker, goed geluid. Toen ik naar voren liep om wat foto's te maken begon er net een nieuw nummer en werd ik, omdat ik vlak voor een speakertoren stond, bijna terug de zaal ingeblazen.

Niet onbelangrijk waren de concerten van Jan Pasveer op zaterdagmiddag met Bach in de Bullekerk. Altijd met een kleine inleiding van de dirigent. Zo kon je met zo’n prachtige cantate weer gerust het weekend in.

\begin{figure}
\centering
\includegraphics[width=0.8\textwidth]{img/ch31/Spandoek-cantate-klein}
% \caption*{\footnotesize }
\end{figure}

Eén keer, met Pasen, ging het fout. Ik had een kaartje gekocht voor de Matthäus Passion en meldde me op de afgesproken tijd in de kerk. Ik was er helemaal klaar voor. De vorige keer kon ik het bij de opening al niet droog houden. Dat is altijd zo overweldigend dat ik bijna m’n keel moest inslikken om niet in tranen uit te barsten. De drie uur daarna zijn wel erg lang. 

De lengte stond me altijd tegen, maar dit jaar ging ik dan toch maar weer eens naar de Matthäus. Alles klaar, stilte in de zaal... bleek dit jaar de Johannes op het programma te staan. Dat had ik helemaal niet opgemerkt. Het kwam niet meer goed. In de pauze heb ik de kerk verlaten. Smitje, de man van Gertru, die aan de deur stond meldde nog dat het nog niet afgelopen was, maar ik zei dat ik me niet goed voelde. Dat was ook zo.

Ik heb ook nog eens in IJmuiden een Matthäus gehoord. Ik werkte daar en velen nodigden me uit om daar dit evenement bij te wonen. Vooruit, kan nooit kwaad om wat \emph{goodwill} te kweken. Je hebt een eigen winkeltje, zou Gerard Reve zeggen. Stond daar als dirigent de pianoleraar van Chrissy voor het geheel. 

Mooi om te zien dat hij z’n vleugels begon uit te slaan. Hij deed het goed. Na het concert kon je hem uitknijpen. 


Maar het blijft wel een hele rit. Ik kan niet zo lang zitten. Dan krijg ik last van m’n knieën en moet ik eigenlijk even lopen, maar dat doe je niet zo snel tijdens een uitvoering. Een Matthäuspassie is voor mij een hele opoffering, mijn lijdensverhaal.

In winter van 1970-1971 bezochten we de tentoonstelling van Salvador Dalí in Museum Boijmans Van Beuningen in Rotterdam. Een grote tentoonstelling met schilderijen en juwelen, waaronder het spraakmakende gouden kloppend hart. Wat mij imponeerde was de Christus aan het kruis. Wij woonden toen in de Botenmakersstraat met een enorm lange gang. Ik zag het al direct voor me, ook zo’n kruis in m’n huis. Het is er, door technisch gebrek mijnerzijds, niet van gekomen.

Een andere tentoonstelling die me nogal aangreep was `Bewogen Beweging' in het Stedelijk in Amsterdam---Kienholz, Tinguely, de Saint Phalle. Bij binnenkomst moest je door een smalle gang waar de vloer en wanden ondersteunt waren door vering. Je raakte je evenwicht totaal kwijt. Een goed begin dus, zeg dat.

Er waren enorme installaties met raderwerken van Tinguely. Dan moest je met een trappetje omhoog en zag je een vossenstaart op en neer bewegen. De beroemde bar van Edward Kienholz was er ook. En een radio waarbij de zenderknop door een machientje geregeld werd en je de hele schaal van zenders afliep. Klokken die allemaal op verschillende snelheden liepen---wat is tijd? 

\begin{figure}
\centering
\includegraphics[width=0.6\textwidth]{img/190dali.jpg}
\caption*{\footnotesize Dalí}
\end{figure}

\chapter{Het Werk}

\thispagestyle{empty}
\begin{flushright}
\begin{figure}
\includegraphics[width=0.8\textwidth, right]{img/epi/epi2}
\end{figure}
\end{flushright}

\section*{Werken, slapen, eten} % (fold)
\label{cha:werkenslapen}

Geld verdienen, werken om geld te verdienen. Er zijn er die werk aanbieden en er zijn er die het werk aannemen. De meeste mensen horen tot de laatste groep, ook mijn vader. Als kind zie je niets anders dan dat de vaders ’s morgens naar hun werk gaan en ’s avonds weer thuis komen van het werk. Alle vaders gingen naar hun werk. En als je broer oud genoeg was ging die ook naar z’n werk.

Mijn vader werkte dus bij scheepswerf Kraaijer (later De Beer). Toen ik voor het eerst met m’n vader mee mocht keek ik m’n ogen uit. Alles is zo groot en zo machtig interessant.

Oudere mannen die je meenemen naar de werkplaats en je van alles laten zien. En vooral nergens aankomen. 

\begin{figure}
\centering
\includegraphics[width=\textwidth]{img/196debeer}
\caption*{\footnotesize Scheepswerf De Beer}
\end{figure}

M’n vader was de baas in het magazijn. Iedereen die iets nodig had klopte aan en vroeg of die kleine van hem was. Zo leerde je veel mensen kennen. 

% duplicate with earlier chapter =>

% Ik heb me daar nog een keer flink gesneden. Van een groot zaagblad leerde ik een mes maken. Eerst de zaagtanden wegslijpen en daarna de vorm erin. Toen dat gedaan was een handvat maken door een flinke maat tape er enkele malen omheen te wikkelen. Ik vond het ook nodig dat het blinkte, dus ging ik het met wat olie poetsen. 

% Daarbij was ik vergeten hoe scherp het mes was geworden en ik sneed dwars door de poetsdoeken in m’n hand. Mijn mes werd gelijk ontdekt en voorlopig in beslag genomen. Het litteken heb ik nog.

Toen Harry ging werken ben ik ook nog wel eens met hem mee geweest naar verffabriek A.W. Sabel. 

\begin{figure}
\centering
\includegraphics[width=\textwidth]{img/ch32/sabel}
\caption*{\footnotesize Verffabriek A.W. Sabel}
\end{figure}

Harry werkte op het laboratorium, maar interessanter was de fabriek zelf, waar grote drijfriemen door het gebouw liepen en het hele zaakje van energie voorzagen. In feite zoals het in oude molens ging.

De oude heer van de Nieuwendijk, vader van Siem van de Nieuwendijk van de sportschool, werkte daar ook. Alpinopet op en een pruim in z’n mond. Ik heb ook eens zo’n pruim als welkomsgeschenk in m’n hand gedrukt gekregen. Bij Sabel ben ik een nagel van m’n rechter ringvinger kwijtgeraakt, die ergens tussen klem kwam te zitten.

Mijn moeder werkte in de avonduren (van 18.00 tot 19.30 uur) bij Het Hart en de Zwaan, later Scado N.V. Ze maakte het kantoor schoon met een vriendin. Dat deed ze al lang en zo kon ze voor mij een vakantiebaantje bemachtigen op het kantoor. 

Op dat kantoor zat een oudere heer en ik mocht (tien jaar oud, denk ik) wat papier heen en weer schuiven. Ik vond het zo prachtig dat ik de hele dag liep te fluiten, tot de oude man in wanhoop uitriep: ``Hou nou toch ’s op met dat gefluit!''

\begin{figure}
\centering
\includegraphics[width=\textwidth]{img/197scado}
\caption*{\footnotesize Het Hart en de Zwaan, later Scado NV}
\end{figure}

Dit was mijn inleiding tot het werk; kennis maken met fabrieken enzovoorts. Mijn echte carrière in de wereld van werk begon wat later.

Nadat m’n vader overleden was ging het nog minder op school. Toen ik voor de tweede keer dreigde te blijven zitten kreeg ik een gesprek bij de direkteur, meneer de Bruin, en het leek hem beter dat ik van school ging. Zo geschiedde. Ik nam me voor veel geld voor m’n moeder te gaan verdienen, \emph{multo dramatico}.

Hoe ik rond 1959 bij Tot \& Beers terechtkwam weet ik niet meer, het zal wel een advertentie zijn geweest. Ik kreeg er een opleiding bij voor electromonteur.

Tot \& Beers zat in een (toen) nieuw gebouw aan de Provincialeweg, waar op de bovenste verdieping een grote assemblageafdeling was. 

Hier werden grote schakelkasten in elkaar gezet. Zonder enige ervaring werden de jongsten aan eenvoudige klusjes gezet, zoals kabeltjes van een transformator in kaarsrechte lijnen naar een ander onderdeel leiden. Als er meer kabeltjes kwamen werden ze keurig samen gebonden en soms kwam je met een dikke tros bij het eindpunt. Zonder te weten wat het allemaal in technische zin betekende heb ik die dingen in elkaar gezet. 

Ik ontmoette daar een al wat oudere jongen, Han Boering. Hij was al zeer uitgekookt. Als hij even weg wilde dan waren er bijvoorbeeld geen zaagjes meer, of een bepaald boortje ontbrak en dan kon hij dat halen bij de Familie Perk of Huijsman. Liever Perk, want daar stond de shagdoos op de toonbank. 

Zo werd ik in het complot gezogen. Als het laatste zaagje of boortje uitgereikt werd moest ik dit breken of onklaar maken, zodat er een nieuwe voorraad gehaald kon worden. 

Op naar de winkel van Perk, achter het postkantoor. Daar stond---zo ik veertig jaar later pas te weten kwam---een jonge aankomende verkoper, Hans Hollander. Met Hans heb ik later nog jaren ’s avonds in de Vrije School geklust. 

\begin{figure}
\centering
\includegraphics[width=\textwidth]{img/199perk}
\caption*{\footnotesize Winkel van Perk (achter het postkantoor)}
\end{figure}

Hans was een gezellige kletser en goeie verkoper. Wij een sigaretje draaien, wat zaagjes inslaan en weer terug. Het einde kwam al snel toen ik het idee kreeg dat ik ook wel boortjes of een schaar kon slijpen. 

Dat het slijpen van boortjes en scharen buiten de mij toegestane werkzaamheden lag, merkte ik pas door de woede van de chef. Die zette me voor straf aan het sorteren van moertjes en boutjes, allemaal metrisch gelukkig. Hij toverde een volle emmer met dit spul voor m’n neus en ik kon aan de gang, M4 bij M4, enzovoorts. 

M’n gevoel zei me naar een andere baan uit te kijken, want dit zou nooit meer goed komen. Gelukkig hoefde je in die tijd je vinger maar in de lucht te steken of je had al ander werk. Zo kwam ik bij de Familie Fris op de Gedempte Gracht terecht.

De Firma Fris was een grote speciaalzaak in huishoudelijke artikelen met een afdeling voor reparatie achter de winkel. In die laatste kon ik beginnen. 

\begin{figure}
\centering
\includegraphics[width=\textwidth]{img/200fris}
\caption*{\footnotesize Winkel van Fris op de Gedempte Gracht}
\end{figure}

Ik had bij Tot \& Beers gewerkt, dus dachten ze dat ik wel iets van electriciteit af wist. Om te beginnen mocht ik het klein huishoudelijk goed repareren, zoals scheerapparaten, koffiemolens en dergelijke. De baas deed de radio’s en de TV’s. De man lag dan soms met dat hele lijf in zo’n grote TV kast. 

En je moest uitkijken voor het hoogspanningskastje. Ik heb er één keer een opdonder van gehad. Een echte electroshock waar je flink \emph{groggy} van raakte.

Naast de reparaties moest ik ook meehelpen met het wegbrengen van aangekochte of gerepareerde spullen (wasmachines, koelkasten en dergelijke). Soms moesten we wel naar drie hoog, zonder lift. Ik installeerde ook TV-antennes. Lekker de daken op, zo’n ding plaatsen en dan de TV afstellen. 

De jongen die de auto reed deed dat al heel lang en was zeer geraffineerd. Hij had speciale trucs om koffie te krijgen. Ik denk dat hij ook nog wel eens iets had met eenzame huisvrouwen. Sommigen gaven wel aanleiding, denk ik nu, want we werden soms ontvangen als ze nog niet aangekleed waren, of alsof ze zich net hadden uitgekleed. Ik had overigens ook wel fantasieën, maar het is nooit tot aktie gekomen. 

\begin{figure}
\centering
\includegraphics[width=\textwidth]{img/201husslage}
\caption*{\footnotesize Het bedrijf van firma Husslage was gevestigd in het witte gebouw rechts aan het eind van de straat.}
\end{figure}

Waar wel een eind aan kwam was m’n dienstverband. Toen ik op een ongelukkige ochtend door de winkel liep om een TV naar de auto te brengen, stootte ik bij de winkeldeur tegen iets in de etalage aan, waardoor achter me de halve uitstalling in elkaar donderde. Ik mocht nooit meer via de winkel lopen en kon maar beter uitkijken naar een andere baas.

Het bedrijf van firma Husslage, gevestigd op de Dam, was eigenlijk een electronicazaak, maar had zich gespecialiseerd in industriële communicatie. Interne geluidssystemen en dergelijke. 

Aan de achterkant van het gecombineerde woonhuis en winkel was een bedrijfje ontstaan, met de ingang aan de Rozengracht. In die tijd was er in die straat ook nog wekelijks de markt, dus gezelligheid troef. 

Omdat ik al wist hoe je draadjes aan iets vast moest maken, maar verder niets, begon ik natuurlijk weer onder aan de ladder. Wel een gezellige ploeg collega’s. Er was een jongen uit Koog aan de Zaan die gek was van oude radio’s en het opvoeren van brommers, een jongen uit Tuindorp Oostzaan, een man met suikerziekte, een chef die gek was van zeilen en altijd gesuikerde sigaren van Ritmeester rookte en natuurlijk de jonge en oude bazen Husslage. 

Het werk bestond voornamelijk uit het in elkaar zetten van communicatiesystemen. De schakelingen liepen toen nog mechanisch, zoals bij oude telefooncentrales. 

\begin{figure}
\centering
\includegraphics[width=\textwidth]{img/203relaistelefooncentrale}
\caption*{\footnotesize Een zogenaamde `zoeker'.}
\end{figure}

Als je een nummer draaide hoorde je de relais van één tot tien lopen om het nummer te vinden---\emph{tikke-tikke-tikke-tikke-tik}, zo’n geluid was dat. 

Ik ben ooit eens in een schakelcentrum van de PTT\footnote{Post-Telegraaf en Telefoon} geweest, waar duizenden van die relais continu aan het schakelen waren om nummers met elkaar te verbinden. Een fantastisch lawaai! 

Om het allemaal te laten werken, werden schakelkasten in elkaar gezet. Met een tekening erbij moest je dan aan het werk. Alle verbindingen met kabeltjes verbonden. Netjes volgens de regels alles recht naast, langs en onder elkaar gebogen en vastgezet. Als je alles erin had gezet moest de kast gekeurd worden. Je zette hem neer bij de chef die keek en zei dan: ``Zeker zelf gemaakt?''

Later werd dat alles electronisch opgelost met transistoren. Zo af en toe moesten we nog op locatie de zaak in elkaar zetten. Ik herinner me nog goed dat we in de Wibautstraat in Amsterdam tussen het Parool en de Volkskrant---elk aan een kant van de straat---een systeem leverden waardoor matrijzen pneumatisch naar de drukkerij vervoerd konden worden. Matrijzen waren de in aluminium gegoten versie van de krant, rond van vorm en klaar om op de drukpers te worden geplaatst. 

Dat was de eerste keer dat ik met het dagbladenwerk kennis maakte. Ook ben ik nog bij Schiphol geweest. Kastjes ophangen, kabels trekken richting de schakelkast en dan weer naar de andere kastjes. Het is allemaal zenden en ontvangen. 

Ook kan ik me werk bij de Hilko aan de Oostzijde herinneren en bij de Honig in de Koog. Allemaal interne communicatie. 

Je drukt op een knopje, ergens anders klinkt een geluid, de ander drukt de antwoordknop in en je kunt praten.	

% \begin{figure}
% \centering
% \includegraphics[width=\textwidth]{img/204hilko}
% \caption*{\footnotesize Hilko}
% \end{figure}

De brommergek op ons werk was Sjako. Tussen de middag zat Sjako z'n brommers op te voeren en kregen we allemaal het virus te pakken. Hij had veel verstand van motoren. 

\begin{figure}
\centering
\includegraphics[width=\textwidth]{img/205DEMM}
\caption*{\footnotesize DEMM}
\end{figure}

Geklooi met cilinders, carburator en in- en uitlaat. Hij had een Demm en nog een andere brommer van een Italiaans merk. Ik had een Puch en de jongen uit Amsterdam Noord een Kreidler. 

Als het markt was haalde een collega wel eens een trucje uit.  Dan begon hij met twee spijlen in z'n handen tegen het open raam naar het marktpubliek te schreeuwen: ``Help, help, ik wil eruit, ze houden me hier vast!''

\section*{Apotheek Hulp Artsen} % (fold)
\label{cha:apotheek}

Ik kwam bij op de administratie bij de Apotheek Hulp Artsen te werken. Daar was ene Ton de chef. Mijn andere collega's waren een jongen uit Westzaan en Margreet die later een boerderij in Nauerna opende waar ze speelgoed verkocht. 

\begin{figure}
\centering
\includegraphics[width=\textwidth]{img/207AHA}
\caption*{\footnotesize Het witte gebouw is Apotheek Hulp Artsen}
\end{figure}

Bij de Apotheek Hulp Artsen was er een kantoor (voor de witte boorden), een fabriek (met overalls) en een technische dienst. Het was een pillenfabriek, waar grote machines pillen draaiden. De technische dienst hield dat allemaal aan de gang. Daarnaast was er ook een groothandel in medicijnen. 

Dit was een periode waarin de `computer' in opkomst kwam. Op het kantoor werd nog gewerkt met een handboekhouding en tegelijkertijd met computerponskaarten voor de administratie. 

\begin{figure}
\centering
\includegraphics[width=\textwidth]{img/208ponskaart}
\caption*{\footnotesize Ponskaart}
\end{figure}

Toen ik daar werkte werd de eerste sorteermachine geplaatst, een IBM. Een enorm apparaat dat een halve kamer besloeg en in een razend tempo de ponskaarten er doorheen joeg, sorteerde en telde. Je kon te weten komen over je bedrijf wat je maar wilde  door er ponskaarten met gegevens in te stoppen. Ton, de chef, heeft heel wat momenten van paniek en wanhoop meegemaakt. Dan vloog zo’n hele stapel kaarten verkeerde kanten op, of bleven ze haken. Teams van IBM kwamen over de vloer, iedereen blij dat er nog een schaduwboekhouding was. Ik was vooral stil verliefd op alle mooie, al wat oudere dames. Die waren wel aantrekkelijk, maar ook meestal al getrouwd. 

Co Smit, drummer van beatgroep de Rolling Four waar ik ook een tijdje bij zat, was chauffeur bij de Apotheek Hulp Artsen. We zijn nog eens met de bedrijfswagen een middagje naar Wijk aan Zee geweest. Onderweg naar het strand ben ik vanaf de rand van de bak m’n portemonnee kwijtgeraakt.

Ik kan me nog een moment herinneren, zo rond 1962, dat Co mij op het werk `even wilde spreken'. Een meisje uit Assendelft had laten weten dat ze zwanger was van mij. Ik had wel eens met haar gevreeën na een optreden, maar omdat iedereen dan op je stond te wachten kwam je meestal niet veel verder dan zoenen enzo. Daar raak je niet zwanger van. 

Ik heb nooit enige \emph{feeling} gehad met boekhouden, maar wel met techniek. Zo liep ik weleens bij de technische dienst binnen om eens te kijken wat ze uitvoerden. Het was HTS-niveau\footnote{? Hogere Technische School}, vanwege het machinepark. 

Toen er bij de technische dienst een vacature kwam, heb ik contact met de chef daar opgenomen en gesolliciteerd. Toen bleek dat dat niet mogelijk was. Je kon niet van kantoor naar de fabriek gaan, dat voelde voor hen als een degradatie. Exit dus.

\section*{Pieter Schoen} % (fold)
\label{cha:schoen}

Henny van Wijngaarden werkte bij Pieter Schoen en via hem ben ik daar ook terecht gekomen. Mijn ervaring op de boekhouding van de Apotheek verdiende zich hier terug.

Het was werk in de voorraadadministratie---geen idee wat ik er heb gedaan. Het was een enorm kantoor met tientallen mensen die papier heen en weer schoven. Ik had het geluk dat ik af en toe naar de fabriek mocht om cijfers op te halen en naar een apart magazijn dat verderop in de Oostzijde lag. 

\begin{figure}
\centering
\includegraphics[width=\textwidth]{img/212pieterschoen}
\caption*{\footnotesize Pieter Schoen aan de Oostzijde}
\end{figure}

Ondertussen speelden we met de band inmiddels de sterren van de hemel. M’n lange haar trok de belangstelling van meneer Winter, een chef van iets. Als ik in het toilet stond kwam hij vaak even langs en prees m’n prachtige haar. Homo was iets waar je wel van had gehoord, maar zo iemand in het echt ontmoeten is wat anders. Een beetje dom lachen en dan gauw weg. 

Meneer Winter was niet onaardig. Hij was de zoon van schoolmeester Winter die ik op de Leeghwaterschool had gehad (degene die je aan je oor uit de bank trok als 'ie kwaad was). 

Toen het succes met de band niet meer tegen te houden was zijn Henny en ik vertrokken bij Pieter Schoen om beroemd te worden. Paul Acket, de grote impresario, wilde ons wel hebben.

Maar het noodlot sloeg al snel toe. We werden belazerd en zaten aan de grond, zonder inkomen. Weer werk zoeken dus.

\section*{Simon de Wit} % (fold)
\label{cha:wit}

Simon de Wit was een grootgrutter zoals ook Albert Heijn er één was. Ik kwam terecht in de kaaspakkerij van Simon, aan de Simon de Witstraat in de Westzijde. De bedoeling was dat de kazen, in allerlei vormen, met handkracht in verpakbare eenheden werden gesneden. Het waren grote messen die op een plankier vastzaten; kaas eronder en het mes er doorheen. Daarna stond er iemand die de er folie omheen deed en dat weer door een sealmachine voerde. Zo kwamen de kazen in de winkel. Na korte tijd had ik genoeg van het werk en de kaas.

\section*{Schoonmaakbedrijf van Heusden} % (fold)
\label{cha:schoonmaakbedrijf}

Er was meer te verdienen in de schoonmaak. Zo maakte ik kennis met van Heusden Schoonmaakbedrijf. Het bedrijf was nog klein. Twee broers en de kinderen, zoals zoon Rob. 

Met Rob begon ik te klussen. Winkels schoonmaken, na sluitingstijd natuurlijk. Maar ook nieuwe huizen voor oplevering. Een speciale klus vond ik die bij Albert Heijn waar we twee keer per week de cacaoafdeling in de fabriek schoonmaakten. In die afdeling werden de grote brokken cacao gekraakt, wat resulteerde in een cacaolaag op de vloer door alles wat er naast de machine viel. Het punt was om die laag los te bikken en af te voeren. Daarna moest je het met een spuit met gloeiend heet water schoonspuiten, nog wat schrobben hier en daar. Als laatste de dweilen eroverheen en klaar was Kees. Het was er altijd bloedheet vanwege de ovens. De man die in die snikhete fabriek die machines bediende was broodmager, werkte in z’n hemd en droop van het zweet. Toen ik vele jaren later een mevrouw interviewde stond er op een tafeltje een foto van de man van de chocolade, haar man. Op de foto ook zo broodmager, hij was inmiddels overleden.

\begin{figure}
\centering
\includegraphics[width=\textwidth]{img/216vanHeusdenschoonmaak}
\caption*{\footnotesize Albert Heijn in de Oostzijde}
\end{figure}

Ik nam afscheid van de van Heusdens toen het begon te lopen met de muziek. We wonnen een concours, hadden een engagement in Rotterdam kregen en een platencontract. We werden beroeps. De Heusdens boden me aan om bij hen te blijven, want het klikte wel, maar er zat meer brood in de muziek. Dertig jaar later waren ze marktleider in Noord-Holland en was Rob de directeur.

\section*{Losse arbeid} % (fold)
\label{cha:lossearbeid}

Na de muziek weer werk zoeken. Ik hoorde dat de eigenaar van café-danszaal \emph{De Lage Horn}, Henny Cornet wel voor werk kon zorgen. Ons eerste optreden na Duitsland was bij hem geweest. Henny was daarnaast onder-aannemer van sloopwerk, havenwerk en andere dingen waar alleen over werd gefluisterd. 

\begin{figure}
\centering
\includegraphics[width=\textwidth]{img/218LageHorn2}
\caption*{\footnotesize De Lage Horn: het witte gebouw rechts}
\end{figure}

Maar eerst nog even dit. Ergens in 1967 had de band twee optreden in de Zaanstreek in hetzelfde weekeinde. Zaterdag in de Volksbond en de volgende avond in de Lindenboomschool in de Koog. Op de zaterdag was Olga aanwezig en zag zij mij wel maar ik haar niet. De dag daarop zag ik haar wel, en hoe. Ik kon m'n ogen niet van haar afhouden. Ze was een opvallende verschijning met haar blonde haar. Het was aan en is niet meer uit gegaan. Ze heeft alles vanaf dat moment meegemaakt.

\begin{figure}
\centering
\includegraphics[width=\textwidth]{img/ch38/olga}
\caption*{\footnotesize Olga}
\end{figure}

Henny Cornet was iemand die ’s morgens tegen z’n vrouw riep dat-ie even brood ging halen en een week later terug kwam. Hij had mensen genoeg om af en toe een klus te doen, zoals de sloop van militair materieel in België. Ik was daar zelf niet bij, maar Joop Spronk, de man van Annie Dik wel, zoals ik later merkte. 

Diezelfde ploeg verscheen ook op andere bouwwerken, overal waar losse arbeid nodig was. Henny en Joop kwam ik later weer tegen bij de bouw van de Coentunnelweg bij Koog aan de Zaan. 

Het belangrijkste voor mij was dat Henny wel wat wist voor een oud muzikant. Hij introduceerde me bij Andries van der Werf, die in de Vinkenstraat woonde, tegenover het politiebureau. Andries had een werkplaats in Assendelft achter garage Metz, bij de Zaandammerweg. Hij werkte ook op lokatie, zoals op de werf van van Hattum \& Blankenvoort in Spaarndam. In de werkplaats was ene Jan de baas en die zei me dat ik moest leren lassen. 

Jan bleek zelf de instructeur op de lascursus in de Ambachtsschool. Jan had gelijk met z’n voorstel om te leren lassen, het beviel me wel.

Ik kwam in een groepje terecht rondom Joop, Willem (die op de Zuiddijk woonde), Leen (die woonde in het Albert Hahnplantsoen, Kankeroor, die had die bijnaam vanwege een slecht afgelopen cafégevecht, Toon Musse, de kunstenaar Theo Blankenstein uit Wormerveer (die was  getrouwd met Ellie Nijzing, die ik nog kende omdat ze in de Jan Bouwmeesterstraat had gewoond) en Jan Bakker van de Haven. Jan was Nederlands Kampioen Judo, een zwaargewicht en goeie ziel. Een mooi stel dus.

Als lasser heb ik onder meer gewerkt aan de Coentunnelweg en aan de brug in Koog aan de Zaan. 

\begin{figure}
\centering
\includegraphics[width=\textwidth]{img/220coentunnelweg}
\caption*{\footnotesize Hier zijn de spanten te zien die ik verlengde of verkortte}
\end{figure}

Dat was mooi werk. Ik had een vaste plek op de brug en de stukken werd op maat voor m’n neus gebracht door de kraan. Dan laste ik ze aan elkaar. `Stapelen' heet dat. Op een dag kwam er een groepje kornuiten langs die ook voor Andries werkten. Ze riepen: ``We gaan koffiedrinken!'' Allemaal in de auto’s en op naar Purmerend. Joop was er, net als Henny Cornet. Waarom naar Purmerend? Ik denk om Andries te pesten, want die kwam later langs en vond niemand. Daar konden die gasten in gedachten al om lachen.

En maar pijpen repareren, schepen slopen, van alles construeren, van dinbalken tot damwanden. Ik zat ook veel op de kraan, een portaalkraan. 

\begin{figure}
\centering
\includegraphics[width=\textwidth]{img/222kraan}
\caption*{\footnotesize Mijn kraan met `Olga' erop geschilderd. Met op de achtergrond scheepswerf Stapel, Spaarndam.}
\end{figure}

Voor de Hoogovens moesten we spoorrails maken. Daar kreeg je vreselijke rugpijn van, want de makkelijkste manier om dat snel te doen was om wijdbeens over de rails te staan, de blokjes te lassen en de rails onder je door te laten trekken. Na verloop van tijd kon je bijna niet meer overeind komen, vooral niet als je zes kilometer moest maken. Ook deden we bij de gasleidingen door het land nog het constructiewerk. Niet de leidingen zelf, maar alle constructie er omheen. Overal een handje bijdragen.

Ik heb weken bij Vlaardingen gewerkt, als enige van de groep Andries tussen allerlei mensen uit Drente. Ik verbleef in een pension en ging vrijdag weer naar huis. In de trein naar Amsterdam viel ik meestal in slaap van vermoeidheid. Eén keer riep een grappenmaker bij aankomst ``Maastricht!''. 

We zijn ook wel eens tijdens het persen van een pijp in de buurt van de Velsertunnel de pijp van 1 meter doorsnee ingereden met een lorry. De kleigrond kwam niet terug met het karretje, dus reden wij zo'n twintig of dertig meter naar binnen en schepten achter in die pijp de grond los. Het was daar zo warm dat je in je t-shirt werkte. Als je er dan na een twintig minuten uit getrokken werd stonden ze buiten om een olievat hun handen te warmen, want het was winter. 

Die grote gaspijp werd door de grond geperst en later vertelde de uitvoerder dat ze precies onder een leiding van de hoogspanningsmast door waren gegaan.Stel je voor dat die pijp die electriciteits leiding had geraakt. Wij maar die pijp in. We hadden voor hetzelfde geld de pijp uit kunnen gaan.

Toen ik m’n lascursussen had gedaan vroeg men of ik niet ook een cursus werktuigbouw zou willen doen. Weer naar de Ambachtschool. 

De stof was wel interessant. Welke krachten spelen er als een brugklep omhoog komt? Op een avond zat iedereen al in de klas en hoorden wij de docenten lekker keuvelen op de gang. Dat duurde maar en wij maar wachten. Toen ben ik de gang op gegaan en vroeg de heren of het nog lang zou nemen. Woedend waren ze, ik kon meteen vertrekken. Moest ik vervolgens bij de directeur op het matje komen. Ik mocht wel weer terugkomen als m’n haar geknipt zou worden. Huh? Ik ben direct vertrokken. Dan maar geen werktuigbouwkunde.

Ons ploegje was een bijzonder zooitje. Je had bijvoorbeeld Toon Muusse, die wel geschoold was om tekeningen te lezen en dikke Willem die dat in de praktijk geleerd had. Altijd gebakkelij. Toon was een alcoholist die op dinsdag wat begon bij te komen van het weekend en er donderdag alweer naar begon te verlangen. Toon ben ik later bij de sociale dienst weer tegen gekomen. Hij woonde toen in het huisje van mijn tante Lena in de Czaar Peterstraat. Verder bestond onze groep ook nog uit Leen, een manusje van alles die vroeger in de walvisvaart had gewerkt, op het schip de Willem Barentsz. Daarnaast was er Theo Blankestein, kunstenaar, nog iemand uit Beverwijk die elke avond wel iets mee naar huis nam en ik. Omdat Theo en ik een beetje met elkaar optrokken werden we ook apart bekeken.

\begin{figure}
\centering
\includegraphics[width=\textwidth]{img/223TheoenOlga.jpg}
\caption*{\footnotesize Theo en Olga}
\end{figure} 

We beheerden daar een enorm park met oud ijzer, dat wil zeggen, allerlei soorten ijzeren balken en pijpen die bij het opspuiten van land gebruikt werden. Balken en pijpen kwamen nadat ze gebruikt waren op de werf terug. Wij moesten alles wat stuk was repareren. Baggerwerk gaat soms met grof geweld gepaard en dat was ook wel aan de toestand van het materieel te zien. Als het gerepareerd werd dan kwam het in de opslag terecht, tot het weer afgeroepen werd voor een volgende klus. Daar waren de kranen voor nodig. 

We bivakkeerden in een houten huisje met een kacheltje. Daar stond de koffie op te pruttelen. We zaten daar met zeker zes man, erg krap dus. Op een keer stootte iemand de koffiepot om en de inhoud viel precies in m'n schoen. Als een speer uit het hok en `m uit proberen te krijgen. De schoen ging nog wel, maar m'n sokken zaten aan het vel van m'n voet gebrand. Dat werd ziektewet. Gelukkig was het net zomer en zat ik met m'n voet op een bankje lekker in het zonnetje.

We werden ook gevraagd om bouwkranen overeind te zetten en na voltooiing van de werken weer neer te halen. Meestal waren het kranen van het merk Liebher die je bij alle grote bouwwerken ziet staan. 

Rondom Schiphol hebben een paar keer een kraan moeten optzetten. Tussen drukke wegen stond dan moederziel alleen nog zo’n kraan. Van Hattum en Blankevoort zat in allerlei klusjes.

Het bedrijf verkocht ook een portaalkraan aan de familie Kan, een palenhandel in de Oostzijde. Met een sleepboot werd de dekschuit met alles erop via het Spaarne naar het Noordzeekanaal gevaren. Ten slotte de Zaan op, om bij Kan Palen aan de Oostzijde te komen. Ik was gelijk thuis, want Hof van Holland is daar om de hoek. 

\begin{figure}
\centering
\includegraphics[width=\textwidth]{img/225kan}
\caption*{\footnotesize Links met dwarsbalk de verkochte portaalkraan.}
\end{figure}

De volgende dag werd het ding met Joh. Schol Kranen\footnote{CHECK name} overeind gezet. Met de oude Johan Schol stond ik bovenop de kraan, die nog in de touwen hing. Ik was in de veronderstelling dat hij nog maar aan één kant vast zat toen ik hem hoorde roepen: ``Gooi maar los!''. Ik zag ons al naar beneden donderen. Maar ik had het mis, het kwam allemaan in orde. 

De veiligheid op het werk had geen hoge prioriteit. Er ging nog wel eens wat mis. Toen we in een schuit aan het lassen en branden waren viel een collega om. Vergiftigd door de gassen en dampen. Geen afvoer, daar werd gewoon niet aan gedacht. Gelukkig ging het al snel beter met de collega toen hij weer in de frisse lucht kwam.

Met de kranen kon ook van alles gebeuren. Er stonden op het terrein van de firma Van Hattem en Blankenvoort twee kranen. Ik werkte veel in beide kranen. In de grote---ongeveer vijftien meter hoog---ben ik een keer bijna het Spaarne ingereden. Het waaide, misschien stormde het wel. Er stonden vrachtwagens die een lading pijpen kwam ophalen. Dat lukt allemaal wel. We pikten de pijpen aan, reden naar de wagens, loste de lading en weer terug voor de volgende. Bij het terugrijden voelde ik al dat de wind grip kreeg op de kraan. 

Het was een portaalkraan, met twee rijdende poten. De besturing gaat via elektromotoren en kent alleen maar vooruit en achteruit. De wind was zo sterk dat ik het zaakje niet meer stil kon zetten. Ik begon naar beneden te schreeuwen uit dat kleine hokje boven in de kraan. 

Gelukkig begrepen ze beneden wat er aan de hand was en konden ze net op tijd grote klemmen, de remmen, om de rails gooien waarop het gevaarte reed. Ik was anders zeker met het hele gevaarte in het Spaarne gestort.

We tilden maar en tilden maar. Wat extra lading betekent dat je sneller klaar bent. Tot iemand eens keek wat de kettingen eigenlijk mochten hijsen. Toen bleek dat we meestal een paar ton teveel te hebben gehesen. Er was gelukkig niks gebeurd.

Ik kreeg een klusje om een aantal gietijzeren tandwielen te slopen. Het waren enorme industriële tandwielen, misschien wel drie meter doorsnee. Met een karretje met zuurstof- en gasflessen ging ik er naartoe en begon. Gietijzer kun je niet echt branden, in de zin van doormidden snijden. Het is meer dat je het ijzer moet laten smelten en er zo doorheen gaat. Dat levert nog veel gespat en vonken op. Je moest oppassen dat je kleding niet in de brand vloog of dat er zo’n druppel gloeiend heet ijzer in je schoen valt (wat me wel eens is overkomen). Gietijzer slopen is een kwestie van doorgaan. Opeens zie ik iemand zwaaien. Ik steek m’n hand op, maar hij blijft zwaaien en komt hard aanrennen. Ik doe m’n kap omhoog om te zien wat er aan de hand is. Stonden de slangen van de gasflessen in de fik. Als die grote flessen ontploffen, geeft dat een behoorlijke knal en overleef je het niet. 

\begin{figure}
\centering
\includegraphics[width=\textwidth]{img/224spuitpijpen}
\caption*{\footnotesize Spuitpijpen}
\end{figure}

\begin{figure}
\centering
\includegraphics[width=\textwidth]{img/227Tandwiel}
\caption*{\footnotesize Tandwiel van ongeveer de omvang als op de foto.}
\end{figure}

Wat mijn besluit om te vertrekken mede heeft bepaald was dat de stemming minder werd. Theo, Leen en ik liepen met de kraan op om iets te doen toen Willem tegen Leen riep: ``Grijp ‘m Leen!'' Daarop sprong Leen boven op me en probeerde mijn gezicht in een plas met water te drukken. Willem stond te krijsen boven in de kraan. Ik was in prima conditie en het lukt me om de aanval  over te nemen en Leen zelf met z’n gezicht in die plas te drukken. Willem deed direct het deurtje van de kraan dicht. Theo stond met grote ogen te kijken, die begreep ook niet wat dit nou weer betekende.

Het waren altijd van die zogenaamde plagerijtjes, vervelende opmerkingen, die we ons maar moesten laten wel gevallen. Jan, onze judokampioen---heel sterk en een veel te goeie jongen---was ook vaak het pispaaltje. Op een keer had hij er genoeg van. We stonden langs de waterkant van het Spaarne en Jan pakt Willem, tilt hem op en houd hem boven het water. Hij zei iets als: ``Niet meer hè, ophouwen nou.'' Jan zei nooit zoveel, maar hij was wel duidelijk.

Willem had er een handje van om van iemands tragiek, iemands ongeluk, gebruik te maken. Dat Toon technisch gezien meer wist moest hij bezuren door opmerkingen over z’n drankgebruik. Het werd er niet prettiger op. ’s Zomers was het te heet, ’s winters te koud, dus: de groeten.

\section*{Dagblad De Tijd} % (fold)
\label{cha:detijd}

Ik zag m’n kans via een advertentie voor opmakers bij dagblad \emph{De Tijd}.

Ik kwam daar als leerling-opmaker te werken. Mijn oudcollega’s vonden het nogal hoog gegrepen en hoonden me als `intellectueel'.

\begin{figure}
\centering
\includegraphics[width=\textwidth]{img/230DeTijd}
\caption*{\footnotesize Nieuwezijds Voorburgwal in Het Kasteel van Aemstel.}
\end{figure}

\emph{De Tijd} was gehuisd aan de Nieuwezijds Voorburgwal in Het Kasteel van Aemstel, een doolhof van trappen en kamertjes. In feite waren het twee gebouwen die naderhand, van binnen, samengevoegd waren. Het duurde even voor je daar je weg kon vinden. In de steeg naast het gebouw was een bakker die boven de toonbank broodjes gekookte worst verkocht en onder de toonbank porno. 

Ik begon bij de kleine advertenties en had m'n eigen kast met letters---zowel onderkast en bovenkast---symbolen, lijnen en wit.

Alles moest je met de hand samenstellen en dan kon er een matrijs van gemaakt worden. De chef was een norse man die er niet op uit was het zichzelf of anderen makkelijk te maken. Er werd verteld dat hij in Indië opzichter was geweest en dat-ie nu z'n zweepje miste.

Aan het steen, daar waar de kranten in elkaar gezet werden, werd je als leerling-opmaker voorlopig nog niet verwacht. De wereld van de couranten was een openbaring voor me: typografen, journalisten, fotografen, matrijsmakers. In die tijd waren er een stel beroemde (en aankomede) journalisten, zoals voor de sport Maarten de Vos, Arie Kuiper, Frans Nypels. Die werden later nog toonaangevender als journalisten. 

Er was de tv-commentator met strik, die vaak al vroeg beschonken was. De nog jonge Fons van Westerloo die op de stoep beweerde ooit als correspondent in Amerika terecht te zullen komen. Dat lukte ook en nu is hij de baas van SBS. Susanne Pièt, die later psycholoog werd, was er ook. Fotograaf Daniel Koning wist al hoe een prent eruit kwam te zien als hij het negatief in de DoKa (donkere kamer, voor het afdrukken van negatieven) afleverde. Hij is nog een keer bij een demonstratie neergeschoten door de politie, per ongeluk. 

Er werkte daar ook een jongen in de DoKa die uit Zaandam kwam,  van nabij de Ooievaarstraat. Ik ben nog wel eens bij hem langs geweest. Hij gaf me toegang tot de DoKa en de telexen, waar alle nieuwsfoto’s van de agentschappen achter elkaar binnen kwamen. De hele dag door een stroom van fotomateriaal. 

Ik heb het altijd een zeer bijzondere periode gevonden. Het bezig zijn met het nieuws, in mijn geval met het maken van advententies. Later haalde ik ook de stukken voor aan het steen in de opmaak. Weer later mocht ik correcties doen. Dan werden via de correctiekamer fouten doorgegeven die dan ook weer in het lood gecorrigeerd moesten worden. Dat deed ik. 

Omdat de pagina een in opmaak in spiegelbeeld en met de bovenkant tegen je buik aan zit, lees je van boven naar beneden en van links naar rechts in het lood de stukken. Zo corrigeer je ook. 

\begin{figure}
\centering
\includegraphics[width=\textwidth]{img/232opmaak.JPG}
\caption*{\footnotesize Opmaak aan het 'steen' - foto Martin Rep}
\end{figure}

En dan de gesprekken van de redacteuren die tegenover je staan en beslissen hoe de pagina er uit zal zien. Dat was na de `verschrikkingen' in de metaal een enorme verademing.

Ik voelde me in een intellectueel gebied binnen gekomen. Daarin hadden mijn oudcollega’s wel gelijk gekregen. De gesprekken in de correctiekamer gingen vaak over de inhoud. 

’s Morgens vroeg als ik van de trein kwam liep ik eerst nog even een stukje door de stad en dan tref je weleens leuke taferelen aan, zoals nachtvolk. Sommige van de foto’s die ik heb gemaakt die zijn ook door de krant gebruikt. Ik had altijd een camera bij me.

\begin{figure}
\centering
\includegraphics[width=\textwidth]{img/233damslapers.png}
\caption*{\footnotesize De Damslapers dier wakker gemaakt worden door de Reiniging}
\end{figure}

\begin{figure}
\centering
\includegraphics[width=\textwidth]{img/23dslaap.png}
\caption*{\footnotesize Iemand die op straat een slaapplaats vond en nog heerlijk lag te snurken terwijl het werkvolk langs hem liep}
\end{figure}

In de zetafdeling werd gewerkt met Linotype's. Dat zijn de zetmachines die de teksten in lood omzetten, waarbij de stukken in kolombreedte uit de machine vallen. De machine had een loodpot waarin lood gesmolten werd. Daarmee werd de tekst gedrukt. Het was er altijd heerlijk warm, de typografen zaten te zweten.

's Morgens gingen we altijd aan de overkant op de Nieuwe Zijds een kop koffie drinken. Als het tijd was om de kaarten in de prikklok te laten stempelen en de koffie was nog niet op, dan ging één van ons even de kaarten van de aanwezigen door de prikklok halen en weer snel terug. Als je kaart door de prikklok liet stempelen bewees je dat je aanwezig was.

De dag dat Tsjecho-Slowakije door de Russen werd 
binnen gevallen en wij allemaal op stoep stonden was er ontzetting. Vooral vlug een krant maken, want dat nieuws kon nog in de meeste edities mee. 

Een krant maken is altijd een spannende klus, omdat er elke dag iets gemaakt moet worden met een deadline. De edities---Amsterdam, Haarlem, Rotterdam--hadden allemaal hun vaste tijd om te drukken. Ik heb ooit de naam van Olga in de beursberichten weten te smokkelen. 

De correctiekamer was een tussenschakel. Alle stukken gingen eerst voor controle naar de correctiekamer (ook wel de `erectiekamer' genoemd omdat er een wat hitsige sfeer hing). Daar werden de taal- en andere fouten eruit gehaald. Toen ik een tijd op zaterdagmorgen werkte was de laatste editie van de zaterdagkrant al de deur uit en zaten journalisten en correctoren nog wat na te drinken en te kletsen. Gezellige boel. 

Met één van de correctoren, een man uit Rotterdam, kwam ik in gesprek over de Zaanstreek. Hij had daar enkele kennissen, onder andere op de Prins Hendrikkade. Ik nodigde hem uit eens bij ons langs te komen als hij weer eens in Zaandam op bezoek was. Dat deed hij en hij was vooral gecharmeerd van Olga. 

Zijn hobby, fotografie, was aan de pikante kant. Zo liet hij  zijn vrouw bijvoorbeeld in het Rijksmuseum een trap oplopen zodat hij haar, terwijl ze geen slip aan had, van onderen kon fotograferen. Hij zag daar wel een rol voor Olga in. Dat is niet door gegaan. 

Zijn vriend op de Prins Hendrikkade ontmoette ik jaren later weer toen ik gevraagd werd in een jazz-ensemble als gitarist deel te nemen. Die vriend speelde drums en klarinet, waardoor  de drums soms wegviel bij een klarinetsolo.

Ondertussen deden Olga en ik samen de avond-MAVO bij het Barleus Lyceum in Amsterdam, tegenover Paradiso. Dat werd later gevolgd door de avond-HAVO,

Olga werkte toen bij de PTT in de hoofdstad. We hadden allebei geen diploma van de middelbareschool. Olga was gezakt voor het MMS-eindexamen\footnote{Middelbare meisjesschool.} en ik had helemaal niks.

Die avondstudie bracht veel drukte met zich mee, maar desondanks was het een leuke tijd. Het was een sport om zonder te betalen met de tram te reizen. We maakten er kennis met veel verschillende mensen met wie het later best goed is gegaan, maar some ook heel slecht. Van werken bij de televisie tot heroïnejunk.

\begin{figure}
\centering
\includegraphics[width=\textwidth]{img/236barleus_0004}
\caption*{\footnotesize Olga krijgt haar diploma en bloemen omdat ze de hoogste score van alle examens had gehaald. Oei, kort rokje.}
\end{figure}

\begin{figure}
\centering
\includegraphics[width=\textwidth]{img/237aschool001}
\caption*{\footnotesize Olga van opzij, Babette Mossel die omhoog kijkt naar de docent Duits. Babette is een gevierd vertaalster geworden.}
\end{figure}

Vanwege mijn studie bij het Barleus (en andere dingen die ik me niet meer herinner) heb ik toen een baan gezocht waar ik wat meer rust had om huiswerk te maken. Die vond ik bij van der Molen in Wormerveer.

Dagblad \emph{De Tijd} verdween in 1974, na 130 jaar op de markt.

\section*{Van der Molen} % (fold)
\label{cha:vandermolen}

Technisch Bureau Van der Molen was gevestigd aan de Zaanweg te Wormerveer. De directeur was de heer van der Molen. Het bureau ontwikkelde fabrieken, bijvoorbeeld voor Coca Cola. Ook ontwikkelden ze wel installaties voor het maken en opslaan van producten, bijvoorbeeld het in flesjes doen van zoals Coca Cola. 

Het bedrijf bestond voornamelijk uit mensen achter tekentafels, met specialisaties zoals pneumatiek, electriciteit, raffineren.

Meneer Soll uit Westzaan ontwikkelde bij van der Molen  sigareninpakmachines. Sigaartje erin, bandje erom, in een doosje doen, doosje dicht, zegeltje erop. Hij maakte ook een machine om de noot in een gevulde koek te drukken. 

Bij Van der Molen heb ik Rob de Boer leren kennen, die daar via constructiebedrijf Kramer en Zwart was komen werken. Ik deed er het archief en de lichtdruk. 

In zo'n bedrijf wil men van alle tekeningen afdrukken hebben om het tekenproces te aanschouwen of om er opnieuw op te tekenen. Van de doorzichtige tekening werd dan een lichtdruk gemaakt. Soms klein, soms tot wel meters lang, indien nodig. 

\begin{figure}
\centering
\includegraphics[width=\textwidth]{img/240zaanweg}
\caption*{\footnotesize Links huis van de familie De Geer, Experts Electronica, De Zaanlander, van der Molen (wit), AMRO bank, Toby Interieur.}
\end{figure}

Wormerveer was wel gezellig om te werken, het was nog dorps. De administratieve benodigdheden haalde ik bij kantoorboekhandel Spaander in de Marktstraat. Ik werd goeie maatjes met de eigenaar. Koffie drinken en kletsen. 

\begin{figure}
\centering
\includegraphics[width=\textwidth]{img/242brommer}
\caption*{\footnotesize Terug, op de Puch, van het werk.}
\end{figure}

Ook bij de Albert Heijn verderop waren we goeie klanten. Bij Albert Heijn haalden we altijd de suiker voor proeven met suiker. Voor Ab van Vliet haalde ik op een karretje zakken suiker, die dan een vat gingen om te kijken hoe snel het in het water werd opgenomen. Nou ja, hij deed dat dan. Je had een ontwikkelgroep die boven zat en de uitwerkgroep beneden. 

Ik zat ook beneden, met weer naast mij het magazijn. Dat magazijn had een achteruitgang waardoor iedereen binnenkwam en weer wegging. Eén ging soms, nadat hij achter binnen was gekomen, de voordeur weer uit, liep door de steeg weer naar achteren, kwam weer binnen en vroeg: ``Is mijn tweelingbroer al binnen?"

\begin{figure}
\centering
\includegraphics[width=\textwidth]{img/ch40/molenvd}
\caption*{\footnotesize Op de foto Constanze en Winnie en de entree waar ik ook nog al eens kon invallen bij ziekte of zeer.}
\end{figure}

Omdat ik eruit zag als een `hippie' vond men het toch ongeschikt als ik daar teveel zat.

De baas zei het als volgt: ``Ik geef een vermogen uit aan een entree die indruk moet maken en dan zit jij erachter.'' Later hoorde ik van hem dat een klant die eerst van mij geschrokken was zeer tevreden was met hoe ik de zaak afhandelde.

Aan de voorkant naast mij zat meneer Borgerhof die vertaalde, de telexen beheerde en de correspondentie deed. Hij woonde in Zaandam aan het Prins Bernhardplein, hield van klassieke muziek en had thuis een schilderij uit de Haagse School wat ik hoopte ooit van hem te erven. Bij afwezigheid van de koffiejufrouw deed ik de koffie. Ik ben nog een keer met een blad met koffie op de trap uitgegleden, waardoor het volle blad omlaag kletterde.

Mijn werkplek was het magazijn waar de lichtdrukmachines stonden en als men zag dat ik aan m'n huiswerk zat zei men al snel: ``Blijf maar zitten, ik doe het zelf wel even.'' Na een ziekte bleek dat ik in die baan helemaal niet gemist werd. Ze konden het met gemak zonder mij af.

Als voorlopige werkoplossing mocht ik de boot van de baas opnieuw lakken (eerst schuren natuurlijk). Daarna mocht ik ook wekelijks z’n tuin bijhouden. Ik dronk met mevrouw van der Molen altijd koffie. Maar toen ze op een keertje bezoek had mocht ik m’n koffie in de keuken opdrinken. Dat vond ik zo laag dat ik de baas vertelde ermee op te willen houden. Gelukkig regelde hij het zo dat ik een uitkering kon krijgen. 

\begin{figure}
\centering
\includegraphics[width=\textwidth]{img/244kantoor.png}
\caption*{\footnotesize Tekenzaal met Rob, op de rug Borgerhoff en Johan Baars.}
\end{figure}

\section*{Uitkering(en)} % (fold)
\label{cha:uitkering_en}

Een uitkering. Of eigenlijk uitkering\emph{en}, in het meervoud, omdat er een paar onderbrekingen inzitten. Je moet namelijk blijven solliciteren en soms is er gewoon niet onderuit te komen om af en toe te werken. 

De eerste tijd moest ik m'n werklozen inschrijfkaart verlengen in het Van Raalte Centrum bij de bedrijfsvereniging. Na een tijdje ben ik gaan werken bij Van der Stadt. Dat was een winkel in huishoudelijke artikelen en sleutels in de Westzijde, nabij de Ambachtschool en tegenover bouwbedrijf Piet Schutte.

\begin{figure}
\centering
\includegraphics[width=\textwidth]{img/ch41/stadt}
\caption*{\footnotesize Op de foto zijn het de twee panden links. De ingang was uiterst links.}
\end{figure}

Ik heb daar vooral veel over sleutels geleerd. Soms werd ik er wel op uitgestuurd als ergens een sleutel vergeten was of er juist een slot op gezet moest worden. Ook maakte ik reservesleutels. Je had dan de keuze uit honderden blanke sleutels die als basis dienden voor de te maken sleutel. In de machine zet je de originele sleutel in een klem, de blanke sleutel in de andere klem en zo wordt het beeld in de blanke sleutel gefreesd. 

Als meneer van der Stadt zelf sleutels maakte hoorde je vaak een enorm gehoest. Hij rookte achter elkaar door en haalde z’n sigaret bijna nooit uit z’n mond. Overal lag as. Als hij eenmaal hoestte bleef hij aan de gang omdat hij telkens de rook weer inademde. Ik leerde daar ook een meisje kennen dat de huishoudelijke artikelen deed. Later leerde ze me hoe je geld kon verdienen door een lager bedrag aan te slaan dan het afgerekende bedrag en het verschil in je zak te steken. Later kom ik haar bij mijn volgende baan weer tegen. 

\begin{figure}
\centering
\includegraphics[width=\textwidth]{img/247school}
\caption*{\footnotesize CHECK Het secretariaat van het bestuur van de gezamenlijke technische scholen in de Zaanstreek met een secretaris (van het bestuur) en drie medewerkers voor de administratie}
\end{figure}

Ik kom te werken bij het secretariaat van het bestuur van de gezamenlijke technische scholen in de Zaanstreek. De secretaris, de heer Riemersma, leefde voor z'n werk. Hij was een kei; wist alles en regelde alles. Hij woonde nog thuis bij zijn moeder. Het hoofd van de boekhouding, meneer van Neerven, was net nieuw. Heel precies en hield van opera. Van thuis had hij een Revox recorder meegenomen waarop hij via enorme spoelen opera's afdraaide. Maar alleen als meneer Riemersma er niet was, want die hielt daar niet zo van.

\begin{figure}
\centering
\includegraphics[width=\textwidth]{img/ch41/IMG_0002}
\caption*{\footnotesize Meneer van Neerven}
\end{figure}

De tweede in rang was het meisje waarmee ik samen in de winkel bij Van der Stadt had gestaan. Ze had een Opel sport-auto eneen man die veel geld verdiende. Zij deed de administratie en nog andere dingen. Dat terwijl ze bij Van der Stadt de zaak bestal.

Het secretariaat was nog een ouderwetse boekhouding met allerlei boeken, telstroken, week en maandstaten, jaarstaten, allerlei overzichten en controle van de boeken. Om gek van te worden. 

Het was gevestigd in de Stationssstraat, in een bovenverdieping van een vroegere school, die nu onderdeel uitmaakte van de technische scholen. Het was een nogal hitsige onderlinge verstandhouding, want onze boekhouder was wel bij tijd en wijle zeer gecharmeerd van het `volkse' van onze vrouwelijke collega, die dat naar mijn mening wat cultiveerde. 

De voorzitter van ons bestuur woonde ook in de Stationsstraat, zodat er wel eens iets ter ondertekening naar hem gebracht moest worden. Hij was lid van de oude industriële elite van de Zaanstreek, een netwerk van ons kent ons. Daar viel onderling wel wat te regelen als het in het belang van de school was. 

Een statige oude man, sympathiek. De verhouding tussen de secretaris en de voorzitter leek op die van mentor en opvolgingskandidaat. Dat is een gevoel, net als dat van het `hitsige' van de boekhouder. Omdat er ingekrompen moest worden, of omdat ik niet voldeed, kwam ik weer terug in een uitkering.

Ten tijde van mijn laatste werkloosheid woonden we op de Hogendijk 98, in de helft van het pand.

\begin{figure}
\centering
\includegraphics[width=\textwidth]{img/249huis}
\caption*{\footnotesize Hogendijk 98. We hadden het 2e huis van het blokje onder één kap}
\end{figure}

In die tijd maakten we ook kennis met Joke en Hilde, zij woonden verderop op de Hogendijk en zaten met woningproblemen. Ze hebben een periode bij ons gewoond. Via hen met Rudi Kahl, de tekenaar en levenskunstenaar. Rudi kwam af en toe op de fiets op bezoek uit Friesland. Toen Annie plotseling met haar nieuwe vriend, Joop Spronk, langs kwam bleek dat iemand te zijn waar ik vroeger in de metaal mee had samen gewerkt. Het was een bonte tijd. Mensen woonden soms een tijdje bij ons en verdwenen dan weer. 

Olga werkte bij sociale zaken en ik was thuis, deed de afwas en zorgde voor het eten. 's Morgens naar het koffiehuis voor de krant, koffie en dan wat huiswerk. Toen er een brand bij houthandel William Pont, uitbrak stond ook net uit het raam te kijken. Vlug m'n camera gepakt en foto's gemaakt.Toen ik op een dag daar weer ens uit het raam stond te kijken en een stroom mensen voorbij zag gaan vroeg ik  me af waar die allemaal heen gingen. Zou er iets aan de hand zijn? Nee, opeens drong het tot me door dat ik stond te kijken naar mensen die naar hun werk gingen. 

\begin{figure}
\centering
\includegraphics[width=\textwidth]{img/brandpont_00021}
\caption*{\footnotesize De brand bij houthandel Pont, eigen foto}
\end{figure}

Dan wordt het echt tijd om weer wat te gaan doen. M'n avondschool was inmiddels achter de rug. Tijdens het afleveren van m'n werkbriefje bij sociale zaken kwam het gesprek op werk daar. Olga had me al verteld van de vacatures bij de sociale dienst. Het bleek dat ze hard mensen nodig hadden. Ik mocht op sollicitatie komen en werd aangenomen met het uitzicht op een HBO-opleiding aan de Sociale Academie onder werktijd. 

\section*{Sociale Zaken} % (fold)
\label{cha:socialezaken}

Zo kwam ik aan de andere kant van de balie bij Sociale Zaken van de gemeente Zaandam terecht. Gerrit Bakker was hoofd van de afdeling Bijstand. Er was nog een directeur die langzaam kinds begon te worden en met een trommeltje door de gangen liep. 

\begin{figure}
\centering
\includegraphics[width=\textwidth]{img/ch42/soos01}
\caption*{\footnotesize Op de foto Jan Wakker en Roel Jansma.}
\end{figure}

Er waren nog wat maatschappelijk werkers, onder andere juffrouw Van Wolveren (je had toen nog juffrouwen, mejuffrouwen en ook nog mevrouwen) en iemand die het hield met de sociaal rechercheur van de dienst. 

Verder hadden we op onze afdeling een andere man die in diensttijd tweedehands auto’s verkocht en Peter Postma die later directeur werd. Het was een overgangsperiode. In Zaandijk werd een nieuw stadhuis gebouwd en er waren plannen om de Zaangemeenten samen te voegen.

Ik heb de overgang van Zaandam naar Zaanstad en de problemen die dat gaf meegemaakt. Het waren jaren waarin de bomen bij de sociale diensten tot ver in de hemel groeiden en ongeveer alles kon. De toenmalige administratie werd later bijna geheel afgedankt, onder andere de heer Ulle. Waar ik kwam te werken gingen ook mensen met pensioen.

Het centrale kantoor in de Westzijde werd ook verlaten en ingeruild voor het gemeentehuis aan de Burcht. Omdat we nu Zaanstad waren gingen we over naar wijkkantoren in elke gemeente van Zaanstad. 

In het stadhuis kreeg ik een plekje aan het raam dat uitkeek op het parkeerterrein van de Burcht. Altijd handig om te zien in wat voor auto klanten komen aanrijden. 

We hadden inmiddels ook een inspecteur in dienst, een oud-politieman van wie je zo nu en dan informatie of tips kreeg. 

\begin{figure}
\centering
\includegraphics[width=\textwidth]{img/ch42/IMG_0001}
% \caption*{\footnotesize Op de foto mijn plekje in het hoekje naast de deur.}
\end{figure}

\begin{figure}
\centering
\includegraphics[width=\textwidth]{img/253-2}
\caption*{\footnotesize Op de foto mijn plekje in het hoekje naast de deur.}
\end{figure}

Tijdens een van m’n eerste werkdagen in 1975 werd ik bij Gerrit Bakker geroepen, die me zei dat er een jongen op de bank lag te slapen. Ik moest maar eens gaan vragen wat er aan de hand was. 

De jongen bleek uit Australië te komen, was ziek en had geen geld. Dat was het begin en de richting die mijn carrière bij de dienst zou nemen: drugsverslaafden. Gerrit benoemde me onmiddellijk tot deskundige en als je die titel eenmaal hebt doet iedereen alsof je dat ook bent. Dat het me niet zo moeilijk viel om interessant te doen kwam omdat ik zelf een gebruiker was en wel wat mensen in het hash- en wietwereldje kende. Ik wist meer termen dan alle anderen bij elkaar. 

Toen mijn status als deskundige eenmaal bekend werd kwamen ze ook allemaal langs. Via de hulpverlening kwam ik in contact met artsen, psychiaters en andere mensen in dat wereldje. Ik ben betrokken geweest bij de opzet van drughulpverlening in de Zaanstreek en samen met de GGD bij de komst van wat later de Breijderstichting zou worden. Ook had ik de hand in spreekuren in jongerencentra en buurthuizen. Ik heb er een paar zien afkicken en anderen verslaafd zien worden; sommigen dood zien gaan, velen grote kunsten uit zien halen. 

In zo’n omgeving gebeuren natuurlijk dingen die buiten het normale patroon van bijstandverlening vallen. Ik viel blijkbaar zo op dat een rechercheur van de plaatselijke  inlichtingendienst (P.I.D.) me polste of ik niet mee kon werken aan het opzetten van een val voor enkele drugsdealers (ben ik niet op ingegaan). Hij was mijn naam tegen gekomen in een rapport over een politie-inval in mijn huis. Dat zat zo. 

Op een avond kwamen we thuis en zaten er politieagenten binnen. De jongen die bij ons inwoonde dealde een beetje. De politie dacht een vangst te kunnen doen omdat ze gehoord hadden van een grote transactie die avond, op ons adres. Maar ze vonden niets. Toen ik binnenkwam zei een agent die ik kende van een ander project: ``Hé, jij hier?''. ``Ja,'' zei ik, ``ik woon hier, dit is mijn huis.'' Het kwam allemaal wel weer goed, maar ik zat er te ver en te veel in en was niet meer onafhankelijk. 

Veel klanten zaten in de scene als gebruiker of dealer. Ze kwamen op kantoor om even wat stuff langs te brengen. Ik kwam bij ze thuis, gebruikte het één en ander. Ze vertelden van allianties, van wie er in wat dealde. Dat kon nooit goed gaan. Dat en andere zaken hebben uiteindelijk tot m’n besluit geleid om bij de dienst weg te gaan. 

Een jongen probeerde af te kicken omdat hij vader zou worden. Het meisje heb ik nadat ze bevallen was aan heroïne verslaafd zien worden, uit een vreemd soort solidariteit met haar vriend. Ze is nu nog steeds verslaafd. 

\begin{figure}
\centering
\includegraphics[width=\textwidth]{img/256tafel}
% \caption*{\footnotesize Op de foto mijn plekje in het hoekje naast de deur.}
\end{figure}

In onze kennissenkring gebruikte iedereen. Geen zware middelen, maar af en toe een jointje bij de koffie. Ik had m’n vaste dealer en zag daar natuurlijk af en toe wel een klantje die ook bij mij klant was. Maar er waren ook jongens en meisjes die meer en zwaarder gebruikten, heroïneklanten. 

Zelf heb ik een aantal malen heroïne gebruikt en al snel ondervonden hoe de honger je in z’n greep krijgt. De flash die je krijgt is enorm.

Gebruik wordt pas zorgelijk als je, voordat je naar je werk gaat, alvast wat rookt. Zo’n dag komt niet meer goed. Fysiek ben je niet in staat het zo vol te houden. We namen mensen in huis die op straat stonden en voor ons weer voor allerlei problemen zorgden. Als ze uit Amsterdam kwamen en wat gescoord hadden klopten ze als eerste bij ons huis aan, omdat wij het dichts bij het station woonden. 

Koos en Toos (schuilnamen) waren, zoals iedereen die bij de soos kwam, een apart stel. Van Koos was bekend dat hij verslaafd was aan hard drugs en van Toos dat ze van Koos hield. Voor Koos regelde ik basisvoorzieningen zoals huur, ziekenfonds en dergelijke, omdat hij anders alles opmaakte. Op een dag kreeg ik van onze eigen administratie het bericht dat hij bij het hoofdkantoor van de Gemeentelijke Socialte Dienst ramen stond te lappen. Abel Zwart, collega ter plaatse, kende hem goed en belde. Z’n toelage vond hij te krap en ging toen maar een paar uurtjes werken, vertelde hij. Maar als je het met hem over werk had was-ie altijd te ziek om iets te doen. 

Later is die opgepakt wegens een overval; vuurwapen bezit en dergelijke. Volgens mij leeft hij niet meer. Hij had alles in zich om dood te gaan---was het niet door een overdosis dan zou het wel door z’n geweldadige omgeving gebeuren. Z’n toenmalige vriendin zie ik nog regelmatig. Zij is er op tijd uitgestapt.

Ik zie nu ook nog geregeld een jongen die toen zwaar verslaafd was, maar nu met z’n vrouw en twee dochters in het park wandelt. Die beweert dat ik hem op het goede pad heb geholpen omdat ik naar hem luisterde. ``Ik kon altijd alles bij je vertellen en je lulde nooit tegen m’n moeder.'' Hij zat echt in smerige zaakjes, zag er ook uit als een macho. Toen een macho tot en met, maar nu is hij vader en gelukkig met z’n dochters. 

En dan de moeder die haar dochter bergafwaarts zag gaan en tenslotte begreep dat ze haar moest laten gaan. Er was niets meer aan te doen, opgegeven toen ze achttien jaar was. Later kreeg het meisje een beroerte, maar omdat ze zo stoned was merkte ze het pas veel later toen ze haar benen niet meer kon bewegen. 

Voor haar moeder, die regelmatig even bij me langs kwam, was officieel geen tijd---ze was geen klant---maar als hulpverlener heb je ook daar natuurlijk een taak. Ook buiten de dienst kwam ik haar wel tegen en dan ontkwam je er niet aan om even met haar te praten over haar dochter. Een aardige vrouw, overigens.

Ik kom in Zaandam altijd weer een man tegen die naar me wuift, schreeuwt. Als we eens stil staan om even wat te praten begint hij altijd te vertellen dat als ik hem toen niet had geholpen met een nieuw huis, z’n hele gezin naar de klote was gegaan.

In het kader van al die activiteiten hield ik ook een spreekuur, 's avonds in het Drieluyck, zeg maar De Kade\footnote{Tegenwoordig De Flux.} van toen. Het leverde niet veel op, maar je staat weer eens in de krant met je werk. Dat viel wel goed bij de directie. 

\begin{figure}
\centering
\includegraphics[width=\textwidth]{img/ch42/krant}
\caption*{\footnotesize Het artikel in de krant over m'n sociale werk in het Drieluyck. Ik was wel ontzettend mager, zo te zien. In het midden Harmen Schaap waarmee ik, namens de Gemeente, in het bestuur van JoJo zat. JoJo was een opvanghuis voor jongeren met begeleiding. Eerst in een flat en later in een woonhuis. Daar zaten ook artsen, psychologen in de begeleidingsgroep. }
\end{figure}

Een ander memorabel persoon uit die tijd was Rein Rus. Op een dag toen ik ’s morgens fietsend over de Wilhelminabrug bij het stadhuis kwam aanrijden zat er een man op de trappen te huilen. Een jaar of vijftig, grijze kop met haar, ongeschoren en een hazenlip. Soms moeilijk te verstaan, zeker als hij huilde. Z’n moeder was dood, hij moest z’n huis uit en nu wist hij het allemaal niet meer. Eerst maar mee naar binnen genomen, kopje koffie en wat praten. 

Z’n moeder was eigenlijk al een hele tijd dood. Hij woonde alleen in een houten huisje op het Hanenpad en werkte bij de sociale werkvoorziening. Een zwak begaafde man die het niet meer alleen kon bolwerken. Heb ik afgesproken dat ik hem ’s middags zou bezoeken en z’n zaken eerst even op een rijtje zou zetten. Toen z'n werkgever en dergelijke gebeld. 

Het werd het begin van een lange relatie, waarbij het voornaamste doel werd om hem in z’n huisje te houden, de zorg om hem heen---onder andere werk, buren, gezinshulp---te organiseren. 

\begin{figure}
\centering
\includegraphics[width=\textwidth]{img/259ReinRus.JPG}
\caption*{\footnotesize Rein}
\end{figure}

In het huisje had hij nooit wat veranderd nadat z’n moeder was overleden. Dat kon ook niet, omdat hij het idee had dat z’n moeder toekeek. Hij praatte ook met haar net zoals met God of Christus. Rein zag soms ook spoken, of iets dat bijvoorbeeld bij hem binnen keek. Zijn angsten hadden natuurlijk allemaal te maken met z’n onvermogen om z’n bestaan zelf vorm te geven. Hij had wel leren schrijven en lezen, maar verder dan het eerste klasniveau was 'ie nooit gekomen. Brieven gooide hij weg, of stopte ze ongeopend in een doos. Hij was er trots op dat hij zo zuinig leefde. Dat deed hij onder andere door de hele dag het ’s morgens gevulde koffiefilter te blijven beschenken met water dat hij uit de regenton buiten haalde. Als je laat in de ochtend kwam had je alleen maar wat bruin water. 

Uiteindelijk is alles wel aardig op z’n pootjes terecht gekomen. Hij kon blijven werken. Toen hij dat niet lang vol hield en zat hij gewoon thuis, kreeg z’n geld. Ik was z’n vriend. Af en toe kwam hij langs als `ie het niet snapte of moest huilen. Dat deed hij pas als `ie aan m’n bureau zat. Had hij de hele weg van huis naar m’n kantoor eraan gedacht en aan het bureau kon hij het pas vertellen. En gelijk huilen.

Ik moest eens op bezoek bij twee broers, dik in de zeventig, die op het Hollandse Pad woonden. Het pad ligt naast de Verkadefabrieken aan de Westzijde. Ze woonden in zo’n typisch Zaans huisje. Bij binnenkomst in de kleine kamer mocht ik aan de tafel, met een dik kleed erop, gaan zitten. Kopje koffie gehad en nadat ik alle vragen voor de bijstand had ingevuld kon ik weer terug naar de dienst. 

\begin{figure}
\centering
\includegraphics[width=\textwidth]{img/260-261pad}
\caption*{\footnotesize Hun woning aan het Hollandse Pad.}
\end{figure}

Toen ik de papieren van tafel wilde oppakken bleken ze vast geplakt aan het tafelkleedje. Om niet alles onder het vettige goedje te laten komen heb ik de papieren dubbel gevouwen. Bij het inleveren kon ik ze maar moeilijk van elkaar krijgen.	

Zigeuners! Kampers! Bij het horen van deze woorden sloeg menig hart een keertje over. 

Dus mocht ik ook een keer naar het kamp---het oude kamp op een veldje achter Bruynzeel. Dat was rampzalig. Er stonden nog een paar wagens rondom een modderige poel, een vreselijke bende. Het was een schande zoals de gemeente dit had laten gaan. Het maakte wel duidelijk waarom er een nieuw kamp gebouwd werd. 

\begin{figure}
\centering
\includegraphics[width=\textwidth]{img/ch42/woonwagenkamp1977}
\caption*{\footnotesize Het oude kamp in 1977}
\end{figure}

Later heb ik nog wel intensief met een gezin op het nieuwe kamp te maken gehad. Zij wilde scheiden en hij niet (familie-eer enzo). Ik meen dat er in goede harmonie een oplossing voor is gevonden toen er een huisje aan de Westzanerdijk vrij kwam, vlak bij het nieuwe kamp. 

Harm en Ilse (schuilnamen) waren een typisch voorbeeld van een stel dat niet met en niet zonder elkaar kon. 

Het begon altijd met een bezoek aan de dienst omdat een deurwaarder hen bezocht had. Geen geld, wel schulden, veel alcohol.

\begin{figure}
\centering
\includegraphics[width=\textwidth]{img/263pstraat}
\caption*{\footnotesize Het huis aan de Prinsenstraat toen de sloop begon.}
\end{figure}

Over de jaren zijn ze veel uit elkaar gegaan en weer bij elkaar gekropen, omdat alles beter was dan een stap in het onbekende. 

Telkens als Ilse weer met een opgezwollen gezicht langs kwam wilde ze scheiden, wilde ze weg. Als we dan stappen in die richting zetten begon ze langzaam terug te trekken. Bovendien moest alles in het geheim, want Harm hield iedereen goed in de smiezen. 

Op een nacht waren de stoppen doorgeslagen. Harm had na een ruzie het dak in de Prinsenstraat beklommen en zat daar ladderzat met een krat bier iedereen uit de slaap te houden. Toen de politie kwam begooide hij die met de lege flessen. Na deze woeste nacht besloot Ilse ermee te kappen. Dat werd makkelijk gemaakt omdat Henk op het bureau vast zat. 

Ik had een adres gevonden waar ze heen kon. Toen we op een van de volgende dagen van alles zaten de bespreken zagen we opeens Harm aan de overkant staan. Hij was weer vrij gelaten en had wel een vermoeden. Toen ie ook nog niet werd binnen gelaten rook ie onraad en bleef dus aan de overkant staan op wacht staan zodat Ilse (en wij) geen kant op konden. Opeens vond hij het genoeg en stormde het gebouw binnen, portier of niet, Harm eiste Ilse op. "Ze zit hier, ik weet, schreeuwde hij", en begon het gebouw te verkennen. 

Ik achter hem aan om hem er weer uit te werken. Hij eerst naar boven de statige trappen op en alles wat los zat gooide hij naar beneden. Toen hij de spurt naar de 1e verdieping nam werd Ilse snel het gebouw uit geloodst. Toen hij weer beneden was en alleen nog losse asbakken gooide konden we hem kalmeren, de kamers laten zien dat ze er echt niet was en geloofde hij ons nog niet. Ze zijn toen inderdaad uit elkaar gegaan. Met Harm, als vaste klant, ben ik nog beste vrienden geworden. Hij is daarna altijd alleen gebleven en later verhuisd omdat z’n buurtje gesloopt werd. Als ik hem tegenkwam, ook toen ik niet meer bij de Sociale Dienst werkte, was het altijd hand opsteken en gedag schreeuwen.

Er kwam een nieuwe directeur. Peter Postma verdween en Harry van der Veen kwam. Met een nieuwe directie kwamen ook de nieuwe plannen. Na de overgang kwamen er hulpsecretarieën in elke deelgemeente. Maar de organisatie was te groot en te log. 

\begin{figure}
\centering
\includegraphics[width=\textwidth]{img/ch42/desoos_0004}
\caption*{\footnotesize We deden ook wel leuke dingen met elkaar, hier een weekeinde zeilen op het IJsselmeer.}
\end{figure}

Iedereen raakte z’n baan officieel kwijt en kon solliciteren naar een plek binnen de nieuwe organisatie. Net als bij de overgang van Zaandam naar Zaanstad kwamen er dan mensen in de knel, vooral de oude garde. Dit was het geval met Gerrit Bakker. Ik vond dat hij schandalig werd behandeld. Dit, én het feit dat ik te diep in de drugscene terecht kwam, deed me besluiten te solliciteren naar een andere baan. 

Ik heb onder andere gesolliciteerd naar een plek op Texel waar ik uiteindelijk werd aangenomen. We hebben Texel een paar keer bezocht en gekeken naar woningen en besloten toen om van de baan af te zien. Het voelde als emigreren. Bij een volgende sollicitatie kwam ik terecht bij de Hulp voor Onbehuisden (HVO) in Amsterdam.

Ondertussen deed ik m’n HBO opleiding Maatschappelijk werk aan de sociale academie CICSA in Amsterdam. Ik heb dat altijd met veel plezier gedaan, met dank aan de gemeente, die mij in staat stelde de opleiding te volgen. Ook hier waren er weer veel interessante mensen die allemaal werkten en de opleiding volgden. 

Zo kon je nog eens bij een medestudent op het werk binnen lopen om te zien hoe het daar ging. Er waren in die tijd allerlei bewegingen die het gezag op school probeerden aan te tasten. Er waren docenten die daar volledig onderdoor gingen en anderen die recht overeind bleven. Zo had ik een scriptiebegeleider die nog in streepjespak en horlogeketting voor de klas stond. Hij was theoloog, vond het allemaal prima en bleef gewoon zichzelf. 

De psychotherapie waar we les in kregen---en welke we soms in de praktijk moesten leren kennen---velde slachtoffers. Zo herinner ik me een weekeinde waar we met z’n allen in een boerderij zaten en de therapeut z’n kunsten op ons losliet. 

Het eindigde ermee dat hij voortijdig vertrok omdat wij het met zijn meedogenloze wijze van optreden niet eens waren. Er werd gescholden, gejankt, nog net niet gevochten en op de tweede avond verdween hij.

\begin{figure}
\centering
\includegraphics[width=\textwidth]{img/ch42/4-9-2009_002}
\caption*{\footnotesize Ik sta er niet op omdat ik de foto nam. Max is de derde van links in ’t zwart. Voor hem in ’t zwart Liesbeth. Die grote blonde is Maarten, die in Amstelveen werkte. Joris ligt op de grond. Naast hem Nico. Max kwam ik in Haarlem nog wel eens tegen. Hij werkte daar voor de Reclassering. Nico heb ik wel eens in Amsterdam opgezocht.}
\end{figure}

\section*{Hulp voor Onbehuisden (HVO)} % (fold)
\label{cha:hvo}

De Roggeveen was een groot, oud gebouw dat diende als opvangcentrum voor vrouwen, kinderen en gezinnen in nood. De Roggeveen bood dagelijks hulp, onderdak, verzorging, ondersteuning, informatie en advies aan ongeveer tweehonderd mensen.

Daarnaast was er ook een crisisopvang in het gebouw. Een opvangvoorziening voor in een acute crisis verkerende vrouwen en één- of twee-oudergezinnen. Bij de crisisopvang bestaat het hulpaanbod uit het aanbieden van een tijdelijke opvangplek, het bezweren van de crisis en het creëren van veiligheid. Indien nodig wordt er ook op een vervolgtraject ingezet. De crisisopvang bestaat uit een unit voor vier à vijf personen.

\begin{figure}
\centering
\includegraphics[width=\textwidth]{img/268roggeveen}
\caption*{\footnotesize }
\end{figure}

Ik begon daar als maatschappelijk werker. De taak was om voor de vrouwen en kinderen die binnenkwamen een goede opvang te regelen zodat ze even tot rust konden komen en daarna naar een oplossing te zoeken, bijvoorbeeld huisvesting. 

Terwijl wij dat werk deden was de directeur bezig met een omvangrijke verbouwing van het pand. Daar stak hij zijn ziel en zaligheid in. Zo kon het gebeuren dat er een familie voor de deur stond waar hij afspraken mee had gemaakt, maar vergeten was dat aan ons mee te delen. Of er kwamen bouwvakkers die een afdeling onder handen wilden nemen terwijl die nog niet ontruimd of overgeplaatst was. 

Na een jaar had ik er genoeg van en nam ontslag. Als intermezzo ging ik een paar weken naar de Findhorn Foundation, nabij Inverness in Schotland. Simon Vinkenoog had erover geschreven en het leek ons wel wat. Het is een commune met een doel: ``Today, the Findhorn Foundation is more about growing `people' than huge vegetables. It is currently home to approximately 400 members from 40 nations who are striving to create a model of a positive vision for humanity and the planet.'' Olga en haar vriendin Lee gingen daar een jaar later naartoe.

Ik was nog maar net aangekomen toen er een vrachtwagen met een achterbak vol mensen voorbij reed. Ze riepen dat ze naar zee gingen, of er iemand mee wilde. Ik sprong er in en zo stonden we aan zee. Een groot deel sprong de zee in, dus ik in m’n onderbroek erachteraan. Maar wat was dat koud zeg. Volgens mij was het de Noordelijke IJszee. Op de terugweg moest iedereen zo veel mogelijk platte stenen van het strand meenemen naar huis. Daarmee werd dakbedekking gemaakt voor de grote hal die nog in aanbouw was. 

\begin{figure}
\centering
\includegraphics[width=\textwidth]{img/ch43/hall}
\caption*{\footnotesize Findhorn. De grote hal in aanbouw.}
\end{figure}

Het waren aangename weken. Ik woonde als betalende gast in het kasteel, naast een oud militair die de gewoonte had ’s morgens voor het open raam z’n oefeningen te doen en daarna luid een militair lied aan te heffen. 

Een jongen uit Schotland was de hele wereld over geweest. India, Nepal, enzovoorts. Hij zei dat het wonderlijk was dat wat hij altijd had gezocht bijna in z’n achtertuin gevonden werd. Iedereen had wel iets waarvoor ze kwamen. Velen hadden belangstelling voor de wonderbaarlijke tuinen, anderen meer voor de spirituele ontwikkeling. In de werkuren heb ik een tuinhuisje geschilderd. 

\begin{figure}
\centering
\includegraphics[width=\textwidth]{img/ch43/group}
\caption*{\footnotesize Onze groep. Ik zit tussen een echtpaar uit Nederland. Hun dochter staat recht achter mij.}
\end{figure}

Ik kwam in contact met een echtpaar uit Nederland en hun dochter. Toen de periode eindigde moest de dochter weer naar school, terwijl het echtpaar besloot langer te blijven. Ze vroegen mij om hun dochter mee te nemen naar Londen. We hebben een tussenstop gemaakt, waarna zij in Londen op de trein naar de boot is gestapt.

Na terugkeer zat de geest van Findhorn nog in mij en Olga. Ik had het werk bij de de Hulp voor Onbehuisden opgegeven. De tijd was rijp voor iets nieuws. Lee nodigde ons uit om mee te gaan naar Scientology in Amsterdam, waar zij een test had gedaan. We waren meteen verkocht. Na een tijdje besloot ik er te gaan werken en tekende een contract voor vijf jaar. 

\section*{Scientology} % (fold)
\label{cha:scientologu}

Het beginnen bij Scientology was een stap in het onbekende. Olga deed cursussen en ik werkte er. Mijn eerste job was, na wat training, \emph{Course Admin}. 

\begin{figure}
\centering
\includegraphics[width=\textwidth]{img/ch44/NZVBwal_0006}
\caption*{\footnotesize Hier op de studiezaal}
\end{figure}

De rol van course admin betekende in de praktijk dat ik hoofd van de studiezaal was waar iedereen die er studeerde de cursussen volgde. Ik heb er ontzettend veel Volendammers zien komen en gaan. Iemand van BZN, een voetballer van Ajax, en later een jurist. Ze kwamen allemaal uit Volendam naar de Nieuwezijds. Je moest wekelijks statistieken bijhouden om te zien hoe goed het ging. Dat zouden meer bedrijven moeten doen.

Ik zorgde voor de materialen, repareerde ze ook, hielp studenten die vast liepen en dergelijke taken. Zelf deed ik ook cursussen. Ik werd zelfs nog \emph{Minister} (dominee) zodat ik de gemeente op zondag kon voorgaan (wat ik inderdaad nog eens gedaan heb). Je verdiende er niks tot bijna niks. Het meeste geld ging naar de organisatie en wat er overbleef werd verdeeld. 

Voor de course admin-functie werd ik ook nog naar een cursus in Kopenhagen gestuurd waar de Sea Organization (kortweg `Sea Org') zat; een hogere organisatie. Bij de Sea Org lopen ze allemaal in uniform. Naast de studie werkte ik ook in het restaurant van Scientology, als afwasser.

\begin{figure}
\centering
\includegraphics[width=\textwidth]{img/ch44/9-25-2009_002}
\caption*{\footnotesizePiet De zondagsdienst}
\end{figure}

Mijn langstlopende taak was het opmaken en drukken van het blad \emph{Theta}. De materialen kwamen uit Kopenhagen of Los Angeles. Ik vertaalde het en maakte het blaadje op. Alles met knippen en plakken. Dan bracht ik het naar de drukker en kreeg het later van hem in een pakketje weer retour. 

Bij \emph{Theta} volgde Michel Matil op die het al een aantal jaren deed. Ik zat veel bij hem vanwege de gezelligheid. We luisterden stiekem ookwel naar tapes van Bhagwan die hij had meegenomen. Dat was natuurlijk tegen alle regels. Toen Michel vertrok---hij ging pedagogiek studeren---nam ik het van hem over omdat ik de procedure kende. 

Er was voor mij toen een eigen kantoortje in de kelder. Soms deed ik nachtdiensten, dat hoorde bij de functie. Ik had m’n radio (BBC World Service), luisterde naar tapes van L. Ron Hubbard en nachtprogramma’s. 

Toen ik op een nacht lekker lag te pitten in de grote zaal werd er door vandalen een raam ingegooid. Dat was een raam van drie bij vijf, dan gaat er heel wat kapot. Ik kreeg zowat een hartverlamming van de knal. Als eerste bracht ik mezelf in veiligheid, om later schoorvoetend kijken wat er gebeurd was en hoe groot de ravage was.

Op een dag kwam er een oekaze vanuit het hoofdkwartier dat iedereen een \emph{drug rundown} moest gaan doen om al het gif uit de lichamen te krijgen. Deze zouden het \emph{clear} worden in de weg zitten. Dus werd iedereen op een programma gezet van hardlopen, vitaminen en sauna. Door het zweten kwamen alle geurtjes van de vitaminen naar buiten. Nog ruik ik bij bepaalde vitaminen die periode. 

In die tijd heb ik ook Brian Eno ontdekt. In de sauna werd prima muziek gedraaid. Toen ik vroeg wie dat was kreeg ik te horen ``Brian Eno met \emph{Music for Films}''.

Op de werkvloer van Scientology in Amsterdam was een grote saamhorigheid onderling. Die werd natuurlijk versterkt door de antipathie en tegenwerking van de buitenwacht. Maar de organisatie deugde niet. Ik had er genoeg van. Het had veel geld gekost en ik ging na die vijf jaar niet verder. Het was tijd om weer wat te gaan doen.

\section*{De Zaanlander} % (fold)
\label{cha:zaanlander}

In de krant las ik een oproep voor muziekrecensenten. Met mijn achtergrond van muzikant leek me dat wel wat. Een paar keer heb ik dat gedaan, maar het was helemaal niet leuk, want ik vind dat iedereen die op een podium gaat staan te bewonderen is. Dat gold ook voor toneelvoorstellingen. Toen kwam de vraag of ik overdag iets kon doen. Zo rolde ik bij \emph{De Zaanlander} binnen. 

Bij \emph{De Zaanlander} heb ik van alles gedaan, zoals het verslaan van raadsvergaderingen, openingen van een nieuw bedrijf, aanbieden van een nieuwe brandspuit, enzovoorts. Je leert politici kennen en ze willen je dan van alles vertellen of soms op de mouw spelden. 

Bij een vergadering van het bestuur van het hoogheemraadschap fluisterde wethouder Wim Nieuwenhuisen (CPN) dat hij een bommetje zou gooien in het bestuur, en dat ik vooral zou moeten letten op de reactie van de wethouder uit Edam. Vooral met de bedoeling dat het in de krant zou komen. Ook ging ik wel naar de rechtbank en het kantongerecht, waar je na afloop nog even aan de rechters om toelichting kon vragen. Heel interessant. 

Ik had het geluk dat er van de redactie één persoon langdurig ziek was en er een ander minstens een jaar op wereldreis ging. Een vast contract zat er echter niet in, omdat in de CAO stond dat je tenminste de school voor journalistiek moest hebben afgerond. Langzamerhand kreeg ik bij \emph{De Zaanlander} steeds minder werk als freelancer. Met teruglopende inkomsten heb ik toen toch maar besloten om verder te kijken.

Heel veel heb ik gesolliciteerd in de sociale sector, waarin ik het meest ervaring had. Al snel liep ik op tegen het feit dat ik bij Scientology had gezeten. Daar wilde men niets mee te maken hebben, zelfs niet als ik solliciteerde bij mensen waar ik vroeger mee had samengewerkt. De periode Scientology liet ik daarom voortaan wijselijk achter het rookgordijn van het journalist-zijn voor \emph{De Zaanlander}.

\section*{Gemeente Velsen} % (fold)
\label{cha:velsen}

Gelukkig had ik beet bij de gemeente Velsen. Mijn sollicitatie was wat ongelukkig, omdat ik m’n brief in plaats van `drugs' het woord `drigs' had gebruikt; een typefout. Alles ging nog op de ouderwetse typemachine, zonder spellingscontrole. Men vond dat wel grappig en door het gesprek viel de voorkeur op mij. Ik was bijzonder blij met deze aanstelling, en in mei 1984 kon ik hier aan de slag.

Met m’n Datsun---wat een lekker karretje---reed ik nu dagelijks naar IJmuiden, onderdeel van de gemeente Velsen. Er zijn vast vele verhalen over deze periode te vertellen, maar de meeste zullen onbesproken blijven omdat ik die gewoon vergeten ben. Alleen de bijzondere gebeurtenissen blijven in de geest aanwezig.

In Velsen kwam eens een man op spreekuur met een koffer. Ongeveer eind vijftig, grijze kop met haar, gebruind uiterlijk. Hij zet z’n koffer op het buro, klikt het deksel los, opent de koffer en begint meteen te huilen. Hij wist het niet meer. Z’n koffer zat vol met brieven, rekeningen, dwangbevelen. 

Het laatste bevel had hij nog in de hand, want daarin stond dat hij z’n huis uit moest. Eerst maar eens zorgen dat hij in z’n huis kon blijven. Daarna heb ik orde op zaken gesteld, schulden geregeld, enzovoorts. Maar ook z’n gezondheid liet te wensen over. Hij dronk ook veel te veel, kreeg de verkeerde mensen over de vloer die zijn bier opdronken. 

Op een keer kreeg ik een telefoontje van de buurvrouw, dat-ie met een kop vol bloed in de straat liep. Ze had de GGD al gebeld. Bleek hij epilepsie te hebben en z’n medicijnen te verwaarlozen. Ook had hij astma. 

Als hij weer eens in het ziekenhuis lag ging ik wel even informeren hoe het was. Dat vond hij zo fijn, dat ik zomaar kwam. Het zijn de kleine dingen die het doen. Zo help je iemand beetje bij beetje, vooral met veel vallen en opstaan, om z’n leven weer wat vorm te geven.

Een heel lang verhaal is dat van twee dames, Janna en Sophie, die samenwoonden (van de damesliefde). Janna had twee kinderen, maar die waren aan haar man toegewezen. Sophie had in kindertehuizen gezeten, de Rekkense inrichtingen. Daarvan werd later bekend dat er een schrikbewind werd gevoerd, waarbij vele kinderen (jongens en meisjes) verkracht werden door het personeel, inclusief de directeur. 

Nu hadden ze samen een flat gekregen en vroegen om hulp om het in te richten. Met dit stel ging het heel langzaam maar zeker mis. Hoe je ook probeerde enige redelijkheid bij ze te bereiken, ze wilden altijd meer. Nooit waren ze tevreden met wat ze konden krijgen. 

Sophie pluisde alle wetboeken na, las veel. Ze had voortdurend het gevoel benadeeld te worden, ook al hielp je ze nog zo goed mogelijk. Ze fokten elkaar op en kwamen dan over hun toeren bij de dienst, omdat wij ze weer wat geflikt zouden hebben. 

Tegelijkertijd hierden ze zich zelf nauwelijks aan afspraken. Steeds grover druggebruik en gekkere dingen. Ze hadden geen cent te makken, maar namen opeens twee grote honden in hun flat, gevolgd door een terrarium met slangen. Dan wilde Janna haar kinderen weer terug, dan wilde Sophie geld voor kunstmatige inseminatie. Soms vlogen de stoelen door de spreekkamer. Op andere momenten hadden we politie aan de lijn die ze moest beschermen tegen de buurt. 

Toen ik al lang iets anders deed heb ik de afloop van het verhaal gehoord. Ze kregen een laagbouw duplexwoning met een schuurtje. Daar kregen ze contact met een buurtbewoner die ze elke maand z'n uitkering hebben afgetroggeld. Uiteindelijk hebben ze de man gemarteld, vermoord en in het schuurtje onder cement begraven. Ze zijn veroordeeld. 

\begin{figure}
\centering
\includegraphics[width=0.7\textwidth]{img/ch46/krant}
\caption*{\footnotesize Hier een krantenartikel, \emph{De Volkskrant} 24 november 2000}
\end{figure}

\begin{figure}
\centering
\includegraphics[width=\textwidth]{img/ch46/crew}
\caption*{\footnotesize Gemeentelijke Sociale Dienst (GSD) Velsen, juli 1984, team Carpe Diem (Pluk de Dag), met Guusta Wegman, Theo Hoek, ikzelf, Sietske ?, Hans van der Heijden. Syll Damave nam de foto.}
\end{figure}

Iemand anders die bij ons kwam was een mevrouw uit Santpoort Noord---zwak begaafd, getrouwd, drie kinderen en nu gescheiden. Ze woonde bij een man in huis die haar, zei ze, misbruikte. Deze man had haar probleem wel bij ons aanhangig gemaakt en gevraagd of wij een eigen woning voor haar konden regelen, want hij kon het niet meer aan. Heel dubbel allemaal. 

We vonden een woning. Vanwege haar zwakbegaafdheid moest er controle komen van de geestelijke gezondheidszorg (GGZ) en maatschappelijk werk. Na een paar weken bleek dat ze de kluts alweer helemaal kwijt was en een zelfmoordpoging had gedaan. Ze werd opgenomen in Duin \& Bosch in Castricum, waar ik haar regelmatig bezocht. 

Toen ze na een klein jaar weer terug mocht heb ik weer een flatje voor haar geregeld. Twee weken nadat ze geïnstalleerd was ging ik bij haar op huisbezoek. Toen ik de straat inreed zag ik nog net een ziekenwagen vertrekken. Ze zat erin, zeiden de buren. Ik er achter aan. Bij het ziekenhuis liep ik met de brancard op om haar mee terug te nemen: ze moest niet het ziekenhuis in, maar weer terug naar Duin en Bosch. Zo geschiedde. Soms moet je heel brutaal zijn.

Ik heb altijd open gestaan voor nieuwe dingen. Toen de volstrekt nieuwe wijk Velserbroek werd gebouwd gaf ik aan die wijk wel te willen beposten. Operatie Velserbroek werd gestart met een wijkgezondheidscentrum. Terwijl er nog geen straten waren aangelegd verstrekten wij al hulp aan net gearriveerde bewoners. Er was een huisarts, tandarts, maatschappelijk werker, iemand van de gemeentelijke sociale dienst, wijkverpleging en apotheek, allemaal in één gebouw. Veel vrouwen met kinderen na scheidingen kwamen daar terecht. Dat heb ik ongeveer twee jaar gedaan. 

Het werd duidelijk dat als mensen niet aan het werk komen ze eeuwig in de bijstand blijven. Daar moest meer aan gebeuren. De overheid lanceerde het Jeugdwerk-Garantieplan (JWG) en de Banenpool. Beide zijn overheidsregelinge die jonge- of langdurig werklozen met subsidie aan het werk moeten helpen, meestal in eerste instantie bij de overheid of semi-overheid. Ik meldde me aan en kreeg de job als degene die de kar ging trekken. Zo was ik eigen baas. 

Arbeidsbemiddeling bij de Sociale Dienst. Ik kreeg een kamertje, een PC en begon te inventariseren wat we allemaal aan niet-werkenden in huis hadden. De wethouders van IJmond gingen verder, ze wilden het regionaal aanpakken. We kregen vaste contacten met het arbeidsburo Beverwijk. De samenwerking moest geïntensiveerd worden. Zo werd besloten werd dat ik vanuit het arbeidsburo parttime zou gaan werken voor het project Ruimbaan.

\begin{figure}
\centering
\includegraphics[width=\textwidth]{img/ch46/velse_0001}
\caption*{\footnotesize Hier met Ron, m’n chef, op bezoek bij een project in Oud-IJmuiden.}
\end{figure}

`Ruimbaan' was de naam voor het samenwerkingsproject van de gehele IJmond inzake arbeidsbemiddeling. Ik bleef drie dagen in Velsen en ging twee dagen in de week naar Beverwijk. In een latere fase kwamen er ook mensen uit de andere gemeenten bij. 

De samenwerking met andere instanties kreeg een nieuwe impuls toen we de mogelijkheid hadden om met de Vrijwilligerscentrale Velsen samen in één gebouw te komen werken. Ook Start Uitzendburo en het GAK (Gemeenschappelijk Administratiekantoor, de uitkeringsinstantie voor ziektewet enzo, nu het UWV) kochten zich in in de samenwerking. Overleg tussen deze instanties vergemakkelijkte de toeleiding. Ik opereerde als schakel tussen alle deze partijen, het Arbeidsburo IJmond en de GSD Velsen. 

Daarnaast zat ik nog bij het regionaal overleg, de stichting die alle Banenpoolers in dienst nam namens de gemeenten. Het werken op zoveel verschillende plekken had ook nadelen, ik had bijvoorbeeld geen echte thuisbasis meer. Toen het nieuwe stadhuis in IJmuiden klaar was moest iedereen weer terug naar de basis. 

Ik deed alle werklozen die drie jaar of langer werkloos waren, via de Banenpool en het Jeugdwerk-Garantieplan. Later kreeg ik assistentie van een collega en een JWG’ster. Ik had onder meer taken voor het JWG overgedragen. Ik behield de langdurige werklozen en projecten zoals het Groenteam bij de Reiniging. 

In de tijd besloten veel gemeenten net om te gaan privatiseren. Er was een afspraak met de wethouder dat gemeentelijke diensten een taak hadden in het opnemen van langdurig werklozen. Er werkten er toen vier bij de Gemeente Reiniging, twee bij Sociale Zaken en een aantal als conciërge bij scholen. De medewerking van de gemeente stond op het spel. 

\begin{wrapfigure}{r}{0.5\textwidth}
\begin{center}
\includegraphics[width=0.48\textwidth]{img/ch46/velsen_0002}
% \caption*{\footnotesize }
\end{wrapfigure}

Ik had het gevoel dat alles langzaam maar zeker in elkaar donderde en dat het aantal langdurig werklozen alleen maar groeide. Dweilen met de kraan open, zoals dat heet. In die jaren fietste ik elke dag van Zaandam naar IJmuiden, via Buitenhuizen. Op een ochtend---ik stond klaar, in fietstenue, om met de fiets naar IJmuiden te rijden---had ik er genoeg van. Ik belde Ron, mijn chef, en vertelde dat ik er mee kapte. Dat ik zo niet verder wilde. Ik werd op non-actief gesteld. Ik wilde wel werken, maar niet meer in deze klus bij Sociale Zaken.

Het grote probleem is de betrokkenheid bij de cliënten. Je komt bij iemand of bij een gezin dat in de problemen zit. Als het alleen om een uitkering gaat kom je niet bij iemand thuis. Je bent er en men legt je het hele hebben en houwen voor. Vaak ga je dan een relatie aan met die mensen om te ze uit de brand te helpen. Het is een afhankelijkheidsrelatie. Mensen zijn geneigd om alle hoop op jou te vestigen en het daarbij te laten. Dat klinkt lekker, maar het is de achterdeur uit. Soms moet je ze stevig wijzen op hun eigen verantwoordelijkheid om uit te sores te komen. 

Het geluk was aan mijn kant toen de personeelsfunctionaris bij me langs kwam en vertelde dat er een vacature kwam bij de archiefdienst van de gemeente Velsen. Of ik daar interesse in had? Dat had ik zeker wel. Na een gesprek met het hoofd en de archivaris kon ik daar beginnen. Wel in Den Haag bij het Rijksarchief een cursus doen voor archiefmedewerker. 

\section*{Archief} % (fold)
\label{cha:archief}

\begin{figure}
\centering
\includegraphics[width=\textwidth]{img/285Janskerk.jpg}
\caption*{\footnotesize Janskerk, onder het dak met trapgeveltje was mijn werkplek}
\end{figure}

Er ging een wereld voor me open. Tot op heden ben ik blij dat ik die stap heb gedaan. Het heeft een wereld voor me geopend die ik nooit voor mogelijk had gehouden. Mijn kijk op het verleden kreeg een gans ander perspectief, want hoe vaak kom je als burger in aanraking met notarisgegevens uit 1600, of archieven van clubjes uit 1816?

IJmuiden was gesticht met de opening van het Noordzeekanaal, maar Oud-Velsen en Santpoort hadden een veel langere geschiedenis. Later werden de archieven van Heemskerk er aan toegevoegd. Velsen had een beperkte studieruimte, maar wel interessante bezoekers. 

De geschiedenis van deze duinstreek is al heel oud. De ontginning van het moerasgebied naar het oosten is bijvoorbeeld vanuit hier begonnen. In dat oosten lag de Zaanstreek. 

Het duingebied was voor de rijken uit Amsterdam een mooie omgeving om een `buiten' neer te zetten en daar werd veel onderzoek naar gedaan. Maar ook een schrijfster als Conny Braam, die over de geschiedenis van Oud IJmuiden heeft geschreven, kwam langs. Ze deed bij ons onderzoek voor deze serie boeken.

Hoe meer je weet, hoe interessanter het wordt. Toen de gemeente Velsen plannen voor nieuwbouw had en het archief te klein werd kwam men voor de vraag te staan: zouden ze zelf een kostbare archiefruimte bouwen, of fuseren met Haarlem waar al veel archieven van gemeenten uit de omgeving lagen? 

Het werd het laatste. We gingen naar Haarlem! In vergelijking met Velsen was Haarlem een walhalla. Het is één van de oudste steden van Holland. Er is een rijke geschiedenis bewaard gebleven in het archief. Ook in de stad zelf is nog het één en ander uit vroeger tijd te bewonderen. Inspirerend om daar te werken. 

\begin{figure}
\centering
\includegraphics[width=\textwidth]{img/ch47/afscheid}
\caption*{\footnotesize Afscheid in Velsen, hand en bloemen van de wethouder. Op de achtergrond Jan Suurmond, archivaris van Velsen.}
\end{figure}

Toen ik er kwam waren de archieven binnen de kerk zelf in een enorme stelling van drie verdiepingen (met lift) opgeslagen. Mijn buro stond op een kamertje in het kleine huisje aan de straatkant. Ik deelde een kamer met Stinie en Marjan.

Stinie was verantwoordelijk voor de verhuizing uit Velsen als aparte klus. In Haarlem deed zij de aquisitie van archieven. Marjan was naast archivaris ook ongehuwd moeder en schrijfster van middeleeuwse misdaadverhalen.

Het was een prima plek en prima collega’s waar ik met plezier heb gewerkt. (Overigens waren alle collega’s prima mensen.)

\begin{figure}
\centering
\includegraphics[width=\textwidth]{img/ch47/Stinie}
\caption*{\footnotesize Stinie met achter haar een wand met door mij gekozen foto's.}
\end{figure}

Steeds meer gemeenten besloten hun archieven bij ons onder te brengen. De naam veranderde in Archiefdienst voor Kennemerland. Gelukkig wilden de buren---een verzorgingstehuis---ook uitbreiden, zodat uiteindelijk een nieuwe archiefruimte onder de grond kwam te liggen. 

Dit alles gebeurde dankzij en door de bezielende leiding van Lieuwe Zoodsma, het hoofd van de dienst. Een prima vent die eerder het archief van Middelburg onder zich had gehad.

De kerk werd gestript, dat wil zeggen, het hele bouwwerk van drie verdiepingen verdween en de lege ruimte ontstond. De archieven kwamen in de kelder, het personeel zat gedurende de verbouding in nieuwe ruimtes naast de kerk met een provisorische trap naar boven. Als ik nu zie wat ervan de kerk gemaakt is het niet te geloven. Prachtige stijlkamers, marmeren vloeren.

Ik kreeg nog de gelegenheid om echt archivaris te worden via een opleiding van één jaar in het Rijksarchief in Den Haag, maar na een paar maanden hield ik het voor gezien. Mijn hart lag er niet. Wat mijn carrière in Haarlem ook geen goed heeft gedaan is dat ik geen enkel intern overleg meer bijwoonde. Al dat geklets was ik meer dan zat. 

Lichamelijke klachten deden ook al geen goed. Eerst met mijn rug en daarna de ogen. Tenslotte ben ik een keer finaal door m’n rug gegaan tijdens het verwerken van archief in het depot. Soms zit in een doos iets van tien gram en soms iets van twee kilo, dat weet je meestal niet op het moment dat je het pakt. In overleg is toen besloten er maar een eind aan te maken. Het kwam het archief goed uit en mij ook. Ik kreeg eervol ontslag, een afscheidsfeestje en dat was dat. Mooie tijd gehad.

\section*{Vrijwilligerswerk} % (fold)
\label{cha:vrijwilligerswerk}

Nadat ik bij Haarlem gestopt was ben ik vrijwilligerswerk gaan doen. Voor de de Patiëntenraad Zaanstreek zat ik onder andere in het bestuur van de GGZ, afdeling Zaandam, in de Westzijde in Zaandam. En namens Zaandam zat ik dan weer in het hoofdbestuur van de GGZ-Dijk en Duin regio Zaanstreek/Waterland.

In Doopsgezind Zorgcentrum Mennistenerf ben ik ook met optekenen en vormgeven van levensverhalen van bewoners. Ook voor het levensverhalenproject van Stichting Welsaen verzorg ik geregeld de layout van verhalen voor mensen die dat niet zelf kunnen. Voor internetkrant \emph{De Zuidkanter} werk ik als redactielid en tevens verslaggever. Inmiddels heb ik ook een eigen website \emph{'zaanseverhalen.nl'} waarop ik artikelen plaats over mensen of gebeurtenissen die een link met De Zaanstreek hebben.

\end{document}
